\documentclass{article} % Dokumenttyp "article" für Artikel

\usepackage[utf8]{inputenc}  % UTF-8-Zeichencodierung
\usepackage[T1]{fontenc}     % Europäische Zeichenunterstützung
\usepackage[ngerman]{babel}  % Deutsche Sprachunterstützung
\usepackage{graphicx} % Am Anfang des Dokuments einfügen, um Grafikbefehle zu laden
\usepackage{caption}
\usepackage[colorlinks=true, linkcolor=blue]{hyperref}
\hypersetup{citecolor = blue}
\usepackage{sectsty}
\usepackage{cite}
\usepackage{ulem}

\usepackage[a4paper, left=3cm, right=3cm, top=2.5cm, bottom=2.5cm]{geometry} % A4-Format mit definierten Rändern

\usepackage{tikz}
\newcommand*\circled[1]{\tikz[baseline=(char.base)]{\node[shape=circle, draw, inner sep=1pt, text height=1.5ex, text depth=.25ex] (char) {#1};}}

\setlength{\footskip}{2cm}       % Abstand zwischen Text und Fußnotenlinie
\setlength{\skip\footins}{1cm}    % Abstand zwischen Fußnotenlinie und -text
\interfootnotelinepenalty = 10000   % Verhindert das Teilen von Fußnoten über Seiten
\setlength{\parindent}{0.0em} % Setzt die Einrückung auf 1em
\sectionfont{\LARGE}      % Setzt \section auf \Large (eine Stufe größer als Standard)
\subsectionfont{\Large}   % Setzt \subsection auf \large (ebenfalls eine Stufe größer)

\subsubsectionfont{\large}

\title{Zusammenfassung Sensorik und Sensorsysteme (TU Wien)}
\author{Omar Filip El Sendiouny \\ \texttt{e1616023@student.tuwien.ac.at}}
\vfill
\date{\today}

\begin{document}

% Titelblatt mit Bild
\begin{titlepage}
    \centering
    \vspace{2cm}
    {\Huge \bfseries Zusammenfassung Sensorik und Sensorsysteme (TU Wien) \par}
    
    \vspace{1cm}
    {\Large Omar Filip El Sendiouny \\ \texttt{e1616023@student.tuwien.ac.at} \par}

    \vspace*{2cm}

    % Bild hinzufügen, Breite anpassen nach Bedarf
    \includegraphics[width=0.75\textwidth]{images/mems-module-hero.png}
    
    % Bild hinzufügen, Breite anpassen nach Bedarf
    \includegraphics[width=0.75\textwidth]{images/mems-gears.png} 

    \vspace*{2cm} 
    
    \vfill
    
    {\large \today \par}
\end{titlepage}

\tableofcontents
\thispagestyle{empty}
\listoftables
\thispagestyle{empty}
\listoffigures
\thispagestyle{empty}
\newpage

\pagenumbering{arabic} % Beginnt die Nummerierung ab hier



% Einleitung --------------------------------------------------------------------------------------------------------------------------------------------------------------


\section*{Einleitung}

Diese Zusammenfassung der Vorlesung \textit{Sensorik und Sensorsysteme} (366.071) an der TU Wien dient als Lernhilfe und wurde sorgfältig erstellt. Nach dem Durchlesen dieser Zusammenfassung sollten Leserinnen und Leser in der Lage sein, die theoretischen Fragen, die derzeit prüfungsrelevant sind, zu beantworten. Sie ist zudem darauf ausgelegt, dass beim Lernen ein vertieftes Verständnis der vermittelten Inhalte entsteht. 

\vspace{1em}

Da trotz sorgfältiger Erstellung Fehler vorkommen können, sind Leserinnen und Leser herzlich eingeladen, auf mögliche Unstimmigkeiten zu achten und diese an \texttt{e1616023@student.tuwien.ac.at} zu melden. Alternativ können Korrekturen direkt im Repository auf GitHub vorgenommen werden: \url{https://github.com/ofes/sensorik-und-sensorsysteme}. Falls eine Zugriffsberechtigung benötigt wird, kann eine Anfrage an \url{https://github.com/ofes} gesendet werden. Gute Lernunterlagen sind mir besonders wichtig, da sie die Motivation beim Lernen erheblich steigern – daher bin ich für jedes Feedback dankbar.

\vspace{1em}

Im ersten Abschnitt dieser Zusammenfassung werden grundlegende Herstellungsmethoden der Mikrosystemtechnik erläutert. Anschließend werden ausgewählte MEMS-Sensoren (mikroelektromechanische Systeme) und ihre Funktionsweisen vorgestellt.

\vspace{1em}

Viel Erfolg beim Lernen!



% Pysical Vapour Deposition -----------------------------------------------------------------------------------------------------------------------------------------------


\clearpage
\section{Pysical Vapour Deposition} % PVD

PVD (Physical Vapour Deposition) umfasst alle \textbf{physikalischen Verfahren zur Abscheidung dünner Schichten aus der Dampfphase}. Dabei werden die Ausgangsmaterialien der Schichten durch physikalische Prozesse wie Erhitzen in die Dampfphase überführt und anschließend auf dem Zielobjekt abgeschieden. Die \textbf{typischen Schichtdicken} liegen \textbf{zwischen 1~nm und 1~µm} und können \textbf{in Ausnahmefällen bis zu 10~µm} erreichen \cite{keplinger2024}.



% Pysical Vapour Deposition -> Begriffsklärung PVD ------------------------------------------------------------------------------------------------------------------------


\vspace{1em}

\subsection{Begriffsklärung für PVD} % PVD Begriffsklärung

Dieser Abschnitt bietet eine Einführung in wichtige, häufig verwendete Begriffe. Es empfiehlt sich, zunächst die nachfolgenden Abschnitte zu lesen und bei Bedarf hier nach unbekannten Begriffen zu suchen.

\begin{itemize}
    \item \textbf{Monolage}: Eine Monolage (auch Monoschicht genannt) beschreibt eine Schicht, die nur aus einer einzelnen Lage von Atomen oder Molekülen besteht \cite{kittel2004}.
    \item \textbf{Freie Weglänge}: Die freie Weglänge eines Atoms oder Moleküls in einem Plasma oder Gas ist die durchschnittliche Strecke, die ein Teilchen (wie ein Atom, Ion oder Molekül) zurücklegt, bevor es mit einem anderen Teilchen kollidiert. Diese Weglänge hängt stark von der Teilchendichte und dem Durchmesser der beteiligten Teilchen ab \cite{kittel2004}.
    \item \textbf{Verdampfen}: Bezeichnet den Übergang eines Stoffes vom flüssigen in den gasförmigen Zustand \cite{kittel2004}.
    \item \textbf{Sublimieren}: Beschreibt den direkten Übergang eines Stoffes vom festen in den gasförmigen Zustand, ohne eine flüssige Phase zu durchlaufen \cite{kittel2004}.
    \item \textbf{Rezipient}: Vakuumkammer \cite{keplinger2024, ohring2002}.
    \item \textbf{Substrat}: Trägerschicht in der Mikroelektronik und Halbleiterfertigung. Der Begriff Substrat bezieht sich allgemein auf das Material oder die Oberfläche, auf der ein Herstellungs- oder Beschichtungsprozess stattfindet \cite{ohring2002}.
    \item \textbf{Wafer}: Dünne Scheibe aus einem Halbleitermaterial, wie z.B. Silizium, und dient als Substrat \cite{sze2006}.
    \item \textbf{Tiegel} (Aufdampfen): Ein Behälter, in dem das Aufdampfmaterial lokal geschmolzen wird, häufig aus Materialien wie Graphit, Keramik oder Wolfram, die hohen Temperaturen standhalten \cite{smith1995, ohring2002}. Beim Elektronenstrahlverdampfen (E-Beam-Verfahren) wird das Aufdampfmaterial im Tiegel erhitzt und verdampft, ohne dass der gesamte Tiegel erhitzt werden muss \cite{smith1995}. Siehe \autoref{fig:Tiegel und Schiffchen} b), c) und d).
    \item \textbf{Schiffchen} (Aufdampfen): Ein kleiner, meist bootförmiger Behälter (daher der Name), der ebenfalls zur Aufnahme des Aufdampfmaterials dient, jedoch direkt beheizt wird, häufig durch Widerstandsheizung. Schiffchen bestehen oft aus Materialien wie Molybdän oder Wolfram, da sie hohe Temperaturen aushalten können \cite{mattox2010handbook, ohring2002}. Siehe \autoref{fig:PVD Aufdampfanlage} und \autoref{fig:Tiegel und Schiffchen} a).
    \item \textbf{Target} (Sputtern): Materialquelle für die Schicht, die auf das Substrat abgeschieden werden soll \cite{keplinger2024}.
    \item \textbf{Inerte Gase}: Auch als \textbf{Edelgase} bekannt, sind chemisch reaktionsträge Gase, die unter normalen Bedingungen kaum mit anderen Elementen oder Verbindungen reagieren. Zu den Edelgasen zählen Helium, Neon, Argon, Krypton, Xenon und Radon. Diese geringe Reaktivität ist auf ihre stabile, vollständig besetzte Elektronenkonfiguration in der äußersten Schale zurückzuführen, die sie energetisch stabil und widerstandsfähig gegenüber chemischen Reaktionen macht \cite{atkins_physical_chemistry, zumdahl_chemistry, silberberg_chemistry}.
    \item \textbf{Gasentladung}: Bezeichnet einen physikalischen Prozess, bei dem ein elektrisches Feld in einem gasförmigen Medium zur Ionisation der Gasmoleküle führt, wodurch leitfähige Plasmen entstehen. Die dabei entstehenden freien Elektronen und Ionen ermöglichen den Stromfluss durch das Gas, was zu sichtbaren Leuchterscheinungen führen kann. Solche Entladungen treten beispielsweise in Leuchtstofflampen, Neonröhren und in der Plasmatechnologie auf \cite{prechtl2005grundlagen, boehler1994, hutchings2018}.
    \item \textbf{Ionen}: Elektrisch geladene Atome oder Moleküle, die durch den Verlust oder Gewinn von Elektronen entstehen. \textbf{Positiv geladene Ionen}, sogenannte \textbf{Kationen}, entstehen durch den Verlust von Elektronen, während \textbf{negativ geladene Ionen}, sogenannte \textbf{Anionen}, durch den Gewinn von Elektronen entstehen. Ionen spielen eine zentrale Rolle in chemischen Reaktionen, da sie die elektrische Ladung transportieren und Reaktionen ermöglichen, die die Bildung oder Trennung von chemischen Bindungen betreffen \cite{petrucci_general_chemistry, chang_chemistry, brown_chemistry_textbook}.
    \item \textbf{Radikale}: Radikale sind Atome oder Moleküle, die mindestens ein ungepaartes Valenzelektron besitzen, was sie extrem reaktiv macht \cite{atkins2010, feynman2011}. Da sie oft instabil sind, neigen sie dazu, mit anderen Molekülen oder Atomen zu reagieren, um eine stabile Elektronenkonfiguration zu erreichen\footnote{Im Kontext von Dünnschichtprozessen wie dem Sputtern werden Radikale oft durch das energiereiche Plasma gebildet, das Bindungen in Molekülen aufbrechen kann \cite{boehler1994, hutchings2018}. Diese Radikale können dann auf der Oberfläche der zu beschichtenden Materialien reagieren, wodurch die Schichtzusammensetzung und -struktur beeinflusst wird \cite{journal_reference}.}.
    \item \textbf{Valenzelektron}: Valenzelektronen befinden sich in der äußersten Schale eines Atoms und bestimmen die chemischen Eigenschaften eines Elements maßgeblich. Sie ermöglichen die Bindung mit anderen Atomen, wodurch Verbindungen entstehen. Edelgase besitzen volle Valenzschalen, was sie besonders stabil und reaktionsträge macht \cite{petrucci_general_chemistry, chang_chemistry, brown_chemistry_textbook}.
    \item \textbf{Fraktionale Verdampfung von Legierungen} (Aufdampfen): Tritt auf, wenn die unterschiedlichen Siedepunkte der Metallkomponenten dazu führen, dass zunächst die Elemente mit niedrigerem Siedepunkt verdampfen, wodurch die Zusammensetzung des verbleibenden Materials verändert wird \cite{harvard_mrsec, vem_guide}. Dadurch können die Dampfphasen eine andere Zusammensetzung als die ursprüngliche Legierung aufweisen, was zu inhomogenen Beschichtungen führen kann \cite{frontiers_mpea, mdpi_vacuum_distillation}.
    \item \textbf{Dampfphasen}: Die gasförmigen Bestandteile, die entstehen, wenn feste oder flüssige Materialien durch Zuführung von Wärme in die Gasphase übergehen, wie es bei der Verdampfung und Sublimation der Fall ist \cite{thermopedia, thoughtco, opentextbc}.
    \item \textbf{UHV}: ``Ultra-High Vacuum'' bezeichnet ein Vakuum mit einem Druck von unter \(10^{-9}\) mbar \cite{atkins_physical_chemistry, ohring2002, smith1995}.
\end{itemize}



% Pysical Vapour Deposition -> PVD-Verfahren ------------------------------------------------------------------------------------------------------------------------------


\vspace{1em}

\subsection{PVD-Verfahren} % PVD Verfahren

Die verschiedenen PVD-Verfahren \textbf{unterscheiden sich} hauptsächlich \textbf{in der} Methode zur \textbf{Erzeugung des Dampfstrahls} \cite{keplinger2024}:

\begin{enumerate}
    \item \textbf{Thermisches Verdampfen (Aufdampfen)}
    \item \textbf{Sputtern (Zerstäubung)}
    \item \textbf{Laserstrahlverdampfen (pulsed laser deposition, PLD)}
    \item \textbf{Molekularbeamepitaxie (MBE)}
\end{enumerate}

Im Folgenden werden lediglich die ersten beiden Methoden erläutert.



% Pysical Vapour Deposition -> Aufdampfen ---------------------------------------------------------------------------------------------------------------------------------


\vspace{1em}

\subsection{Aufdampfen} % PVD Aufdampfen

Das thermische Aufdampfen gehört zu den ältesten Methoden zur Herstellung dünner Schichten (\autoref{fig:PVD Aufdampfanlage}). Dabei wird das \textbf{Material der Aufdampfquelle auf} eine Temperatur von \textbf{500 bis 3000°C erhitzt}, bis es einen ausreichend hohen Dampfdruck\footnote{Ein hoher Dampfdruck ist beim Aufdampfen entscheidend, da er sicherstellt, dass genügend Atome oder Moleküle des Materials in die Gasphase übertreten und eine dampfartige Wolke bilden. Diese Wolke ermöglicht eine ausreichende Partikelzufuhr zum Zielsubstrat, wodurch sich eine gleichmäßige, dünne Schicht bildet. Bei zu geringem Dampfdruck wären nur wenige Teilchen verdampft, was zu einer ungleichmäßigen und ineffizienten Beschichtung führen würde.} entwickelt \cite{keplinger2024}.

\vspace{1em}

Meist schmilzt das Material dabei, doch einige Stoffe, wie beispielsweise Chrom, erreichen bereits vor dem Schmelzpunkt einen ausreichend hohen Dampfdruck und sublimieren, indem sie direkt aus dem festen Zustand verdampfen \cite{keplinger2024}.

\vspace{1em}

Der \textbf{entstehende Dampf} breitet sich aus und \textbf{kondensiert an} allen \textbf{kälteren Oberflächen} in der Vakuumkammer, \textbf{einschließlich des Substrats}. Der Prozess findet im \textbf{Vakuum bei Drücken} von typischerweise \textbf{unter $\mathbf{10^{-6}}$~mbar} statt \cite{keplinger2024}.

\vspace{1em}

Dabei ist die \textbf{freie Weglänge der Teilchen deutlich größer als} die \textbf{Distanz zwischen} der \textbf{Aufdampfquelle und} dem \textbf{Substrat}. Dies führt dazu, dass die \textbf{Teilchen nur selten mit} dem \textbf{Restgas kollidieren und sich geradlinig von der Quelle zum Substrat bewegen} \cite{keplinger2024}.

\vspace{1em}

Die \textbf{kinetische Energie der Teilchen, die auf} der \textbf{Substratoberfläche kondensieren}, beträgt etwa \textbf{0,1 bis 0,5~eV}, was im Vergleich zum Sputtern (bis zu 100~eV) \textbf{sehr gering} ist \cite{keplinger2024}.

\begin{figure}[htb!]
    \centering
    \includegraphics[width=.9\textwidth]{images/PVD Aufdampfanlage.png} % Pfad und Dateiname des Bildes angeben
    \captionsetup{labelfont=bf} % Setzt die Bildnummer fett
    \caption{Schematischer Aufbau einer PVD Aufdampfanlage \cite{keplinger2024}.}
    \label{fig:PVD Aufdampfanlage}
\end{figure}



% Pysical Vapour Deposition -> Aufdampfen -> Funktionsweise ---------------------------------------------------------------------------------------------------------------


\vspace{1em}

\subsubsection{Funktionsweise}

Basierend auf \autoref{fig:PVD Aufdampfanlage} wird im Folgenden die Funktionsweise dieses Verfahrens erläutert \cite{keplinger2024}:

\begin{enumerate}
    \item \textbf{Vakuum}: Im Rezipienten wird mit dem Pumpsystem ein Vakuum erzeugt.
    \item \textbf{Aufheizen}: Das Aufdampfmaterial wird auf eine Temperatur von 500 bis 3000~°C erhitzt.
    \item \textbf{Öffnen des Shutters}: Sobald ein ausreichend hoher Dampfdruck erreicht ist, wird der Shutter weggeschwenkt, um die Bedampfung des Substrats zu ermöglichen. Wird der Shutter zu früh geöffnet oder ist er während des Aufheizens nicht vorhanden, kann es zu einer ungleichmäßigen Abscheidung kommen bzw. es können unerwünschte Oxide des Aufdampfmaterials auf das Substrat abgeschieden werden. Der Weg der Teilchen zum Substrat wird daher durch ein Schirmblech (Shutter) blockiert.
    \item \textbf{Abscheidung am Substrat}: Das Aufdampfmaterial kondensiert nun mit einer konstanten Rate sowohl auf dem Substrat als auch auf dem Schwingquarz. Der Schwingquarz dient zur Messung der Schichtdicke, indem Änderungen seiner Resonanzfrequenz erfasst werden.
\end{enumerate}



% Pysical Vapour Deposition -> Aufdampfen -> Schichtdickenmessung ---------------------------------------------------------------------------------------------------------


\vspace{1em}

\subsubsection{Schichtdickenmessung}

Zur Messung der Schichtdicke wird ein \textbf{piezoelektrisches Schwingquarzplättchen} verwendet, das \textbf{nahe\footnote{Da Schwingquarz und Substrat unterschiedliche Abstände zur Aufdampfquelle haben und sich zudem unterschiedlich nah an der Achse des Aufdampfstrahls befinden, ergeben sich unterschiedliche Aufdampfraten. Diese Abweichungen werden durch einen experimentell zu bestimmenden Geometriefaktor (Tooling-Factor) korrigiert.} am Substrat positioniert} ist. Durch Anlegen einer Wechselspannung wird das Quarzplättchen in eine (Scher-)Schwingung versetzt. \textbf{Wenn} das \textbf{Aufdampfmaterial auf} dem \textbf{Quarz kondensiert} und seine Masse zunimmt, \textbf{sinkt} die \textbf{Resonanzfrequenz} des Quarzes. Diese Abnahme der Frequenz ist \textbf{proportional zur Dicke der abgeschiedenen Schicht} und dient als Messgrundlage. Die Resonanzfrequenz startet typischerweise bei etwa 5~MHz und sollte, wenn sie auf 4~MHz fällt, zum Austausch des Quarzplättchens führen, um präzise Messwerte sicherzustellen \cite{keplinger2024}.



% Pysical Vapour Deposition -> Aufdampfen -> Erwärmung mittels Elektronenstrahlquelle (E-Beam-Verfahren) ------------------------------------------------------------------


\vspace{1em}

\subsubsection{Erwärmung mittels Elektronenstrahlquelle (E-Beam-Verfahren)}

Bei dieser Methode wird ein \textbf{Elektronenstrahl gezielt auf} einen \textbf{kleinen Bereich des Aufdampfmaterials fokussiert}, sodass das Material \textbf{lokal aufgeschmolzen} wird, während der Großteil fest bleibt (\autoref{fig:Tiegel und Schiffchen} b), c) und d)). Der \textbf{Tiegel} selbst wird dabei \textbf{nicht auf hohe Temperaturen erhitzt}, was das \textbf{Risiko einer Kontamination durch} das \textbf{Tiegelmaterial verringert und höhere Arbeitstemperaturen} als bei herkömmlichen Quellen, wie Schiffchenquellen, ermöglicht (\autoref{fig:Tiegel und Schiffchen} a)). Um eine \textbf{effektive Wärmeableitung} sicherzustellen, besteht der \textbf{Tiegel aus Kupfer} und wird direkt \textbf{vom Kühlwasser durchströmt}. Das \textbf{Aufdampfmaterial} muss \textbf{leitfähig} sein, da sich andernfalls elektrische Ladungen ansammeln könnten, die den Elektronenstrahl ablenken könnten \cite{keplinger2024}.

\vspace{1em}

Ein \textbf{Permanentmagnet in der Nähe der Elektronenstrahlquelle lenkt} den \textbf{Strahl um 270° ab, sodass} die \textbf{Quelle} selbst \textbf{nicht beschichtet wird} (\autoref{fig:Tiegel und Schiffchen} b) und c)). Die \textbf{Steuerung des Systems} ermöglicht zudem ein \textbf{Fokussieren oder Defokussieren des Strahls}, eine \textbf{präzise Positionierung über Spulen} sowie eine \textbf{kontrollierte Leistungsanpassung} bis zu 5~kW. Durch ein leichtes Wobbeln des Strahls kann ein größerer Bereich des Aufdampfmaterials überstrichen werden, was zu einem gleichmäßigen Erhitzen führt \cite{keplinger2024}.

\begin{figure}[htb!]
    \centering
    \includegraphics[width=.9\textwidth]{images/Tiegel und Schiffchen.png} % Pfad und Dateiname des Bildes angeben
    \captionsetup{labelfont=bf} % Setzt die Bildnummer fett
    \caption{%
    a) Aufdampfschiffchen, mit Aluminiumoxid (milchige Bereiche) beschichtet, um das Benetzen durch das Aufdampfmaterial zu minimieren. b) Schematischer Aufbau einer Elektronenstrahlquelle: S...Spule, A...Anode, K...Kathode, W...Wehneltzylinder, T...Tiegel mit Aufdampfmaterial, L...Wasserkühlung, M...Permanentmagnet. c) Elektronenstrahlquelle im Inneren der Vakuumkammer (Rezipient). d) Vierlochtiegel für die Elektronenstrahlquelle, der Beschichtungen aus bis zu vier verschiedenen Materialien ermöglicht 
    \cite{keplinger2024}.
    }
    \label{fig:Tiegel und Schiffchen}
\end{figure}



% Pysical Vapour Deposition -> Sputtern (Zerstäubung) ---------------------------------------------------------------------------------------------------------------------


\vspace{1em}

\subsection{Sputtern (Zerstäubung)} % Sputtern (Zerstäubung)

Der Sputterprozess ist ein \textbf{Abscheidungsverfahren}, bei dem \textbf{Atome aus dem Target}, der Materialquelle für die gewünschte Schicht, \textbf{durch energiereiche Ionen herausgeschlagen} und \textbf{auf} dem \textbf{Substrat abgeschieden} werden. \autoref{fig:Sputteranlage} zeigt den schematischen Aufbau einer Sputteranlage \cite{keplinger2024}.

\begin{figure}[htb!]
    \centering
    \includegraphics[width=.7\textwidth]{images/Sputtern.png} % Pfad und Dateiname des Bildes angeben
    \captionsetup{labelfont=bf} % Setzt die Bildnummer fett
    \caption{Schematischer Aufbau einer Sputteranlage \cite{keplinger2024}.}
    \label{fig:Sputteranlage}
\end{figure}



% Pysical Vapour Deposition -> Sputtern (Zerstäubung) -> Funktionsweise ---------------------------------------------------------------------------------------------------


\vspace{1em}

\subsubsection{Funktionsweise} % Funktionsweise

Anhand der \autoref{fig:Sputteranlage} wird die Funktionsweise der Sputteranlage näher erläutert \cite{keplinger2024}:

\begin{enumerate}
    \item In einem \textbf{inerten Gas}, typischerweise \textbf{Argon} mit einem \textbf{Druck von 0,2 bis 10 Pa}, wird \textbf{durch} eine \textbf{Gasentladung} ein \textbf{Plasma erzeugt}, das die \textbf{für} den \textbf{Sputterprozess benötigten Kationen ($\mathbf{Ar^+}$) bereitstellt}.
    \item Durch die \textbf{Spannung zwischen Plasma und Sputtertarget} (Schichtmaterial) werden die \textbf{Ionen zum Target hin beschleunigt} und \textbf{erreichen} dabei \textbf{Energien von einigen 10~eV}.
    \item \textbf{Trifft} ein \textbf{Ar-Ion} auf die \textbf{Targetoberfläche}, kommt es zu \textbf{elastischen und inelastischen Stößen mit} den \textbf{Targetatomen}. \textbf{Inelastische Stöße erzeugen Sekundärionen} (Sekundärelektronen) \textbf{für} das \textbf{Plasma}, während \textbf{elastische Stöße} eine \textbf{Stoßkaskade} auslösen, die \textbf{durch Impulsübertragung Atome aus} dem \textbf{Target herausschlagen} kann (\autoref{fig:Sputtern Stosskaskade}). Eine Energie von \textbf{mindestens 10 bis 30~eV} ist dafür \textbf{notwendig}, wobei die Sputtereffizienz mit steigender Energie bis zu einem Maximum von etwa 10~keV zunimmt. Danach sinkt die Effizienz, da die Ionen zu tief eindringen und weniger Energie an die Oberfläche übertragen wird. Die \textbf{Effizienz dieser Stoßkaskaden hängt} zudem \textbf{vom Massenverhältnis} zwischen \textbf{Sputterion und Targetatom ab}.
    \item Die \textbf{herausgeschlagenen Atome bewegen} sich \textbf{in Richtung des Substrats}. Auf ihrem Weg \textbf{stoßen} sie \textbf{mehrfach mit} den \textbf{Atomen des Prozessgases zusammen}, wodurch sie einen Teil ihrer \textbf{Energie verlieren und} ihre \textbf{Winkelverteilung breiter} wird.
    \item Die \textbf{Teilchen treffen auf} das \textbf{Substrat} (z.B. mit einer Energie von 10~eV) und \textbf{lagern sich dort ab}. Aufgrund der breiten \textbf{Winkelverteilung} werden dadurch \textbf{nicht nur} die \textbf{Oberseite}, sondern \textbf{auch} die \textbf{Seitenflächen der Strukturen auf} dem \textbf{Substrat beschichtet} (daher auch der Name ``Sputtern'' -- \textit{to sputter} = zerstäuben).
\end{enumerate}

\begin{figure}[htb!]
    \centering
    \includegraphics[width=.5\textwidth]{images/Sputtern Stoßkaskade.png} % Pfad und Dateiname des Bildes angeben
    \captionsetup{labelfont=bf, width=.6\textwidth} % Setzt die Bildnummer fett
    \caption{Herauslösen eines Targetatoms durch die vom Sputterion ausgelöste Stoßkaskade \cite{keplinger2024}.}
    \label{fig:Sputtern Stosskaskade}
\end{figure}



% Pysical Vapour Deposition -> Sputtern (Zerstäubung) -> Sputterverfahren -------------------------------------------------------------------------------------------------


\vspace{1em}

\subsubsection{Sputterverfahren} % Sputterverfahren

Im Folgenden werden ausgewählte Sputterverfahren näher erläutert und deren Funktionsweise beschrieben.

\vspace{0.0em}

\paragraph{\uline{DC-Sputtern:}} Hier wird die \textbf{Gasentladung mittels} einer \textbf{Gleichspannung} erzeugt. Daher muss das \textbf{Target leitfähig} sein, da es sich sonst auflädt und die Gasentladung unterbrochen würde. Diese Form des Sputterns \textbf{eignet sich} somit nur \textbf{für Metalle und Halbleiter} \cite{keplinger2024}.

\vspace{0.0em}

\paragraph{\uline{AC-Sputtern:}} Ein \textbf{hochfrequentes elektrisches Feld} wird \textbf{über} einen \textbf{Kondensator an} die \textbf{Elektroden angelegt}, wodurch ein \textbf{Gleichspannungsanteil blockiert} wird. \textbf{Ionen können} dem \textbf{Feldwechsel ab etwa 100~kHz nicht mehr folgen}, während \textbf{Elektronen weiterhin Energie aufnehmen} und \textbf{mehr Gasteilchen ionisieren}. Dadurch entsteht eine \textbf{höhere Plasmarate als beim DC-Sputtern}, und der \textbf{Druck kann bei gleicher Sputterrate auf etwa $\mathbf{10^{-1}}$ bis $\mathbf{10^{-2}}$~Pa gesenkt werden}\footnote{Da die herausgeschlagenen Targetatome aufgrund der geringeren Anzahl an Kollisionen mit den Prozessgasatomen weniger gestreut werden, erreichen sie das Substrat direkter. Dadurch werden die Seitenflächen der Strukturen auf dem Substrat weniger beschichtet als beim DC-Sputtern.} \cite{keplinger2024}. \\

Die \textbf{Vorteile} umfassen \cite{keplinger2024}:

\begin{itemize}
    \item \textbf{Auch Isolatoren} wie \( \mathrm{Al}_2\mathrm{O}_3 \) (Aluminiumoxid) und Halbleiter \textbf{können gesputtert (zerstäubt) werden}.
    \item Das \textbf{Substrat erhitzt sich weniger} als beim DC-Sputtern.
    \item \textbf{Sputterraten} sind \textbf{bis zu zehnmal höher} als beim DC-Sputtern.
\end{itemize}

Ein \textbf{Nachteil} ist der \textbf{höhere apparative Aufwand} \cite{keplinger2024}.

\vspace{0.0em}

\paragraph{\uline{Magnetronsputtern:}} Beim Magnetronsputtern sind \textbf{Permanentmagneten hinter} der \textbf{Kathode} angebracht, die ein Magnetfeld erzeugen (\autoref{fig:Magnetronsputtern}). Dieses bewirkt, dass sich die \textbf{Elektronen in Zykloidenbahnen bewegen} (\autoref{fig:Zykloiden}) und \textbf{länger in der Nähe des Targets bleiben}\footnote{Dies ist effektiv, wenn die Magnetfeldlinien parallel zur Oberfläche verlaufen. Liegen sie jedoch parallel zur Elektronenflugbahn (senkrecht zur Targetoberfläche), wirkt keine ablenkende Kraft auf die Elektronen.}, wodurch \textbf{mehr Gasatome ionisiert} werden und die \textbf{Plasmadichte steigt}. Dies \textbf{erhöht} die \textbf{Sputterrate und ermöglicht} es, den \textbf{Druck bei gleicher Sputterrate deutlich zu senken}, was \textbf{dichtere Schichten erzeugt}. Aufgrund der \textbf{kürzeren Beschichtungszeit} wird das Magnetronsputtern \textbf{in industriellen Anlagen häufig verwendet} \cite{kittel2004}. \\

\begin{figure}[htb!]
    \centering
    \includegraphics[width=.65\textwidth]{images/Magnetronsputteranlage.png} % Pfad und Dateiname des Bildes angeben
    \captionsetup{labelfont=bf} % Setzt die Bildnummer fett
    \caption{Schematischer Aufbau einer Magnetron-Sputteranlage \cite{keplinger2024}.}
    \label{fig:Magnetronsputtern}
\end{figure}

\begin{figure}[htb!]
    \centering
    \includegraphics[width=.75\textwidth]{images/Zykloiden.png} % Pfad und Dateiname des Bildes angeben
    \captionsetup{labelfont=bf} % Setzt die Bildnummer fett
    \caption{Zykloiden \cite{hoegel2024}}
    \label{fig:Zykloiden}
\end{figure}

Die Grundlage für diese Überlegung ist die Lorentzkraft, welche die Kraft beschreibt, die auf eine bewegte elektrische Ladung in einem elektrischen und magnetischen Feld wirkt. Diese Kraft wird durch
$$
\vec{F} = e (\vec{E} + \vec{v} \times \vec{B})
$$
beschrieben. Dabei stellt \( \vec{F} \) die Lorentzkraft in Newton dar, \( e \) ist die Elementarladung mit einem Wert von etwa \( 1{,}602 \cdot 10^{-19} \, \mathrm{C}\), \( \vec{E} \) das elektrische Feld in Volt pro Meter (V/m), \( \vec{v} \) die Geschwindigkeit der Ladung in Metern pro Sekunde (m/s), und \( \vec{B} \) die magnetische Flussdichte in Tesla (T). Das Kreuzprodukt \( \vec{v} \times \vec{B} \) zeigt, dass die Kraft \( \vec{F} \) senkrecht sowohl zur Bewegungsrichtung \( \vec{v} \) als auch zum Magnetfeld \( \vec{B} \) steht \cite{rossnagel2003magnetron, mattox2010handbook, anders2010advances, kouznetsov1999new, prechtl2005grundlagen}.

\vspace{0.0em}

\paragraph{\uline{Reaktives Sputtern:}} Obwohl \textbf{dem Sputterprozess mehr Materialien zur Verfügung stehen als dem Aufdampfen}, gibt es \textbf{dennoch Stoffe, bei denen sich eine Komponente ungenügend niederschlägt}. Durch die \textbf{Zugabe eines Reaktionsgases zum Plasma} können jedoch \textbf{chemische Reaktionen mit den Targetatomen in der Gasphase} eingeleitet werden, \textbf{sodass sich} die \textbf{gewünschten Verbindungen abscheiden}. Ein \textbf{Beispiel} ist die Bildung von \textbf{Aluminiumoxid und -nitrid}:
$$
\mathrm{2Al} + \mathrm{3O}_2 \rightarrow \mathrm{Al}_2\mathrm{O}_3
$$
$$
\mathrm{2Al} + \mathrm{N}_2 \rightarrow \mathrm{2AlN}
$$
Dieses Verfahren, bekannt als reaktives Sputtern, wird oft \textbf{zur Herstellung von Oxiden, Nitriden, Oxinitriden} (z.B., \(\mathrm{SiO}_x\mathrm{N}_y\)) \textbf{und Carbiden} verwendet. Es lässt sich zudem \textbf{mit anderen Sputterverfahren}, wie dem Magnetronsputtern, \textbf{kombinieren}. Ein \textbf{Vorteil} des reaktiven Sputterns besteht darin, dass die \textbf{Stöchiometrie\footnote{Stöchiometrie bezieht sich auf das Verhältnis der Atome oder Moleküle in einer chemischen Verbindung. In einem Material wie Aluminiumoxid (\(\mathrm{Al}_2\mathrm{O}_3\)) beschreibt die Stöchiometrie das Verhältnis von Aluminium- zu Sauerstoffatomen, das hier exakt 2:3 ist. Im Kontext des reaktiven Sputterns bedeutet das, dass die Stöchiometrie der abgeschiedenen Schicht -- also das Verhältnis der Elemente im abgeschiedenen Material -- durch den Druck des zugeführten Reaktionsgases beeinflusst werden kann. Das bedeutet, dass durch die Menge des Reaktionsgases im Plasma das Verhältnis der chemischen Bestandteile in der abgelagerten Schicht gezielt gesteuert wird. Zum Beispiel könnte eine Erhöhung des Sauerstoffdrucks im Plasma die Menge an Sauerstoff in der Schicht erhöhen, was zur Bildung einer Schicht mit einem höheren Anteil an Sauerstoff führt \cite{anders2010reactive, ohring2002, mattox2010handbook, rossnagel2003magnetron}.} der abgeschiedenen Schicht durch} den \textbf{Gasdruck des Reaktionsgases gezielt beeinflusst} werden kann \cite{keplinger2024}.



% Pysical Vapour Deposition -> Vergleich Aufdampfen - Sputtern ------------------------------------------------------------------------------------------------------------


\vspace{1em}

\subsection{Vergleich Aufdampfen - Sputtern}

Abschließend werden in \autoref{tab:Vergleich Aufdampfen - Sputtern} das thermische Aufdampfen und das Sputtern gegenübergestellt.

\begin{table}[htb!]
    \centering
    \renewcommand{\arraystretch}{1.5} % Erhöht den Zeilenabstand in der Tabelle
    \begin{tabular}{|p{6cm}|p{6cm}|}
        \hline
        \multicolumn{2}{|c|}{\textbf{1. Dampferzeugung}} \\ \hline
        \textbf{Aufdampfen} & \textbf{Sputtern} \\ \hline
        Thermischer Prozess & Ionenbeschuss und Impulsübertragung \\
        Geringe kinetische Energie (1500°C: $E \approx 0,1 \, \mathrm{eV}$) & Hohe kinetische Energie ($E = 10 - 40 \, \mathrm{eV}$) \\
        Neutrale Teilchen & Neutrale Teilchen, Kationen, Radikale, (Anionen) \\
        Fraktionierte Verdampfung bei Legierungen & Relativ gleichmäßiges Verdampfen aller Komponenten \\
        Verbindungen dissoziieren & Geringe Dissoziation \\ \hline
        \multicolumn{2}{|c|}{\textbf{2. Transport}} \\ \hline
        \textbf{Aufdampfen} & \textbf{Sputtern} \\ \hline
        Teilchen im Hochvakuum & Teilchen im Plasma \\
        Keine Kollisionen & Viele Kollisionen \\
        Große freie Weglänge & Kleine freie Weglänge \\
        Teilchen bewegen sich geradlinig von der Quelle zum Substrat & Breite Richtungsverteilung der beim Substrat ankommenden Teilchen \\ \hline
        \multicolumn{2}{|c|}{\textbf{3. Kondensation}} \\ \hline
        \textbf{Aufdampfen} & \textbf{Sputtern} \\ \hline
        Wenig Restgasatome in den Schichten & Starker Einbau von Atomen des Plasmas \\
        Kaum Beeinflussung des Substrats & Starke Beeinflussung durch Ionen \\
        Seitenflächen werden nicht beschichtet, ebene Schichten & Alle Flächen werden beschichtet \\
        Mittelmäßige Haftung der Schichten & Gute Haftung \\
        Hohe Beschichtungsraten & Geringe Beschichtungsraten \\
        Lange Prozesszeit durch UHV & Geringe Anforderung an die Vakuumqualität \\
        Strukturieren durch Ätzen oder Lift-off & Strukturieren hauptsächlich durch Ätzen, fallweise durch Lift-off \\ \hline
    \end{tabular}
    \captionsetup{labelfont=bf} % Setzt die Bildnummer fett
    \caption{Vergleich Audampfen - Sputtern \cite{keplinger2024}}
    \label{tab:Vergleich Aufdampfen - Sputtern}
\end{table}

\vspace{1em} % KEEP?

``Verbindungen dissoziieren'' bedeutet, dass chemische Bindungen im Material beim Aufdampfen thermisch zerfallen\footnote{Thermischer Zerfall beschreibt den Prozess, bei dem chemische Verbindungen bei hohen Temperaturen in ihre Bestandteile zerfallen, da die Bindungsenergie durch zugeführte Wärme überwunden wird. Dieser Zerfall kann beim Aufdampfen auftreten und die Zusammensetzung der Schicht beeinflussen \cite{ohring2002, smith1995}.} können, was die Zusammensetzung verändert. Im Gegensatz dazu tritt beim Sputtern nur eine geringe Dissoziation auf, da der Ionenbeschuss die chemische Stabilität der Verbindungen weitgehend erhält \cite{ohring2002, smith1995, rossnagel2003magnetron}.

\vspace{1em} % KEEP?

``Strukturieren durch Ätzen oder Lift-off'' sind Verfahren zur Formgebung von Dünnschichtstrukturen. Beim Ätzen wird überschüssiges Material selektiv durch chemische oder physikalische Methoden entfernt, wodurch nur die gewünschte Struktur verbleibt. Das Lift-off-Verfahren hingegen nutzt eine Maske, die nach der Beschichtung entfernt wird, sodass das Material nur an den zuvor maskierten Stellen verbleibt \cite{ohring2002, smith1995, prechtl2005grundlagen}. Mehr dazu folgt in den späteren Kapiteln.



% Chemical Vapour Deposition ----------------------------------------------------------------------------------------------------------------------------------------------


\clearpage
\section{Chemical Vapour Deposition} % CVD

PVD-Verfahren sind essenzielle Prozesse in der Herstellung miniaturisierter Sensoren, vor allem für metallische Schichten und Dielektrika. Diese Verfahren erfordern feste Ausgangsstoffe, sind jedoch problematisch bei Materialien, die im Prozess disproportionieren\footnote{Disproportionierung ist ein Prozess, bei dem ein Stoff sich aufteilt: Ein Teil gibt Elektronen ab (Oxidation) und ein anderer Teil nimmt Elektronen auf (Reduktion). Dadurch entstehen zwei neue Stoffe mit dem gleichen Element, aber in verschiedenen Zuständen \cite{petrucci_general_chemistry}. In PVD-Prozessen kann dies problematisch sein, weil beim Erhitzen zum Beispiel Metallteile aufbrechen und nicht als ganzes Metall in die Schicht eingebaut werden. Stattdessen bilden sich zwei verschiedene Verbindungen, die nicht alle Eigenschaften des Ausgangsstoffes besitzen, was die Qualität der Schicht beeinflussen kann.}, etwa durch Schmelzen beim Aufdampfen. Dadurch können wichtige Komponenten unvollständig in die Schicht eingebaut werden. Im Gegensatz dazu werden \textbf{bei} der \textbf{Chemical Vapor Deposition (CVD) Gase in den Rezipienten einer Anlage geleitet}, wo sie \textbf{chemisch reagieren und} die \textbf{Reaktionsprodukte am Substrat}, oft ein Wafer, eine \textbf{Schicht bilden}. Diese chemische Reaktion \textbf{erfordert Energie}, die \textbf{thermisch oder durch Gasentladung} zugeführt wird \cite{keplinger2024CVD}.

\vspace{1em}

Die \textbf{Unterscheidung} zwischen \textbf{CVD und PVD} ergibt sich durch die \textbf{Form der zugeführten Ausgangsstoffe (Precursor)}: Bei \textbf{PVD-Prozessen} werden diese \textbf{in fester Form} und bei \textbf{CVD-Prozessen} als \textbf{Gase} zugeführt. Die thermische Oxidation eines Silizium-Wafers wird nicht als CVD-Prozess betrachtet, da nur der Sauerstoff gasförmig ist, während das Silizium bereits fest im Reaktor vorhanden ist \cite{keplinger2024CVD}.

\vspace{0.0em}

\paragraph{Vorteile:}

\begin{itemize}
    \item \textbf{Hohe Reinheit} der Schichten, da die Ausgangsgase in hoher Reinheit verfügbar sind.
    \item \textbf{Gute Kantenbedeckung}, auch bei ungünstiger Geometrie.
    \item \textbf{Herstellung} vieler \textbf{verschiedener Schichtmaterialien} möglich.
    \item \textbf{Zusammensetzung kann} während des Prozesses \textbf{angepasst werden}.
    \item \textbf{Gute Prozesskontrolle} durch eine \textbf{Vielzahl leicht beeinflussbarer Parameter}: Gase, Flussraten, Temperatur, Druck, Frequenz und Leistung des Plasmas.
\end{itemize}

\vspace{0.0em}

\paragraph{Nachteile:}

\begin{itemize}
    \item Relativ \textbf{hohe Temperaturen} sind oft \textbf{erforderlich}, wodurch \textbf{mechanische Spannungen} in die Schicht eingebaut werden. Dies führt zu einer \textbf{hohen thermischen Belastung des Wafers} und der darauf befindlichen Strukturen, was Dotierprofile ändern oder Metallschichten zerstören kann.
    \item \textbf{Schwierige Handhabung} der häufig \textbf{giftigen und brennbaren Gase}.
    \item \textbf{Für viele gewünschte Schichten} sind \textbf{keine gasförmigen Precursoren verfügbar}.
    \item Die \textbf{Vielzahl relevanter Parameter erfordert hohe Sachkenntnis} des Betreibers der Anlage, insbesondere bei Nichtstandard-Prozessen.
\end{itemize}



% Chemical Vapour Deposition ->  Begriffsklärung für CVD ------------------------------------------------------------------------------------------------------------------

\vspace{1em}

\subsection{Begriffsklärung für CVD} % CVD  Begriffsklärung

Dieser Abschnitt bietet eine Einführung in wichtige, häufig verwendete Begriffe. Es empfiehlt sich, zuerst die nahcfolgenden Abschnitte zu lesen und hier bei Bedarf unbekannte Begriffe nachzuschauen. \textbf{Begriffe}, die bereits \textbf{in den vorherigen Abschnitten erklärt} wurden, sind \textbf{hier nicht erneut aufgeführt}.

\begin{itemize}
    \item \textbf{Precursor}: Ausgangsstoffe bzw. Vorläufersubstanzen (Gase) \cite{keplinger2024CVD}. 
    \item \textbf{Reaktanden}: Stoffe, die bei einer chemischen Reaktion miteinander reagieren, um neue Produkte zu bilden. Man könnte sie als die ``Zutaten'' einer chemischen Reaktion betrachten \cite{petrucci_general_chemistry, chang_chemistry}.
    \item \textbf{Diffusionsprozess}: Ein Diffusionsprozess beschreibt die Bewegung von Teilchen von einem Bereich hoher Konzentration zu einem Bereich niedriger Konzentration, bis die Teilchen gleichmäßig verteilt sind \footnote{Man kann es sich vorstellen wie Tinte, die sich langsam in einem Glas Wasser verteilt, ohne dass man umrühren muss.} \cite{bergman2011, incropera2007}. In der Wissenschaft beschreibt ``diffundieren'' den Prozess, bei dem Teilchen sich aufgrund eines Konzentrationsgefälles bewegen \cite{crank1975}.
    \item \textbf{Erzwungene Konvektion}: Erzwungene Konvektion bedeutet, dass der Transport von Stoffen, wie Gasen, durch eine äußere Kraft unterstützt wird, zum Beispiel durch einen Ventilator oder einen Gasstrom. Dadurch gelangen die Reaktanten gezielt an ihren Bestimmungsort \cite{incropera2007, bergman2011, kays2005, bejan2013}.
    \item \textbf{Konformität}: Verhältnis der Beschichtungsdicke auf den vertikalen Flächen zu der auf den horizontalen Flächen der Strukturen am Substrat \cite{keplinger2024CVD}.
\end{itemize}



% Chemical Vapour Deposition -> Funktionsweise ----------------------------------------------------------------------------------------------------------------------------


\vspace{1em}

\subsection{Funktionsweise} % CVD Funktionsweise

Die Schichtbildung basiert auf einer \textbf{Reihe von Teilprozessen}, die je nach Reaktionsgasen komplex ablaufen können (\autoref{fig:Schichtbildung CVD}). Hier werden \textbf{beispielhaft} die Teilprozesse zur \textbf{Bildung von Siliziumnitrid beschrieben}, das häufig als \textbf{Isolationsschicht} dient \cite{keplinger2024CVD}.

\begin{figure}[htb!]
    \centering
    \includegraphics[width=.6\textwidth]{images/Schichtbildung CVD.png} % Pfad und Dateiname des Bildes angeben
    \captionsetup{labelfont=bf, width=.7\textwidth} % Setzt die Bildnummer fett
    \caption{Prozesse der Schichtbildung bei der CVD-Methode: 1...Gaszufuhr von SiH$_4$ und NH$_3$, 2...Diffusion der Reaktanten durch die Grenzschicht, 3...Adsorption der Reaktanten an der Waferoberfläche, 4...chemische Reaktion, 5...Oberflächendiffusion der Reaktionsprodukte, 6...Desorption der Nebenprodukte und 7...Abtransport durch den Gasstrom \cite{keplinger2024CVD}.}
    \label{fig:Schichtbildung CVD}
\end{figure}

\begin{enumerate}
    \item \textbf{Gaszufuhr}: Die Reaktanden SiH$_4$ (Silan) und NH$_3$ (Ammoniak) werden in die Reaktionskammer geleitet und strömen über das Substrat. Dabei gelangen die Ausgangsstoffe durch erzwungene Konvektion zum Substrat \cite{keplinger2024CVD}.
    \item \textbf{Diffusion}: Die Geschwindigkeit des Gases nimmt zur Oberfläche hin innerhalb der Prandtl'schen Grenzschicht\footnote{Die Prandtl'sche Grenzschicht ist der Bereich in der Nähe einer Oberfläche, in dem die Strömungsgeschwindigkeit eines Fluids von der maximalen Geschwindigkeit bis auf Null abnimmt. Innerhalb dieser Schicht verlangsamt sich das Fluid aufgrund der Reibungskräfte an der Grenzfläche \cite{white2006}.} bis auf Null ab. Deshalb müssen die Reaktanden diese Schicht durch einen Diffusionsprozess überwinden\footnote{Die Reaktanden diffundieren durch die Prandtlsche Grenzschicht zur Oberfläche \cite{bergman2011, incropera2007}.} \cite{keplinger2024CVD}.
    \item \textbf{Adsorption}: Die Moleküle, die zur Oberfläche hin diffundieren, ``kleben'' an der Oberfläche fest (adsorbieren). Dort bleiben sie aber nicht an einer festen Stelle, sondern bewegen sich je nach Temperatur mehr oder weniger stark über die Oberfläche \cite{keplinger2024CVD, bergman2011, incropera2007}.
    \item \textbf{Reaktion}: Die Ausgangsstoffe reagieren an der Oberfläche. Bei der Bildung von Siliziumnitrid folgt die Reaktion der chemischen Gleichung:
    $$
    3\mathrm{SiH}_4 + 4\mathrm{NH}_3 \rightarrow \mathrm{Si}_3\mathrm{N}_4 + 12\mathrm{H}_2
    $$
    Ein Teil der Reaktion kann jedoch auch schon in der Gasphase ablaufen. Das ist meist unerwünscht, weil dabei feste Partikel entstehen können, die sich dann als Staub ablagern. Um sicherzustellen, dass die Reaktion vor allem an der Oberfläche stattfindet, hält man den Prozessdruck typischerweise zwischen 0,01 und 10~mbar oder verdünnt die Gase mit einem Trägergas\footnote{Durch das Reduzieren des Drucks oder die Zugabe eines Trägergases wird die Dichte der Ausgangsstoffe verringert, sodass die Moleküle seltener aufeinanderstoßen. Dadurch haben die Reaktanten eine höhere Chance, die Oberfläche zu erreichen und dort zu reagieren, statt bereits in der Gasphase zu reagieren.} \cite{keplinger2024CVD}.
    \item \textbf{Oberflächendiffusion}: Durch diesen Vorgang wird das Reaktionsprodukt Si$_3$N$_4$ (Siliziumnitrid) in die Schicht eingebaut\footnote{Damit ist gemeint, dass das Reaktionsprodukt in die Oberfläche diffundiert. Es wandert also in die Schicht hinein und verteilt sich.} und möglichst bis an die Schichtkante transportiert \cite{keplinger2024CVD}.
    \item \textbf{Desorption}: Bei der chemischen Reaktion entstehen nicht nur die gewünschten Schichtbestandteile, sondern auch Nebenprodukte (wie hier H$_2$), die an der Oberfläche ``kleben'' (adsorbiert sind) bleiben. Diese müssen sich wieder lösen (desorbieren) und dann durch die Grenzschicht ``hindurchwandern'' (diffundieren), um entfernt zu werden \cite{keplinger2024CVD}.
    \item \textbf{Abtransport}: Im letzten Schritt werden die Nebenprodukte vom konvektiven Gasfluss mitgenommen und abtransportiert \cite{keplinger2024CVD}.
\end{enumerate}

Je nach \textbf{Beschichtungsparametern}, insbesondere der \textbf{Temperatur}, können auch \textbf{Nebenprodukte in die Schicht eingebaut} werden, was die \textbf{Schichtqualität erheblich beeinflusst}. \textbf{Siliziumnitride}, die \textbf{bei niedrigen Temperaturen abgeschieden} werden, e\textbf{nthalten mehr Wasserstoff} und sind daher \textbf{weniger ätzresistent} als Hochtemperaturnitride. Auch die \textbf{Oberflächendiffusion hängt stark von} der \textbf{Temperatur ab}: \textbf{Niedrigere Temperaturen} führen zu geringerer Beweglichkeit auf der Oberfläche und somit zu \textbf{poröseren Schichten} \cite{keplinger2024CVD}.



% Chemical Vapour Deposition -> CVD-Anlagen -------------------------------------------------------------------------------------------------------------------------------


\vspace{1em}

\subsection{CVD-Anlagen} % CVD-Anlagen

Die verschiedenen CVD-Anlagen können nach unterschiedlichen Kriterien wie Betriebs- oder Konstruktionsparametern klassifiziert werden. Es gibt zahlreiche mögliche Kombinationen, wobei jedoch nicht alle theoretischen Optionen vollständig genutzt werden. Einige dieser Kriterien werden im Folgenden genauer besprochen \cite{keplinger2024CVD}:

\begin{itemize}
    \item \textbf{Verbrauch an Reaktanden}
    \item \textbf{Reaktionsort}
    \item \textbf{Energiezufuhr}
    \item \textbf{Erwärmte Anlagenteile}
    \item \textbf{Druck}
\end{itemize}



% Chemical Vapour Deposition -> CVD-Anlagen -> Verbruach an Reaktanden ----------------------------------------------------------------------------------------------------


\vspace{1em}

\subsubsection{Verbrauch an Reaktanden} % Verbruach an Reaktanden

Ein wichtiges Kriterium ist der Verbrauch der Reaktanden, der darüber entscheidet, \textbf{welcher CVD-Reaktor für welchen Anwendungsfall} geeignet ist. Man unterschiedet zwischen Differential- und Integralreaktoren \cite{keplinger2024CVD}.

\vspace{0.0em}

\paragraph{\uline{Differnetialreaktor:}} \textbf{Fast die gesamte Menge der Ausgangsstoffe, die in die Reaktionskammer eintritt, verlässt diese auch wieder}. Lediglich eine \textbf{vergleichsweise kleine (differentielle) Menge wird tatsächlich verbraucht} und abgeschieden. Aus Sicht des Materialeinsatzes ist \textbf{dieser Reaktortyp} zwar \textbf{unökonomisch}, jedoch bleiben die \textbf{Konzentrationen im Reaktor konstant}, was \textbf{gleichmäßige Beschichtungen} ermöglicht \cite{keplinger2024CVD}.

\vspace{0.0em}

\paragraph{\uline{Integralreaktor:}} \textbf{Entlang des Gasstroms} entsteht ein \textbf{deutlicher Konzentrationsgradient}, wobei \textbf{stromabwärts} ein \textbf{Mangel an Reaktanden} auftritt (\textit{starved reactor}). Diese Reaktoren sind aufgrund der \textbf{hohen Abscheideraten gut für} die \textbf{Produktion} geeignet. \textbf{Bei hohen Temperaturen und entsprechend hoher Oberflächenbeweglichkeit} können \textbf{auch hier homogene Schichten} erzielt werden.



% Chemical Vapour Deposition -> CVD-Anlagen -> Reaktionsort ---------------------------------------------------------------------------------------------------------------


\vspace{1em}

\subsubsection{Reaktionsort} % Reaktionsort

Bei diesem Kriterium wird unterschieden, \textbf{an welchem Ort in der CVD-Anlage die Reaktanden miteinander reagieren} \cite{keplinger2024CVD}.

\vspace{0.0em}

\paragraph{\uline{Homogene CVD:}} Die \textbf{Reaktanden reagieren bereits in der Gasphase}, wobei das \textbf{Reaktionsprodukt durch} die 
\textbf{Grenzschicht zum Substrat diffundiert} und sich \textbf{dort abschiedet}. Solche CVD-Prozesse führen häufig zu \textbf{schlecht haftenden Schichten} mit 
deutlich \textbf{mehr Defekten} \cite{keplinger2024CVD}.

\vspace{0.0em}

\paragraph{\uline{Heterogene CVD:}} Die Schicht bildet sich durch die \textbf{chemische Reaktion direkt an der Substratoberfläche}. Aufgrund der \textbf{besseren Schichteigenschaften} wird diese Variante \textbf{in der Mikrosystemtechnik bevorzugt} \cite{keplinger2024CVD}.



% Chemical Vapour Deposition -> CVD-Anlagen -> Energiezufuhr --------------------------------------------------------------------------------------------------------------


\vspace{1em}

\subsubsection{Energiezufuhr} % Energiezufuhr

Ein weiteres wesentliches Kriterium ist die Energiezufuhr, die bestimmt, \textbf{wie die für die Reaktion benötigte Energie bereitgestellt wird} \cite{keplinger2024CVD}.

\vspace{0.0em}

\paragraph{\uline{Thermische Anregung:}} Das \textbf{Substrat} wird \textbf{mithilfe von Öfen oder Heizplatten auf} die erforderliche \textbf{Reaktionstemperatur gebracht} \cite{keplinger2024CVD}.

\vspace{0.0em}

\paragraph{\uline{Plasma:}} Durch eine \textbf{Gasentladung} entsteht ein Plasma, das den \textbf{Großteil der Energie für die chemische Reaktion} bereitstellt, weshalb man von \textbf{PECVD (\textit{plasma enhanced} CVD)} spricht. Die \textbf{thermische Belastung der Oberfläche und des Sunstrats} bleibt dabei \textbf{gering}\footnote{In der Physik bezeichnet man ein Plasma als ein teilweise oder vollständig ionisiertes Gas. Plasmen werden in ``heiß'' und ``kalt'' unterteilt, je nachdem, ob thermisches Gleichgewicht zwischen den Teilchen besteht. In einem heißen Plasma haben Elektronen und Ionen die gleiche Temperatur (typischerweise $kT > 5000$ K), die notwendig ist, um Elektronen aus Atomen herauszuschlagen. In einem kalten Plasma hingegen ist nur die Temperatur der Elektronen hoch, während die Ionen und neutralen Gasteilchen deutlich kühler bleiben. Das liegt daran, dass Elektronen etwa 100.000-mal leichter als Ionen sind und somit kaum Energie mit diesen austauschen können, ähnlich wie ein Tischtennisball, der eine Billardkugel trifft. Aufgrund des geringen Drucks in Plasmaanlagen ist die freie Weglänge groß, was zu wenigen Kollisionen führt, bevor die Teilchen die Anlage verlassen. Bei der Anregung des Plasmas durch eine Gasentladung werden die Elektronen erhitzt, ohne das Gas insgesamt zu erwärmen. Die hochenergetischen Elektronen ionisieren oder dissoziieren das Restgas und ermöglichen chemische Reaktionen, die normalerweise bei niedrigen Temperaturen nicht stattfinden würden, wodurch die thermische Belastung des Substrats und der Oberflächen gering bleibt \cite{keplinger2024CVD}.} \cite{keplinger2024CVD}.

\vspace{0.0em}

Das \textbf{Plasma kann kapazitiv} durch parallel angeordnete Elektroden (Parallelplattenanlage) -- zu sehen in \autoref{fig:PECVD} -- \textbf{oder induktiv} durch eine Spule um die Reaktorkammer (ICP-Anlage, \textit{inductive coupled plasma}) \textbf{angeregt werden}. Dabei wird häufig die industriefreigegebene Frequenz von 13,56~MHz verwendet, und die Anlagen sind aufgrund ihrer \textbf{starken elektromagnetischen Strahlung} abgeschirmt \cite{keplinger2024CVD}.

\begin{figure}[htb!]
    \centering
    \includegraphics[width=\textwidth]{images/PECVD-Anlage.png} % Pfad und Dateiname des Bildes angeben
    \captionsetup{labelfont=bf, width=.9\textwidth} % Setzt die Bildnummer fett
    \caption{Schematischer Aufbau einer PECVD-Anlage, bei der das Plasma durch eine Gasentladung erzeugt wird (Parallelplattenanlage). Die Elektronentemperatur beträgt $T_\mathrm{el} = 10\,000{-}90\,000 \, \mathrm{K}$ (1–8 eV), die Ionentemperatur $T_\mathrm{ion} = 500{-}1\,000 \, \mathrm{K}$, und der Druck liegt bei $p = 0.1{-}10 \, \mathrm{mbar}$ \cite{keplinger2024CVD}.}
    \label{fig:PECVD}
\end{figure}

\vspace{0.0em}

\paragraph{\uline{Laser:}} Durch die \textbf{gezielte lokale Erwärmung des Substrats mittels Laser (LCVD)} können \textbf{Schichten schreibend auf} das \textbf{Substrat aufgetragen} und sogar \textbf{dreidimensionale Strukturen} erzeugt werden (\autoref{fig:LCVD}). Wie bei den meisten schreibenden Verfahren eignet sich der LCVD-Prozess \textbf{gut für Prototypen und Forschungszwecke}, ist jedoch \textbf{weniger geeignet für industrielle Anwendungen} \cite{keplinger2024CVD}.

\begin{figure}[htb!]
    \centering
    \includegraphics[width=.55\textwidth]{images/Laser CVD.png} % Pfad und Dateiname des Bildes angeben
    \captionsetup{labelfont=bf, width=.65\textwidth} % Setzt die Bildnummer fett
    \caption{Laser-unterstützte CVD: Der Laser erwärmt das Substrat punktuell, wodurch die chemische Reaktion und Schichtabscheidung nur im Bereich des Laserflecks stattfinden \cite{keplinger2024CVD}.}
    \label{fig:LCVD}
\end{figure}



% Chemical Vapour Deposition -> CVD-Anlagen -> Erwärmte Anlagenteile ------------------------------------------------------------------------------------------------------


\vspace{1em}

\subsubsection{Erwärmte Anlagenteile} % Erwärmte Anlagenteile

Bei einer CVD-Anlage ist ebenfalls zu berücksichtigen, welche \textbf{Teile des Reaktors} erwärmt werden \cite{keplinger2024CVD}.

\vspace{0.0em}

\paragraph{\uline{Hot-Wall-Reaktor:}} Der \textbf{Substrathalter (Susceptor) mit} den \textbf{Substraten und die Reaktorkammer werden durch} einen \textbf{Ofen mit Widerstandsheizung erwärmt} (\autoref{fig:Hot-Wall-Reaktor}). Die \textbf{Wafer stehen frei} und können \textbf{dicht gepackt} werden, wodurch die \textbf{Beschichtung} (z.B. Polysilizium) \textbf{auf beiden Seiten bei Temperaturen zwischen 400 und 1000°C} erfolgt. Allerdings werden \textbf{auch die Reaktorwände beschichtet}, was durch \textbf{Partikelbildung} zu Problemen führen kann \cite{keplinger2024CVD}.

\begin{figure}[htb!]
    \centering
    \includegraphics[width=.6\textwidth]{images/Hot-wall-reactor.png} % Pfad und Dateiname des Bildes angeben
    \captionsetup{labelfont=bf} % Setzt die Bildnummer fett
    \caption{Schematische Anordnung beim Hot-wall-Reaktor \cite{keplinger2024CVD}.}
    \label{fig:Hot-Wall-Reaktor}
\end{figure}

\vspace{0.0em}

\paragraph{\uline{Cold-Wall-Reaktor:}} Der \textbf{Waferhalter} wird hier entweder \textbf{resistiv, induktiv oder durch Strahlungslampen erhitzt}, während die \textbf{Reaktorwände kalt} bleiben (\autoref{fig:Cold-Wall-Reaktor}). Dies führt zu \textbf{geringeren Ablagerungen an den Wänden} und dadurch zu \textbf{weniger Partikelkontaminationen} im Vergleich zum Hot-Wall-Reaktor \cite{keplinger2024CVD}.

\begin{figure}[htb!]
    \centering
    \includegraphics[width=.575\textwidth]{images/Cold-wall-reactor.png} % Pfad und Dateiname des Bildes angeben
    \captionsetup{labelfont=bf} % Setzt die Bildnummer fett
    \caption{Schematischer Aufbau des Cold-wall-Reaktors \cite{keplinger2024CVD}.}
    \label{fig:Cold-Wall-Reaktor}
\end{figure}



% Chemical Vapour Deposition -> CVD-Anlagen -> Druck ----------------------------------------------------------------------------------------------------------------------


\vspace{1em}

\subsubsection{Druck} % Druck

Schließlich ist auch der Druck zu berücksichtigen, \textbf{unter dem die Reaktionen stattfinden} \cite{keplinger2024CVD}.

\vspace{0.0em}

\paragraph{\uline{Atmospheric Pressure CVD (APCVD):}} Bei diesem Verfahren erfolgt die \textbf{Abscheidung einer Oxidschicht unter Atmosphärendruck} und ohne Evakuierung der Anlage bei einer Temperatur von z.B. \textbf{400°C}. Die Oxidschicht bildet sich über \textbf{folgende chemische Reaktionen}\footnote{Bei diesen Reaktionen handelt es sich um Oxidationsreaktionen, bei denen aus dem Ausgangsstoff Silan (\(\mathrm{SiH}_4\)) und Sauerstoff (\(\mathrm{O}_2\)) Siliziumdioxid (\(\mathrm{SiO}_2\)) gebildet wird \cite{keplinger2024CVD}. Dies geschieht über zwei mögliche Reaktionswege: 1. \(\mathrm{SiH}_4 + 2 \, \mathrm{O}_2 \rightarrow \mathrm{SiO}_2 + 2 \, \mathrm{H}_2\mathrm{O}\): Silan reagiert mit zwei Molekülen Sauerstoff und bildet Siliziumdioxid sowie Wasser (\(\mathrm{H}_2\mathrm{O}\)) als Nebenprodukt. 2. \(\mathrm{SiH}_4 + \mathrm{O}_2 \rightarrow \mathrm{SiO}_2 + 2 \, \mathrm{H}_2\): Hier reagiert Silan mit einem Molekül Sauerstoff, wodurch ebenfalls Siliziumdioxid entsteht, jedoch ist das Nebenprodukt Wasserstoffgas (\(\mathrm{H}_2\)). Beide Reaktionen finden bei typischen CVD-Prozesstemperaturen statt und führen zur Bildung einer Siliziumdioxidschicht (\(\mathrm{SiO}_2\)) auf dem Substrat. Die Wahl der Reaktion hängt von den Prozessbedingungen wie Temperatur und Sauerstoffkonzentration ab \cite{keplinger2024CVD}.}:
$$
\mathrm{SiH}_4 + 2 \, \mathrm{O}_2 \rightarrow \mathrm{SiO}_2 + 2 \, \mathrm{H}_2\mathrm{O}
$$
$$
\mathrm{SiH}_4 + \mathrm{O}_2 \rightarrow \mathrm{SiO}_2 + 2 \, \mathrm{H}_2
$$
Aufgrund der \textbf{niedrigen Prozesstemperatur} werden die \textbf{Seitenwände (Flanken) der Strukturen weniger stark beschichtet als die horizontalen Bereiche}, was zu einer \textbf{geringen Konformität}\footnote{Die Konformität beschreibt das Verhältnis der Beschichtungsdicke an den Seitenwänden zur Beschichtungsdicke in horizontalen Bereichen \cite{keplinger2024CVD}.} führt \cite{keplinger2024CVD}.
    
\vspace{0.0em}

\paragraph{\uline{Low Pressure CVD (LPCVD):}} Werden die \textbf{Schichten in} einem \textbf{evakuierten Reaktor abgeschieden}, handelt es sich um Low Pressure CVD. Die \textbf{Drücke} liegen typischerweise im Bereich von \textbf{10 bis 100 Pa}. Ein häufiges \textbf{Anwendungsbeispiel} ist die \textbf{Abscheidung dünner Schichten aus Polysilizium oder Siliziumnitrid}. Diese Schichten weisen eine \textbf{nahezu perfekte Konformität} auf, d.h., die \textbf{vertikalen Schichten in Kanälen haben fast die gleiche Dicke wie die horizontalen Flächen}, mit einer \textbf{Konformität von bis zu 0,98} \cite{keplinger2024CVD}.



% Chemical Vapour Deposition -> Atomlagenabscheidung ----------------------------------------------------------------------------------------------------------------------


\vspace{1em}

\subsection{Atomlagenabscheidung} % Atomlagenabscheidung

Bei der Atomlagenabscheidung (\textit{atomic layer deposition}, ALD) werden \textbf{in einem zyklischen Verfahren verschiedene Gase abwechselnd in den Reaktor geleitet}. Die einzelnen \textbf{Vorläufersubstanzen bilden monomolekulare Schichten}\footnote{Siehe Monolage (Begriffsklärung für PVD).}, die das \textbf{Schichtwachstum auf eine Moleküllage begrenzen. Zwischen} den \textbf{Prozessschritten mit} den \textbf{Vorläufersubstanzen} wird der \textbf{Reaktor mit} einem \textbf{inerten Gas gespült, um unerwünschte Reaktionen} im Reaktorvolumen \textbf{zu verhindern} \cite{keplinger2024CVD}.

\vspace{1em}

Ein Zyklus läuft wie folgt ab \cite{keplinger2024CVD}:

\begin{enumerate}
    \item \textbf{Reaktand 1}: Bildung einer monomolekularen Schicht.
    \item \textbf{Spülgas} (z.B. N$_2$, Ar) oder Abpumpen, um den ersten Reaktanden zu entfernen.
    \item \textbf{Reaktand 2}: Durch die selbstbegrenzende Reaktion mit der ersten Schicht wird eine weitere monomolekulare Schicht gebildet. Dieser Prozessschritt kann auch eine Plasmabehandlung beinhalten, die die Oberfläche wieder für den ersten Schritt aktiviert.
    \item \textbf{Spülgas}
\end{enumerate}

\vspace{1em}

Durch \textbf{Zählen der Zyklen} kann eine \textbf{Genauigkeit der Schichtdicke von einer einzelnen Atomlage} erreicht werden \cite{keplinger2024CVD}.

\paragraph{\uline{Beispiel:}} Als Beispiel ist in \autoref{fig:ALD} die \textbf{Bildung von Aluminiumoxid (\(\mathrm{Al_2O_3}\))} dargestellt. Im ersten Schritt wird \textbf{Trimethylaluminium (TMA) (\(\mathrm{Al(CH_3)_3}\)) in} den \textbf{Reaktor geleitet}, das sich \textbf{durch Abspaltung von Methan ($\mathrm{CH_4}$) an} die \textbf{Sauerstoffatome des Substrats anlagert}. Dieser \textbf{Prozess endet, sobald keine OH-Gruppen mehr ``sichtbar''} sind und die \textbf{Oberfläche nur noch Methylgruppen\footnote{Methylgruppen sind chemische Gruppen mit der Formel \(\mathrm{CH_3}\), die aus einem Kohlenstoffatom und drei Wasserstoffatomen bestehen. Sie sind oft Bestandteil organischer Moleküle und beeinflussen die physikalischen und chemischen Eigenschaften von Verbindungen \cite{petrucci_general_chemistry}.} aufweist}. \textbf{Nach} dem \textbf{Spülen wird Wasserdampf eingeleitet}, der \textbf{mit} den \textbf{Methylgruppen reagiert}. Dabei wird \textbf{Methan abgespalten und abgepumpt}. Der \textbf{Prozess endet, sobald keine Methylgruppen mehr vorhanden} sind und die \textbf{Oberfläche von Hydroxylgruppen\footnote{Hydroxygruppen sind funktionelle Gruppen mit der Formel \(\mathrm{OH}\), die aus einem Sauerstoff- und einem Wasserstoffatom bestehen. Sie sind charakteristisch für Alkohole und beeinflussen die Polarität und Reaktivität von Molekülen, an denen sie gebunden sind \cite{petrucci_general_chemistry}.} bedeckt} ist. Nach einem weiteren \textbf{Spülprozess} beginnt der \textbf{Zyklus erneut} mit dem Trimethylaluminium-Schritt \cite{keplinger2024CVD}.

\begin{figure}[htb!]
    \centering
    \includegraphics[width=.8\textwidth]{images/Atomlagenabscheidung.png} % Pfad und Dateiname des Bildes angeben
    \captionsetup{labelfont=bf} % Setzt die Bildnummer fett
    \caption{Abscheidung von \(\mathrm{Al_2O_3}\) (Aluminiumoxid) mittels ALD \cite{keplinger2024CVD}.}
    \label{fig:ALD}
\end{figure}

\vspace{1em}

\textbf{Zusammengefasst (chemisch auf den Punkt gebracht)} verläuft der Prozess wie folgt \cite{keplinger2024CVD}:

Im ersten Schritt reagiert Trimethylaluminium (\(\mathrm{Al(CH_3)_3}\)) mit den OH-Gruppen auf der Oberfläche. Dabei spaltet sich Methan ($\mathrm{CH_4}$) ab (\autoref{fig:ALD} oben links):
$$
\mathrm{Al(CH_3)_3} + \mathrm{OH} \rightarrow \mathrm{Al(CH_3)_2O} + \mathrm{CH_4}
$$
Dieser Schritt läuft ab, bis keine OH-Gruppen mehr auf der Oberfläche vorhanden sind und die Oberfläche vollständig aus \(\mathrm{CH_3}\)-Gruppen besteht (\autoref{fig:ALD} oben rechts).

Danach wird Wasserdampf (\(\mathrm{H_2O}\)) in den Reaktor geleitet, der mit den \(\mathrm{CH_3}\)-Gruppen reagiert. Auch in diesem Schritt spaltet sich Methan ($\mathrm{CH_4}$) ab (\autoref{fig:ALD} unten rechts):
$$
\mathrm{H_2O} + \mathrm{CH_3} \rightarrow \mathrm{OH} + \mathrm{CH_4}
$$
Dieser Schritt endet, sobald keine \(\mathrm{CH_3}\)-Gruppen mehr sichtbar sind und die Oberfläche wieder vollständig von \(\mathrm{OH}\)-Gruppen bedeckt ist -- wie zu Beginn des Zyklus.

Dieser Prozess wiederholt sich und ermöglicht den schichtweisen Aufbau von Aluminiumoxid (\(\mathrm{Al_2O_3}\)) durch die Atomlagenabscheidung (ALD), wobei in jedem Zyklus eine monomolekulare Schicht entsteht.



% Lithographie ------------------------------------------------------------------------------------------------------------------------------------------------------------


\clearpage
\section{Lithographie}

Die Lithographie ist ein zentrales Verfahren in der Mikrosystemtechnik zur \textbf{Herstellung feiner Strukturen auf Halbleitermaterialien}. Dabei werden \textbf{spezifizierte Muster auf eine (zuvor bedampfte) Waferoberfläche übertragen}, die \textbf{anschließend geätzt oder anderweitig verarbeitet} werden, um Mikrochips, Sensoren und andere Miniaturkomponenten (MEMS) zu erzeugen.

\vspace{1em}

Das \textbf{grundlegende Prinzip der Lithographie} umfasst mehrere Schritte: Zuerst wird ein \textbf{lichtempfindlicher Lack (dünner Film)}, ein sogenannter ``\textbf{Photoresist (Photolack)}'', \textbf{auf} das \textbf{Substrat aufgetragen. Durch Masken}, die \textbf{bestimmte Bereiche abdecken}, wird der Photoresist \textbf{selektiv belichtet}. Diese belichteten Bereiche verändern ihre chemischen Eigenschaften, sodass sie \textbf{je nach Resisttyp} (positiv oder negativ) \textbf{entfernt oder erhalten} bleiben. \textbf{Nach} dem \textbf{Entwicklungsprozess bleiben} die \textbf{gewünschten Strukturen auf dem Substrat sichtbar} und dienen als \textbf{Vorlage für weitere Prozessschritte} wie zum Beispiel \textbf{Ätzen} \cite{madou2002}.

\vspace{1em}

In der Mikrosystemtechnik sind \textbf{verschiedene Lithographietechniken} etabliert, darunter die Photolithographie (optische Lithographie) und Elektronenstrahllithographie. Die \textbf{Photolithographie} ist besonders \textbf{für großflächige Produktionen} geeignet, da sie \textbf{hohe Auflösung und Prozessgeschwindigkeit} bietet. Bei der \textbf{Elektronenstrahllithographie} hingegen wird mit einem feinen Elektronenstrahl gearbeitet, wodurch eine \textbf{noch höhere Auflösung} erreicht wird, was besonders \textbf{für Nanostrukturen} entscheidend ist \cite{wolf2000}.

\vspace{1em}

Diese Verfahren bieten die Grundlage für das Design und die Herstellung von Mikro- und Nanosystemen, wie sie in modernen elektronischen und mechanischen Anwendungen benötigt werden. \textbf{In dieser Ausarbeitung} wird jedoch \textbf{ausschließlich} die \textbf{optische Lithographie näher erläutert.}



% Lithographie -> Begriffsklärung für Lithographie ------------------------------------------------------------------------------------------------------------------------


\vspace{1em}

\subsection{Begriffsklärung für die Lithographie}

Begriffe, die bereits in den vorigen Kapiteln erläutert wurden, werden hier nicht erneut angeführt.

\begin{itemize}
    \item \textbf{Photoresist}: Eine lichtempfindliche Schicht (Photolack), die zur Strukturierung von Mikro- und Nanometer-Bereichen dient \cite{madou2002}.
    \item \textbf{Maske} (Photomaske): Eine Vorlage mit einem Muster aus lichtdurchlässigen und lichtundurchlässigen Bereichen, die auf den Photoresist projiziert wird. Sie dient dazu, das gewünschte Strukturmuster zu übertragen \cite{wolf2000}.
    \item \textbf{Kohärenzgrad von Licht}: In der Lithographie beschreibt der Kohärenzgrad des Lichts, wie gleichmäßig die Lichtwellen zueinander schwingen. Eine hohe Kohärenz bedeutet, dass die Lichtwellen eine feste Phasenbeziehung haben -- sie schwingen synchron, wie zwei Freunde, die auf einem Trampolin immer gleichzeitig springen. Diese Synchronität ermöglicht scharfe Interferenzmuster und kann die Präzision der abgebildeten Strukturen erhöhen \cite{Mack2006, BornWolf1999}.

    Eine niedrige Kohärenz bedeutet hingegen, dass die Lichtwellen weniger synchron sind und keinen konstanten Abstand zueinander halten, wie mehrere Personen, die auf einem Trampolin jeweils in ihrem eigenen Rhythmus springen. Diese ``Unordnung'' der Wellen reduziert Interferenzeffekte, was bei der Lithographie helfen kann, eine gleichmäßigere Belichtung zu erreichen und störende Muster zu vermeiden \cite{Mack2006}.
    \item \textbf{LIGA-Verfahren}: LIGA steht für \textbf{Lithographie, Galvanoformung und Abformung}. Es ist ein Verfahren der Mikrosystemtechnik zur Herstellung hochpräziser Mikrostrukturen mit hohen Aspektverhältnissen. Dabei werden in drei Schritten ein Muster per Röntgenstrahllithographie erstellt, dieses galvanisch mit Metall gefüllt (``Galvanisch gefüllt'' bedeutet, dass das Muster im Prozess durch eine elektrochemische Abscheidung (Galvanik) mit Metall aufgefüllt wird. Dabei wird Metall in die Vertiefungen der Form abgeschieden, wodurch stabile und präzise metallische Strukturen entstehen.) und anschließend abgeformt (``Abformung'' bedeutet in diesem Zusammenhang, dass das erzeugte Muster als Form oder Negativ verwendet wird, um weitere Strukturen durch das Einfüllen eines Materials in diese Form herzustellen. Das Ziel ist, exakte Kopien der Struktur zu erzeugen, indem das Material in die Form gegossen oder gepresst und dann herausgenommen wird.). Das Verfahren eignet sich für präzise Anwendungen in der Feinmechanik und Mikrosystemtechnik \cite{Mescheder2004}.
    \item \textbf{DRIE-Verfahren}: DRIE steht für \textbf{Deep Reactive Ion Etching}, ein Verfahren zur Herstellung tief strukturierter Mikrokomponenten mit hohen Aspektverhältnissen in der Mikrosystemtechnik. Es nutzt reaktive Ionen\footnote{Reaktive Ionen sind geladene Teilchen, die chemisch sehr aktiv sind und durch Ionisierung von Gasen in einem Plasma entstehen. In Prozessen wie dem DRIE werden diese Ionen verwendet, um gezielt Material abzutragen. Die reaktiven Ionen reagieren mit der Oberfläche des Materials und erzeugen chemische Verbindungen, die sich ablösen lassen, wodurch sehr präzise Strukturen entstehen \cite{henning2002, wolf2000}.}, die durch ein Plasma erzeugt werden, um das Material gezielt abzutragen. Durch DRIE lassen sich sehr präzise und vertikale Strukturen in Materialien wie Silizium erzeugen, was es ideal für die Herstellung komplexer Mikrostrukturen in Anwendungen wie MEMS macht \cite{madou2002, li2014drie}.
\end{itemize}



% Lithographie -> Photoresist ---------------------------------------------------------------------------------------------------------------------------------------------


\vspace{1em}

\subsection{Photoresist}

Wie bereits erwähnt, handelt es sich beim \textbf{Photoresist} um einen \textbf{lichtempfindlichen Lack}, der \textbf{auf} das \textbf{Substrat aufgetragen} wird. Bei einem sogenannten \textbf{``Negativresist''} verbleibt nach Belichtung und Entwicklung eine Struktur, die invers zur Struktur auf der Photomaske ist: \textbf{An} den \textbf{Stellen, an denen} die \textbf{Photomaske transparent} ist, wird der \textbf{Negativresist belichtet und dadurch unlöslich.} Bei einem \textbf{``Positivresist''} hingegen \textbf{erhöht sich} die \textbf{Löslichkeit der belichteten Bereiche}. Dadurch führen die \textbf{undurchsichtigen Bereiche der Photomaske zu den nicht löslichen und somit verbleibenden Schichten} nach Belichtung und Entwicklung \cite{gerald2006}. Zur Veranschaulichung soll die \autoref{fig:photoresist1} dienen.

\begin{figure}[htb!]
    \centering
    \includegraphics[width=.9\textwidth]{images/photoresist.jpg} % Pfad und Dateiname des Bildes angeben
    \captionsetup{labelfont=bf} % Setzt die Bildnummer fett
    \caption{Vergleich Positiv- und Negativresist \cite{sciencedirect_photoresist}.}
    \label{fig:photoresist1}
\end{figure}

\vspace{1em}

Photoresiste müssen die folgenden \textbf{Anforderungen} erfüllen \cite{schmid2024}:

\begin{itemize}
    \item \textbf{Hohe Strukturgenauigkeit}\footnote{Hohe Strukturgenauigkeit bedeutet, dass der Photoresist die Fähigkeit besitzt, sehr präzise und detaillierte Muster auf dem Substrat zu erzeugen. Dies ist wichtig, um die gewünschten Strukturen exakt abzubilden und sicherzustellen, dass die resultierenden Muster genau den Spezifikationen entsprechen. Eine hohe Strukturgenauigkeit ist entscheidend, da kleinste Abweichungen die Funktionalität der Bauteile beeinträchtigen können \cite{madou2002, wolf2000, sciencedirect_photoresist}.}
    \item \textbf{Gute Maßhaltigkeit}\footnote{Das bedeutet, dass der Photoresist die Abmessungen des gewünschten Musters präzise beibehalten muss, selbst nach den Prozessen der Belichtung, Entwicklung und weiteren Bearbeitungsschritten. Eine hohe Maßhaltigkeit stellt sicher, dass die Struktur exakt in der vorgesehenen Größe und Form erhalten bleibt \cite{madou2002, wolf2000}.}
    \item \textbf{Hohe Sensitivität}\footnote{Hohe Sensitivität bedeutet, dass der Photoresist schon bei geringer Belichtungsenergie zuverlässig reagiert, was zu einer effizienteren und schnelleren Strukturierung führt \cite{madou2002, sciencedirect_photoresist}.}
    \item \textbf{Gute Haftung}
    \item \textbf{Gute Resistenz beim Ätzen}\footnote{Gute Resistenz beim Ätzen bezieht sich auf die Fähigkeit des nicht löslichen Teils des Photoresists, dem Ätzprozess standzuhalten, sodass die darunterliegende Schicht geschützt bleibt \cite{madou2002, wolf2000}.} (daher der Begriff „Resist“)
    \item \textbf{Hohe thermische und chemische Belastbarkeit}
    \item \textbf{Gute Entfernbarkeit (Strippen)} (nach dem Ätzen)
\end{itemize}

\vspace{1em}

Eine detaillierte Gegenüberstellung von Positiv- und Negativresisten ist in \autoref{tab:Vergleich Positiv- und Negativresist Teil 1} und \autoref{tab:Vergleich Positiv- und Negativresist Teil 2} dargestellt.

\begin{table}[htb!]
    \centering
    \renewcommand{\arraystretch}{1.5} % Erhöht den Zeilenabstand in der Tabelle
    \begin{tabular}{|p{6cm}|p{6cm}|}
        \hline
        \multicolumn{2}{|c|}{\textbf{Vergleich Positiv- und Negativresist}} \\ \hline
        \textbf{Positivresist} & \textbf{Negativresist} \\ \hline
        Erhöhte Löslichkeit der belichteten Bereiche (\textbf{belichtete Bereiche werden entfernt}) & Verringerte Löslichkeit der belichteten Bereiche (\textbf{belichtete Bereiche bleiben erhalten}) \\
        \textbf{Hohe Strukturgenauigkeit möglich} (wird hauptsächlich für feine Muster verwendet) & \textbf{Weniger strukturgenau}, besonders bei kleinsten Strukturen (eignet sich für grobe, robuste Strukturen) \\
        \textbf{Gute Maßhaltigkeit} bei kleinen Strukturen & \textbf{Geringere Maßhaltigkeit} da sie zum Quellen neigen \\
        Erfordert \textbf{höhere Belichtungsenergie} & Erfordert \textbf{niedrigere Belichtungsenergie} \\ \hline
    \end{tabular}
    \captionsetup{labelfont=bf} % Setzt die Bildnummer fett
    \caption{Vergleich Positiv- und Negativresist Teil 1 \cite{madou2002, wolf2000}}
    \label{tab:Vergleich Positiv- und Negativresist Teil 1}
\end{table}

\begin{table}[htb!]
    \centering
    \renewcommand{\arraystretch}{1.5} % Erhöht den Zeilenabstand in der Tabelle
    \begin{tabular}{|p{6cm}|p{3cm}|p{3cm}|}
        \hline
        \textbf{Vergleichsparameter} & \textbf{Positiv-Resist} & \textbf{Negativ-Resist} \\ \hline
        Minimale Strukturbreite & \textbf{< 0.5 µm} & ca. 2 µm \\ \hline
        Quellen beim Entwickeln & \textbf{nein} & ja \\ \hline
        Aspektverhältnis & \textbf{gut} & mäßig \\ \hline
        Thermische Stabilität & \textbf{gut (bis 200 °C)} & mäßig \\ \hline
        Staubpartikel & \textbf{kaum Einfluss} & bewirken Löcher \\ \hline
        Passivierung bei \textbf{Plasma-Ätzung} & \textbf{sehr gut} & mäßig \\ \hline
        Passivierung bei \textbf{Nasschemie} & mäßig & \textbf{sehr gut} \\ \hline
        Adhäsion (Haftung) zu Si & gut & \textbf{sehr gut} \\ \hline
        Kosten & \textbf{teurer} & günstiger \\ \hline
    \end{tabular}
    \captionsetup{labelfont=bf} % Setzt die Bildnummer fett
    \caption{Vergleich Positiv- und Negativresist Teil 2 \cite{madou2002, wolf2000}}
    \label{tab:Vergleich Positiv- und Negativresist Teil 2}
\end{table}  

\vspace{1em}

Die angeführten \textbf{Vergleichsparameter} in der \autoref{tab:Vergleich Positiv- und Negativresist Teil 2} beschreiben wichtige \textbf{wichtige Eigenschaften von Positiv- und Negativresisten} und werden nun kurz erläutert: 

\vspace{1em}

Die \textbf{Minimale Strukturbreite} gibt an, wie klein die erzeugten Muster sein können, was für die Herstellung feinster Strukturen entscheidend ist \cite{madou2002, sciencedirect_photoresist}. 

\vspace{1em}

\textbf{``Quellen beim Entwickeln''} beschreibt, ob sich der \textbf{Resist beim Entwickeln ausdehnt}, was die \textbf{Präzision der Strukturen beeinflusst} \cite{hoegel2024}. 

\vspace{1em}

Das \textbf{Aspektverhältnis} beschreibt das \textbf{Verhältnis von Höhe zu Breite der Strukturen} und \textbf{beeinflusst deren Detaillierung}. Ein \textbf{hohes (gutes) Aspektverhältnis} bedeutet, dass \textbf{höhere Strukturen entwickelt bzw. feine Details präziser dargestellt} werden können (\autoref{fig:high-aspect-ratio}). \textbf{Ist das Aspektverhältnis nicht optimal, könnten hohe Strukturen kippen oder Details verloren gehen}. Grundsätzlich gilt: Je größer das Aspektverhältnis und je kleiner die absolute Größe der Struktur, desto schwieriger wird die Fertigung \cite{madou2002, wolf2000, gerald2006, wikipedia_aspektverhaeltnis}.

\begin{figure}[htb!]
    \centering
    \includegraphics[width=.75\textwidth]{images/high-aspect-ratio.jpg} % Pfad und Dateiname des Bildes angeben
    \captionsetup{labelfont=bf, width=.85\textwidth} % Setzt die Bildnummer fett
    \caption{Kohlenstoff-MEMS-Strukturen mit hohem Aspektverhältnis (Höhe zu Breite) \cite{carbonMEMS}.}
    \label{fig:high-aspect-ratio}
\end{figure}

\vspace{1em}

\textbf{Staubpartikel} können die Struktur beeinflussen: Während Positivresisten weniger empfindlich auf Staub reagieren, können sie \textbf{bei Negativresisten Löcher verursachen}. Dies \textbf{liegt daran, dass Staubpartikel während der Belichtung als Schattenwurf fungieren}. Bei einem Negativresist (belichtete Bereiche werden unlöslich) führt ein solcher Schattenwurf dazu, dass der \textbf{betroffene Bereich unbelichtet und somit beim Entwickeln löslich bleibt}, was kleine Löcher im Resist hinterlässt \cite{madou2002, sciencedirect_photoresist}.

\textbf{Positivresisten} hingegen \textbf{reagieren weniger empfindlich auf Staub}, da bei ihnen die belichteten (und nicht die unbelichteten) Bereiche entfernt werden. Ein \textbf{Staubpartikel auf einem Positivresist verhindert zwar die Belichtung eines Punktes, aber dieser Bereich bleibt im Nachhinein bestehen}\footnote{Im schlimmsten Fall bleibt zu viel Photolack zurück. Nach dem Entfernen der Staubpartikel muss dann lediglich erneut belichtet werden, anstatt -- wie beim Negativresist -- neuer Photolack aufzutragen werden.}, während die belichteten, von Staub freien Bereiche entfernt werden. Daher entstehen in Positivresisten keine Löcher durch Staubpartikel \cite{wias_photoresist}.

\vspace{1em}

Die \textbf{``Passivierung bei Plasma-Ätzung'' und ``Nasschemie''} beschreiben, \textbf{wie gut der Resist die darunterliegende Schicht während der Ätzprozesse schützt}, was besonders bei empfindlichen Schichten wichtig ist \cite{schmid2024}. 

\vspace{1em}

\textbf{Plasma-Ätzung} ist ein trockenes Ätzverfahren, bei dem ionisiertes Gas (Plasma) genutzt wird, um Material gezielt von einer Oberfläche zu entfernen. Dabei entsteht eine physikalisch-chemische Reaktion zwischen dem Plasma und dem zu ätzenden Material, was präzise und kontrollierte Strukturierungen ermöglicht. Dieses Verfahren wird häufig in der Mikroelektronik eingesetzt, da es sehr feine Strukturen erlaubt \cite{madou2002, wolf2000}.

\vspace{1em}

\textbf{Nasschemie} hingegen nutzt flüssige Chemikalien, um Material durch chemische Reaktionen zu lösen und abzutragen. Dabei wird das Substrat in eine Lösung getaucht, die das Material selektiv auflöst. Diese Methode ist besonders verbreitet, weil sie einfach anzuwenden ist und sich gut für großflächige Ätzungen eignet, jedoch weniger präzise als Plasma-Ätzung ist \cite{rogers2008, gerald2006}.



% Lithographie -> Funktionsweise ------------------------------------------------------------------------------------------------------------------------------------------


\vspace{1em}

\subsection{Funktionsweise}

In diesem Abschnitt wird der Prozessablauf für die Lithographie, wie in \cite{schmid2024, gerald2006} beschrieben, erklärt. Zur besseren Veranschaulichung zeigt die \autoref{fig:lithography} einen vereinfachten Ablauf des Lithographieprozesses.

\begin{figure}[htb!]
    \centering
    \includegraphics[width=\textwidth]{images/lithography.png} % Pfad und Dateiname des Bildes angeben
    \captionsetup{labelfont=bf, width=\textwidth} % Setzt die Bildnummer fett
    \caption{Vereinfachter Ablauf des Lithographieprozesses mit Negativresist \cite{wias_photoresist}.}
    \label{fig:lithography}
\end{figure}

\begin{enumerate}
    \item \textbf{Reinigen und Trocknen des Substrats} (z. B. bei 300°C).
    \begin{itemize}
        \item \textbf{Ziel}: Sicherstellung einer guten Haftung zwischen Substrat und Photoresist.
    \end{itemize}

    \item \textbf{Optional}: Vorbereitung des Substrats zur Verbesserung der Resist-Haftung.
    \begin{itemize}
        \item \textbf{Ziel}: Falls der Resist schlecht auf $\mathrm{Si}$, $\mathrm{SiOH}$ (Siliziumhydroxid) oder $\mathrm{SiO}_2$ (Siliziumoxid bzw. Siliziumdioxid) haftet, Aufdampfen eines Haftvermittlers, z. B. Hexamethyldisilazan (HMDS).
    \end{itemize}

    \item \textbf{Aufschleudern des Resists} (mittels Spin-Coating in \autoref{fig:spin-coating}):
    \begin{itemize}
        \item \textbf{Ziel}: Gleichmäßiges Aufbringen des Resists mit definierter Dicke.
        \item Spin-Coating-Parameter: 900~U/min, Beschleunigung von 15 $\mathrm{U}/\mathrm{s}^2$, 30 Sekunden Spin-Dauer, ergibt eine Schichtdicke von ca. 1,6~µm.
    \end{itemize}

    \item \textbf{Softbake} (Resist-Prebake = Vorhärtung des Resists):
    \begin{itemize}
        \item \textbf{Ziel}: Bildung eines stabilen Resistfilms und weitgehende Entfernung von Lösungsmitteln.
        \item Durchführung im Umluftofen bei 90°C für 30 Minuten.
    \end{itemize}

    \item \textbf{Optional}: Randentlackung und erneutes Trocknen des Substrats.

    \item \textbf{Justierung und Belichtung}:
    \begin{itemize}
        \item \textbf{Ziel}: Übertragung der Maskenstruktur in den Resist.
        \item In der Produktion erfolgt eine automatische Justierung und Belichtung.
        \item In der Mikrostrukturtechnik wird oft eine doppelseitige Belichtung zur Bearbeitung der Waferrückseite angewendet.
    \end{itemize}

    \item \textbf{Schichthärtung} (Post-Exposure Bake):
    \begin{itemize}
        \item \textbf{Ziel}: Reduktion des Einflusses einer inhomogenen Belichtungsdosis durch stehende Wellen im Resist.
    \end{itemize}

    \item \textbf{Entwickeln des Resists}:
    \begin{itemize}
        \item \textbf{Ziel}: Herausbildung des Strukturmusters im Photoresist.
        \item Beispiel: Sprühentwicklung mit verdünntem Entwickler AZ 400 K (1:4), Entwicklungsdauer 150 Sekunden bei 38°C.
        \item Nach dem Entwickeln Spülen des Resists mit deionisiertem Wasser\footnote{Deionisiertes Wasser, auch DI-Wasser genannt, ist Wasser, aus dem gelöste Ionen entfernt wurden, um eine hohe Reinheit zu erreichen. Der Prozess der Deionisation entfernt Kationen wie $\mathrm{Na}^+$ und $\mathrm{Ca}^{2+}$ sowie Anionen wie $\mathrm{Cl}^-$ und $\mathrm{SO}_4^{2-}$ \cite{swatchuk2008, weast2004}. Deionisiertes Wasser wird durch Ionenaustausch gewonnen und weist eine geringe Leitfähigkeit auf. In der Mikroelektronik, Labortechnik und Medizin wird es aufgrund seiner Reinheit verwendet \cite{handbook2007}. In der Photolithographie wird DI-Wasser zum Spülen von Photolacken eingesetzt, um die Reinheit -- und somit die Qualität --  der Strukturen zu sichern \cite{peeters1994}.}.    
        \item Trocknung des Resists durch Trockenschleudern.
    \end{itemize}

    \item \textbf{Hardbake} (Resist-Postbake):
    \begin{itemize}
        \item \textbf{Ziel}: Verdampfung von Lösungsmittelresten und Verbesserung der Resisthaftung auf dem Wafer für folgende Ätz- oder Implantationsprozesse.
        \item Durchführung im Umluftofen bei 100-180°C für 20 Minuten.
    \end{itemize}

    \item \textbf{Optionaler Reinigungsschritt}:
    \begin{itemize}
        \item \textbf{Ziel}: Descum-Schritt im $\mathrm{O}_2$-Plasma zur Entfernung von Rückständen.
    \end{itemize}

    \item \textbf{Strukturübertragung} ins Substrat (Prozessschritt wie z.B. \textbf{Ätzen}):
    \begin{itemize}
        \item \textbf{Ziel}: Übertragung der Resiststruktur in die darunterliegende Schicht bzw. ins Substrat.
        \item Durchführung mittels Verfahren wie Ätzen, Abscheiden, Galvanik oder Ionenimplantation.
    \end{itemize}

    \item \textbf{Entfernen des Resists} (Strippen):
    \begin{itemize}
        \item \textbf{Ziel}: Entfernung der Resiststruktur nach der Strukturübertragung durch Strippen \\ (Lackstrippen).
        \item Nasschemisches Strippen mit Lösungsmitteln (z. B. Aceton) oder Veraschen im Sauerstoffplasma.
    \end{itemize}
\end{enumerate}

\begin{figure}[htb!]
    \centering
    \includegraphics[width=.65\textwidth]{images/spin-coating.png} % Pfad und Dateiname des Bildes angeben
    \captionsetup{labelfont=bf, width=.75\textwidth} % Setzt die Bildnummer fett
    \caption{Das Auftragen des Photoresists erfolgt mittels ``Spin-Coating'': a) Der Photoresist wird mittig auf den ruhenden Wafer aufgetragen. b) Der Wafer wird beschleunigt -- die zeitliche Änderung der Winkelgeschwindigkeit $\omega$ ist ungleich Null. c) Bei konstanter Winkelgeschwindigkeit wird der Photoresist aufgrund der Zentrifugalkräfte gleichmäßig an die Ränder des Wafers gedrückt und so auf der Oberfläche verteilt. d) Schließlich wird das Lösungsmittel mittels ``Softbake'' im Photoresist verdampft \cite{Sankapal2023}.}
    \label{fig:spin-coating}
\end{figure}

%Zusammengefasst:
%
%\begin{enumerate}
%    \item Wafer Reinigen
%    \item Wafer Spin-Coating mit Photoresist
%    \item Softbake (Prebake zum Entfernen des Lösungsmittels aus dem Photoresist)
%    \item Justiereung und Belichtung (Exposure)
%    \item Schichthärtung (Post-Exposure Bake)
%    \item Entwicklung
%    \item Hardbake (Postbake)
%    \item Strukturübertragung durch Ätzen
%    \item Resist entfernen (Strippen)
%\end{enumerate}



% Lithographie -> Belichtungsverfahren ------------------------------------------------------------------------------------------------------------------------------------


\vspace{1em}

\subsection{Belichtungsverfahren}

Zur Belichtung stehen \textbf{drei Verfahren} zur Verfügung, die in \textbf{zwei Hauptkategorien} unterteilt werden und im Folgenden detailliert beschrieben werden. Diese \textbf{Kategorien} sind die Schattenprojektion und die \textbf{abbildende Projektion (Projektions-Belichtung)}. Die \textbf{Schattenprojektion gliedert sich} dabei weiter \textbf{in} die \textbf{Kontaktbelichtung} und die \textbf{Proximity-Belichtung}.

\vspace{1em}

Ergänzend sei erwähnt, dass die \textbf{Masken} aus einer \textbf{Quarzplatte mit einer strukturierten Chromschicht}, dem sogenannten \textbf{Absorber}, bestehen. Aufgrund dieser Struktur werden sie \textbf{auch als Chrommasken bezeichnet} \cite{schmid2024, Mescheder2004}.

\vspace{1em}

Zur besseren Veranschaulichung zeigt \textbf{\autoref{fig:belichtung}} die drei Verfahren in vereinfachter Darstellung: a) die Kontaktbelichtung (Schattenprojektion), b) die Proximity-Belichtung (Schattenprojektion) und c) die Projektionsbelichtung (Abbildungsprojektion). Die \textbf{schwarzen „Balken“} zwischen der Maske und dem Photoresist bei den Schattenprojektionen sowie zwischen der Maske und der Optik bei der Abbildungsprojektion stellen die zuvor genannten \textbf{Absorber} dar \cite{schmid2024}.

\begin{figure}[htb!]
    \centering
    \includegraphics[width=\textwidth]{images/belichtung.png} % Pfad und Dateiname des Bildes angeben
    \captionsetup{labelfont=bf, width=\textwidth} % Setzt die Bildnummer fett
    \caption{Die drei Belichtungsverfahren in der Lithographie: a) Kontaktbelichtung, b) Proximity-Belichtung und c) Projektionsbelichtung. Die zu strukturierende Schicht ist hier leider nicht eingezeichnet. Diese befindet sich zwischen Wafer und Resist (z.B. Siliziumdioxid $\mathrm{SiO}_2$) \cite{cowburn1997}.}
    \label{fig:belichtung}
\end{figure}



% Lithographie -> Belichtungsverfahren -> Kontaktbelichtung ---------------------------------------------------------------------------------------------------------------


\vspace{1em}

\subsubsection{Kontaktbelichtung (Schattenprojektion)}

\vspace{1em}

Bei der Kontaktbelichtung wird \textbf{mittels parallelem Licht} eine \textbf{Maske im 1:1-Maßstab in den Photolack} auf einer Silizium-Oberfläche \textbf{übertragen} (\autoref{fig:kontaktbelichtung}) \cite{Mescheder2004}.

\begin{figure}[htb!]
    \centering
    \includegraphics[width=.65\textwidth]{images/kontaktbelichtung.png} % Pfad und Dateiname des Bildes angeben
    \captionsetup{labelfont=bf, width=\textwidth} % Setzt die Bildnummer fett
    \caption{Kontaktbelichtung im Detail \cite{schmid2024}.}
    \label{fig:kontaktbelichtung}
\end{figure}

\vspace{1em}

Dieses \textbf{Verfahren ist recht einfach} und bietet einige \textbf{Vorteile}: \textbf{Da} die \textbf{Maske auf dem Wafer aufliegt}, entstehen \textbf{kaum Abbildungsfehler}, und es können \textbf{sehr kleine Details bis zu} etwa \textbf{1~µm} präzise abgebildet werden. Ein \textbf{weiterer Vorteil} ist, dass \textbf{kein monochromatisches} (einfarbiges) Licht \textbf{benötigt} wird, sodass \textbf{hohe Lichtintensitäten und dadurch kürzere Belichtungszeiten möglich} sind. Da das \textbf{gesamte Muster in einem Schritt auf den gesamten Wafer belichtet} wird (ein sogenanntes \textbf{Full-Wafer-Verfahren}\footnote{Im Gegensatz zur Projektionsbelichtung, bei der nur ein Bereich nach dem anderen belichtet wird \cite{Mescheder2004}.}), ist es zudem schnell und \textbf{ermöglicht} einen \textbf{hohen Produktionsdurchsatz} \cite{schmid2024}.

\vspace{1em}

Allerdings gibt es auch \textbf{Nachteile}. \textbf{Da die Maske direkten Kontakt mit dem Wafer hat, entstehen leicht Defekte auf der Maske und im Photolack durch mechanische Beanspruchung} und Abrieb, was die \textbf{Lebensdauer der Maske} verkürzt, die \textbf{Defektdichte erhöht} und die \textbf{Kosten steigen} lässt \cite{schmid2024}. Weiterhin ist es \textbf{schwierig}, die \textbf{Maske gleichmäßig und flächig auf} dem \textbf{Wafer aufzulegen. Unterschiede in der Oberflächenstruktur des Wafers, Gasbildung oder eingeschlossene Luftblasen} führen zu \textbf{Abbildungsfehlern, da das Licht} an diesen Stellen \textbf{gestreut oder gebeugt wird.}

\vspace{1em}

\textbf{Zusätzlich kann sich der Wafer während der Fertigung leicht verziehen, was bedeutet, dass die Lage der Maskenmuster nicht immer exakt stimmt.} Solche \textbf{Verformungen im Mikrometerbereich lassen sich im Full-Wafer-Verfahren nicht einfach korrigiere}n und können das fertige Produkt beeinträchtigen \cite{schmid2024}.



% Lithographie -> Belichtungsverfahren -> Proximity-Belichtung ------------------------------------------------------------------------------------------------------------


\vspace{1em}

\subsubsection{Proximity-Belichtung (Schattenprojektion)}

\textbf{Ähnlich wie} bei der \textbf{Kontaktbelichtung} wird auch bei der Proximity-Belichtung eine \textbf{Maske im 1:1-Maßstab mittels parallelem Licht auf} den \textbf{Photolack} einer Silizium-Oberfläche \textbf{übertragen}. Bei diesem Verfahren besteht \textbf{jedoch ein geringer Abstand von etwa 10~µm bis 30~µm zwischen Maske und Wafer} (\autoref{fig:proximity-belichtung}) \cite{Mescheder2004}. Die Proximity-Belichtung hat sich in der \textbf{Mikrosystemtechnik} als \textbf{Standardverfahren} etabliert \cite{schmid2024}.

\begin{figure}[htb!]
    \centering
    \includegraphics[width=.4\textwidth]{images/proximity-belichtung.png} % Pfad und Dateiname des Bildes angeben
    \captionsetup{labelfont=bf, width=.5\textwidth} % Setzt die Bildnummer fett
    \caption{Proximity-Belichtung im Detail. Der Abstand zwischen Resist und Maske ist mit \( g \) (gap) bezeichnet, und \( b \) steht für die (minimale) Linienbreite des Absorbers \cite{schmid2024}.}
    \label{fig:proximity-belichtung}
\end{figure}

\vspace{1em}

Ein \textbf{wesentlicher Vorteil} dieser Methode ist die \textbf{hohe Lebensdauer der Maske}, da sie keinen direkten Kontakt mit dem Wafer hat. Zudem handelt es sich auch hier um ein \textbf{Full-Wafer-Verfahren}, was, wie bereits erwähnt, zu einem \textbf{sehr hohen Durchsatz} führt \cite{schmid2024}.

\vspace{1em}

Zu den \textbf{Nachteilen} zählt die \textbf{geringere Strukturauflösung}, die \textbf{durch} die \textbf{Beugung des Lichts an den Kanten der Maske begrenzt} wird. Wichtig ist also \textbf{minimale Linienbreite} $b$ in \autoref{fig:proximity-belichtung}. Sie lässt sich durch: 
$$
b \geq \sqrt{g\lambda}
$$ 
bestimmen, wobei $g$ der Abstand zwischen Wafer und Maske (gap) und $\lambda$ die Wellenlänge des Lichts ist (\autoref{fig:proximity-belichtung}). Bei einem Abstand von $g = 20 \, \mathrm{\mu m}$ und einer Wellenlänge von $\lambda = 400 \, \mathrm{nm}$ ergibt sich zum Beispiel $b \geq 3 \, \mathrm{\mu m}$ \cite{schmid2024}. 

\vspace{1em}

Um die \textbf{Auflösung} zu \textbf{verbessern}, können \textbf{kurzwelligeres Licht oder Phasenmasken}\footnote{Phasenmasken sind spezielle Masken in der Lithographie, die die Phase des Lichts verändern, um den Kontrast an den Kanten des projizierten Musters zu erhöhen. Dies erfolgt durch Phasenverschiebung mittels unterschiedlicher Materialdicken oder Brechungsindizes, was feinere Strukturen ermöglicht. Diese Technik hilft, die minimale Linienbreite zu reduzieren und eignet sich besonders für kürzere Wellenlängen \cite{Lin1992, Levenson1982, Mack2006}.} eingesetzt werden \cite{schmid2024}.

\vspace{1em}

Für \textbf{Auflösungen im Mikrometer-Bereich} muss der Abstand fast null betragen, was zur bereits erwähnten Kontaktbelichtung führt. Eine Alternative bietet die \textbf{Röntgenstrahllithographie}, die bei einem \textbf{Abstand von 30-50~µm} eine \textbf{Auflösung von unter 0,5~µm und} eine \textbf{hohe Tiefenschärfe}\footnote{Tiefenschärfe beschreibt in der Lithographie den Bereich entlang der optischen Achse, in dem ein Objekt scharf abgebildet wird. Eine größere Tiefenschärfe erlaubt eine dickere Schicht, in der das Muster scharf projiziert werden kann. Sie ist entscheidend für die präzise Übertragung feiner Details, da eine geringe Tiefenschärfe zu Unschärfen und damit zu einer beeinträchtigten Strukturqualität führen kann \cite{Mack2006}.} ermöglicht. Diese Technik wird auch im LIGA-Verfahren eingesetzt, um tiefe Lackstrukturen zu erzeugen \cite{Mescheder2004}.



% Lithographie -> Belichtungsverfahren -> Projektionsbelichtung -----------------------------------------------------------------------------------------------------------


\vspace{1em}

\subsubsection{Projektionsbelichtung (abbildende Projektion)}

Die Projektionsbelichtung nutzt ein \textbf{Abbildungssystem (Spiegel oder Linsen) zwischen Maske und Wafer}, um die \textbf{Maskenvorlage meist verkleinert auf} den \textbf{Wafer} zu \textbf{projizieren} (\autoref{fig:projektionsbelichtung}). \textbf{Übliche Maßstäbe sind 10:1 und 5:1}. Die Maske wird \textbf{außerdem nur abschnittsweise auf} den \textbf{Wafer projiziert. Oft enthält} die \textbf{Maske nur eine funktionale Einheit}\footnote{``Step and Repeat''-Verfahren \cite{schmid2024}.} (z.B. einen Chip oder Sensor) \cite{schmid2024,Mescheder2004}.

\begin{figure}[htb!]
    \centering
    \includegraphics[width=.45\textwidth]{images/projektionsbelichtung.png} % Pfad und Dateiname des Bildes angeben
    \captionsetup{labelfont=bf, width=.55\textwidth} % Setzt die Bildnummer fett
    \caption{Projektionsbelichtungssytem mit Öffnungswinkel $\alpha$ der Optik \cite{schmid2024}.}
    \label{fig:projektionsbelichtung}
\end{figure}

\vspace{1em}

Die \textbf{Auflösung} (minimal auflösbare Strukturgröße) wird \textbf{durch Beugung}, die \textbf{Wellenlänge} $\lambda$, den \textbf{Kohärenzgrad des Lichtes}, die \textbf{numerische Apertur}:
$$
\mathrm{N_A} = \sin{\alpha},
$$ wobei $\alpha$ der Öffnungswinkel der Optik ist (\autoref{fig:projektionsbelichtung}), \textbf{und das optische System begrenzt} \cite{schmid2024}. 

\vspace{1em}

Da \textbf{nur kleine Felder} (ca. $15 \times 15 \, \mathrm{mm^2}$) \textbf{belichtet} werden können, \textbf{muss der Wafer schrittweise bewegt werden, um} die \textbf{gesamte Si-Scheib}e (z.B. 6-Zoll-Wafer) \textbf{zu belichten} \cite{Mescheder2004,schmid2024}. Damit diese Anwendung funktioniert, muss ein Filter \textbf{monochromatisches Licht} erzeugen. Spiegel in der Belichtungsanlage erhöhen die Lichtintensität. Mit der \textbf{Projektionsbelichtung} ist der \textbf{Stand der Technik in der Mikroelektronik} die Belichtung von Strukturen \textbf{bis hinab zu 100 nm und darunter} \cite{schmid2024}.

\vspace{1em}

Die Projektionsbelichtung bietet einige \textbf{Vorteile}: Durch die \textbf{Verwendung vergrößerter Masken} (Reticles) ist die\textbf{ Herstellung einfacher und präziser} (weil der Maßstab nicht 1:1 ist), was eine \textbf{bessere Kontrolle ermöglicht}. Da sich \textbf{auf jeder Maske nur ein Chip} befindet, ist die \textbf{Fertigung (der Maske)} zudem \textbf{kostengünstiger}. Das Einzelchip-Belichtungsverfahren (``Step and Repeat'') \textbf{erlaubt eine genaue Justierung für jeden Chip} und \textbf{ermöglicht die Korrektur von nichtlinearem Waferverzug} \cite{schmid2024}. 

\vspace{1em}

\textbf{Nachteile} sind jedoch \textbf{die hohen Kosten der erforderlichen Geräte}, da extrem korrigierte Optiken mit \textbf{hoher numerischer Apertur ($\mathrm{N_A}$ bis zu 0,6) benötigt} werden. Die Verwendung von \textbf{monochromatischem Licht} kann Dispersion ausgleichen, \textbf{führt} jedoch \textbf{zu stehenden Wellen, die die Strukturgenauigkeit beeinträchtigen} können. Das \textbf{begrenzte Belichtungsfeld führt zu} einem \textbf{geringeren Durchsatz und erfordert wiederholte Justierung}. Die \textbf{geringe Tiefenschärfe schränkt} zudem das \textbf{Aspektverhältnis} der Strukturen \textbf{ein} \cite{schmid2024}.



% Lithographie -> SU-8 Photoresist ----------------------------------------------------------------------------------------------------------------------------------------


\vspace{1em}

\subsection{SU-8 Photoresist} % Resin 3D Druck?

Der Photoresist SU-8 ist ein \textbf{epoxidbasierter Negativ-Resist, der mit herkömmlicher UV-Photolitho-graphieausrüstung verwendet wird} und die \textbf{Herstellung von Strukturen bis zu einer Höhe von zwei Millimetern} ermöglicht -- \textbf{Röntgenstrahlung} ist dabei \textbf{nicht erforderlich}. SU-8 zeichnet sich durch ein \textbf{hohes Aspektverhältnis von bis zu 30:1} aus, was auf den \textbf{hohen Kontrast}\footnote{Der hohe Kontrast bei einem Photoresist wie SU-8 bedeutet, dass sich die belichteten und unbelichteten Bereiche sehr klar und scharf voneinander unterscheiden lassen. Das ist wie bei einem Bild mit kräftigen Farben: Die Unterschiede sind leicht zu erkennen. Bei SU-8 sorgt der hohe Kontrast dafür, dass die belichteten Stellen des Resists sehr genau und sauber härten, während die unbelichteten Bereiche weich bleiben und später entfernt werden können. 

Der Kontrast ist so hoch, weil SU-8 auf einem speziellen Epoxidharz basiert, das besonders gut auf UV-Licht reagiert. Dadurch entstehen scharfe, klare Strukturen, die nicht ``verwischen'', was bei der Herstellung von sehr hohen und präzisen Strukturen hilft \cite{Lorenz1997, Jensen2002, DelCampo2007}.} des Resists zurückzuführen ist und die \textbf{Realisierung hoher Strukturen} erlaubt. Zudem besitzt SU-8 eine \textbf{hohe chemische Resistenz} und kann daher \textbf{auch als Ätzmaske eingesetzt} werden \cite{schmid2024}. Mit Hilfe von SU-8 entwickelte Strukturen sind in der \autoref{fig:su8-1}, der \autoref{fig:su8-2} und der \autoref{fig:su8-3} dargestellt. 

\begin{figure}[htb!]
    \centering
    \includegraphics[width=.65\textwidth]{images/su8-1.jpg} % Pfad und Dateiname des Bildes angeben
    \captionsetup{labelfont=bf, width=.75\textwidth} % Setzt die Bildnummer fett 
    \caption{Nach Lithographie mit SU-8 Photoresist entstandene Wände mit sehr hohem Aspektverhältnis \cite{SU8Photoresist}.}
    \label{fig:su8-1}
\end{figure}

\begin{figure}[htb!]
    \centering
    \includegraphics[width=.6\textwidth]{images/su8-2.jpg} % Pfad und Dateiname des Bildes angeben
    \captionsetup{labelfont=bf, width=.7\textwidth} % Setzt die Bildnummer fett
    \caption{Nach Lithographie mit SU-8 Photoresist entstandene Säulen \cite{Tian2005}.}
    \label{fig:su8-2}
\end{figure}

\begin{figure}[htb!]
    \centering
    \includegraphics[width=.75\textwidth]{images/su8-3.jpg} % Pfad und Dateiname des Bildes angeben
    \captionsetup{labelfont=bf, width=.85\textwidth} % Setzt die Bildnummer fett
    \caption{Nach Lithographie mit SU-8 Photoresist entstandene Strukturen und entstandenes Logo von KemLab Inc. \cite{KemLab2023}.}
    \label{fig:su8-3}
\end{figure}

Zusammengefasst bietet der Photoresist SU-8 zahlreiche \textbf{Vorteile}: Er \textbf{kann durch Aufschleudern aufgetragen werden}, wobei seine \textbf{hohe Viskosität} die \textbf{Herstellung dicker Schichten ermöglicht}. SU-8 erreicht \textbf{Aspektverhältnisse von über 20:1} und \textbf{haftet gut auf Silizium und Glas}, insbesondere bei kleinen oder niedrigen Strukturen. Der Resist zeichnet sich durch \textbf{hohe chemische Stabilität und biologische Kompatibilität} aus, was ihn \textbf{für optische, mechanische und Fluidik-Elemente sowie Galvanikformen\footnote{Siehe LIGA-Verfahren.} ideal} macht. Zudem ist SU-8 im Vergleich zu Prozessen wie DRIE und LIGA \textbf{kosten- und zeiteffizient} und liefert ähnliche Ergebnisse \cite{schmid2024, Kayaku2020}.

\vspace{1em}

Zu den \textbf{Nachteilen} zählen: Die \textbf{hohe Viskosität erschwert} das \textbf{Handling} im Vergleich zu Standardlacken. \textbf{Nach dem Auftragen erfordert SU-8} ein \textbf{Postbake, das zu einem Volumenschwund von etwa 7,5 \% führt} und \textbf{dadurch innere Spannungen} sowie eine \textbf{reduzierte Haftung verursacht}. Daher ist SU-8 \textbf{für großflächige Strukturen weniger geeignet} \cite{schmid2024, Kayaku2020}.



% Lithographie -> Lift-Off Verfahren --------------------------------------------------------------------------------------------------------------------------------------


\vspace{1em}

\subsection{Lift-Off-Verfahren}

Das Lift-Off-Verfahren ist eine \textbf{Methode zur Strukturierung metallischer Dünnfilme ohne Ätzen}. Zunächst wird ein \textbf{Photoresist auf} den \textbf{Wafer aufgebracht und strukturiert}, um die gewünschten Muster zu definieren. \textbf{Danach wird} das \textbf{Metall ganzflächig abgeschieden}, wobei es sowohl \textbf{auf dem freigelegten Substrat als auch auf dem Photoresist landet}. \textbf{Entscheidend} ist, dass die \textbf{Flanken des Resists steil oder negativ} sind, \textbf{damit das Metall auf diesen Flächen nicht haften bleibt}. \textbf{Anschließend} wird der \textbf{Photoresist in einem Lösungsmittelbad entfernt} (gestrippt), wobei \textbf{das darauf liegende Metall ebenfalls abgelöst} wird, während das \textbf{Metall auf dem Substrat die definierte Struktur} bildet \cite{madou2002, smith1997, kern1990}. Der Unterschied zwischen dem zuvor beschriebenen Standardverfahren und dem Lift-Off-Verfahren in der Lithographie wird in \autoref{fig:lift-off} veranschaulicht. Beispiele dieses Verfahrens unter dem Elektronenmikroskop sind in \autoref{fig:lift-off irl} und \autoref{fig:mlo-irl} dargestellt.

\begin{figure}[htb!]
    \centering
    \includegraphics[width=\textwidth]{images/lift-off.jpg} % Pfad und Dateiname des Bildes angeben
    \captionsetup{labelfont=bf, width=\textwidth} % Setzt die Bildnummer fett
    \caption{Vergleich des Standardverfahrens in der Lithographie mit Negativresist (links) und dem Verfahren mittels Lift-Off (rechts) \cite{ZeloofWebsite}.}
    \label{fig:lift-off}
\end{figure}

\begin{figure}[htb!]
    \centering
    \includegraphics[width=\textwidth]{images/lift-off irl.png} % Pfad und Dateiname des Bildes angeben
    \captionsetup{labelfont=bf, width=\textwidth} % Setzt die Bildnummer fett
    \caption{Mit dem Lift-Off-Verfahren hergestellte Kupferbahn auf einem Wafer. Links: Wafer mit strukturiertem Photoresist, wobei die negativen Flanken gut zu erkennen sind. Mitte: Querschnitt der gleichen Anordnung nach dem Sputtern. Auf dem Photoresist und dem Wafer wurde Kupfer abgeschieden. Rechts: Endresultat nach dem Strippen -- eine Kupferbahn \cite{jsr_lift_resists}.}
\label{fig:lift-off irl}
\end{figure}

\begin{figure}[htb!]
    \centering
    \includegraphics[width=.65\textwidth]{images/mlo-irl.jpg} % Pfad und Dateiname des Bildes angeben
    \captionsetup{labelfont=bf, width=.75\textwidth} % Setzt die Bildnummer fett
    \caption{Metallbahnen auf einem Wafer, hergestellt durch das MLO-Verfahren (MLO, Metal-Lift-Off, entspricht dem klassischen Lift-Off-Verfahren). Im oberen Bild ist der Photoresist mit negativen Flanken zu sehen, der mit Metall beschichtet wurde. Das untere Bild zeigt das Endresultat nach dem Entfernen des Photoresists mit einem geeigneten Lösungsmittel \cite{ClassOneLiftOff}.}
    \label{fig:mlo-irl}
\end{figure}



%Lift-Off Lösungsmittel & Aceton & Methylethylketon \\ \hline
% \vspace{1em}
 
%\textbf{``Lift-off''} bezeichnet das \textbf{Verfahren zum Entfernen des Resists} und hängt von den eingesetzten Chemikalien ab, wie Aceton oder Methylethylketon \cite%{wolf2000, rogers2008}. 



% Ätztechnik --------------------------------------------------------------------------------------------------------------------------------------------------------------


\clearpage
\section{Ätztechnik}

Die Ätztechnik ist ein zentrales Verfahren in der Mikrosystemtechnik, um \textbf{gezielt Material von der Oberfläche eines Festkörpers zu entfernen} und feine Strukturen zu erzeugen. Dabei kann das Ätzen entweder nasschemisch oder trocken erfolgen. \textbf{Beim nasschemischen Ätzen} werden \textbf{chemische Reagenzien verwendet, um} die \textbf{Oberfläche eines Materials gezielt abzutragen}. Das \textbf{Trockenätzen} hingegen \textbf{arbeitet mit Teilchen hoher kinetischer Energie}, die durch \textbf{physikalisch-chemische Prozess}e die \textbf{Oberfläche angreifen und Material abtragen}. Beide Methoden bieten spezifische Vorteile und Herausforderungen und werden in der Mikrosystemtechnik je nach Anforderung eingesetzt \cite{schmid2024Aetzen}.



% Ätztechnichk -> Begriffsklärung für die Ätztechnik ----------------------------------------------------------------------------------------------------------------------


\vspace{1em}

\subsection{Begriffsklärung für die Ätztechnik}

Wie in den vorherigen Kapiteln erwähnt, werden bereits erklärte Begriffe hier nicht erneut behandelt.

\begin{itemize}
    \item \textbf{Ätzmasken} (z.B. Photoresist nach Hardbake): Um gezielte Ätzprofile zu erzielen, werden Masken oder Ätzmasken eingesetzt. Eine Ätzmaske ist eine Schutzschicht, die bestimmte Bereiche der Oberfläche vor dem Ätzprozess bewahrt und damit festlegt, wo das Material abgetragen wird. Sie besteht oft aus Materialien, die eine hohe Resistenz gegenüber dem verwendeten Ätzmittel aufweisen \cite{madou2002, kern1983, wolf2000, sze1988, plummer2000}.
    \item \textbf{KOH-Ätzen}: Eine spezifische Form des anisotropen Ätzens ist das KOH-Ätzen, bei dem Kaliumhydroxid (KOH) als Ätzmittel verwendet wird. Beim KOH-Ätzen von Silizium weisen die \{111\}-Kristallebenen die kleinste Ätzrate auf, was zu \textbf{Strukturen mit definierten Kanten und Winkeln} führt. Dies ermöglicht \textbf{präzise Geometrien}, die in der Mikrosystemtechnik besonders wichtig sind \cite{madou2002, seidel1990, williams2003, li1997}.
    \item \textbf{Miller'sche Inidzes und Kristallebenen}: In einem Kristall wie Silizium (\autoref{fig:Si-Kristall}) sind die Atome in einem regelmäßigen 3D-Gitter angeordnet. Diese Anordnung ermöglicht es, das Material entlang bestimmter Schichten oder ``Kristallebenen'' zu schneiden. Diese Ebenen werden durch Miller'sche Inidzes wie \{100\}, \{110\} oder \{111\} beschrieben. Die Zahlen zeigen, wie die Schnitte in Bezug auf das Gitter verlaufen \cite{kittel2004, ashcroft1976, miller1839, hull1999, hammond2009}. Zur Veranschaulichung zeigt \autoref{fig:Richtungen} wichtige Richtungen im kubischen Kristallsystem, während \autoref{fig:Miller} die wesentlichen Ebenen in kubischen Elementarzellen darstellt. Diese Ebenen werden durch die Miller'schen Indizes beschrieben.

    Die \{111\}-Ebenen in Silizium sind besonders stabil und schwer zu ätzen, da die Atome in einer dichten, dreieckigen Anordnung vorliegen. Diese enge Packung macht es schwierig, Atome herauszulösen, was zu einer geringeren Ätzrate führt \cite{kittel2004, ashcroft1976}. Beim KOH-Ätzen wird daher die \{111\}-Ebene langsamer abgetragen als andere Kristallebenen, wodurch schräge Wände und präzise Winkel entstehen \cite{seidel1990}. Diese stabilen \{111\}-Ebenen werden gezielt genutzt, um Strukturen mit scharfen Kanten und exakten Winkeln zu erzeugen \cite{madou2002}.

    \textbf{Zusammengefasst}: Die Bezeichnungen \{100\}, \{110\}, \{111\} usw. geben an, wie die Kristallebenen im Gitter verlaufen und wie stabil sie sind. Die Wahl der Kristallebene beeinflusst die Ätzrate und ist entscheidend für die präzise Strukturierung in der Mikrotechnik \cite{kittel2004, ashcroft1976, madou2002}.

    \item \textbf{\{100\}-Silizium-Wafer und \{110\}-Silizium-Wafer}: Diese Waferarten \textbf{unterscheiden sich durch ihre kristallografische Orientierung}, die durch die \textbf{Miller-Indizes} \{100\} bzw. \{110\} beschrieben wird. Diese Indizes geben an, wie die Atomschichten im Kristallgitter des Siliziums zur Wafer-Oberfläche orientiert sind.

    \textbf{\{100\}-Silizium-Wafer}:
    \begin{itemize}
        \item Die Oberfläche eines \{100\}-Wafers ist parallel zur \{100\}-Ebene des Kristallgitters, was bedeutet, dass die Silizium-Atome in einer quadratischen Anordnung vorliegen.
        \item Diese Orientierung wird häufig in der Halbleiterfertigung verwendet, da sie gute mechanische und elektrische Eigenschaften aufweist.
        \item Beim anisotropen KOH-Ätzen bilden die \{111\}-Ebenen (die schräg zur \{100\}-Ebene verlaufen und eine niedrigere Ätzrate aufweisen) typische schräge Wände mit einem Winkel von 54,74°. Dies ist nützlich für die Herstellung präziser Strukturen in der Mikrosystemtechnik \cite{madou2002}.
    \end{itemize}
    
    \textbf{\{110\}-Silizium-Wafer}:
    \begin{itemize}
        \item Bei einem \{110\}-Wafers ist die Oberfläche parallel zur \{110\}-Ebene des Kristallgitters. Diese Anordnung führt dazu, dass die Atome dichter gepackt und in länglichen Reihen angeordnet sind.
        \item Diese Orientierung bietet andere mechanische Eigenschaften und wird in bestimmten Anwendungen bevorzugt, wo besondere strukturelle Stabilität erforderlich ist.
        \item Beim anisotropen KOH-Ätzen ergibt die Orientierung des \{110\}-Wafers schräge Wände mit einem kleineren Winkel von 35,26°, da die \{111\}-Ebenen hier anders zur Oberfläche verlaufen als beim \{100\}-Wafer. Dieser Winkel ist besonders für präzise Strukturierungen in der Mikrotechnik interessant \cite{seidel1990}.
    \end{itemize}
    
    \textbf{Zusammengefasst}: Die Wahl zwischen einem \{100\}- und einem \{110\}-Silizium-Wafer beeinflusst die Struktur der bearbeiteten Oberflächen und die entstehenden Winkel beim anisotropen Ätzen. Die \{100\}-Orientierung erzeugt Winkel von 54,74° und ist für allgemeine Halbleiteranwendungen weit verbreitet, während die \{110\}-Orientierung mit einem Winkel von 35,26° für spezielle strukturelle Anforderungen verwendet wird \cite{kittel2004, ashcroft1976}.
\end{itemize}

\begin{figure}[htb!]
    \centering
    \includegraphics[width=.45\textwidth]{images/si-crystal.png} % Pfad und Dateiname des Bildes angeben
    \captionsetup{labelfont=bf, width=.55\textwidth} % Setzt die Bildnummer fett
    \caption{Elementarzelle des Siliziumkristalls \cite{wilderness_silicon}.}
    \label{fig:Si-Kristall}
\end{figure}

\begin{figure}[htb!]
    \centering
    \includegraphics[width=\textwidth]{images/Elementarzelle Richtungen.png} % Pfad und Dateiname des Bildes angeben
    \captionsetup{labelfont=bf, width=\textwidth} % Setzt die Bildnummer fett
    \caption{Wichtige Richtungen im kubischen Kristallsystem: Die Richtung [111] entsteht durch die vektorielle Zusammensetzung aus jeweils einem Schritt in x-, y- und z-Richtung \cite{grundlagen_konstruktionswerkstoffe}.}
    \label{fig:Richtungen}
\end{figure}

\begin{figure}[htb!]
    \centering
    \includegraphics[width=.85\textwidth]{images/Elementarzelle Miller'sche Indizes.png} % Pfad und Dateiname des Bildes angeben
    \captionsetup{labelfont=bf, width=.95\textwidth} % Setzt die Bildnummer fett
    \caption{Darstellung wichtiger Ebenen in kubischen Elementarzellen. Beispiel: Die (100)-Ebene schneidet die x-Achse bei 1 und verläuft parallel zur y- und z-Achse, da sie diese in der Unendlichkeit ($\infty$) schneidet. Die Miller'schen Indizes dieser Ebene ergeben sich aus den Kehrwerten der Schnittpunkte mit den Achsen: $1/1$ für x, $1/\infty$ für y und $1/\infty$ für z, woraus die Indizes (100) resultieren \cite{grundlagen_konstruktionswerkstoffe}.}
    \label{fig:Miller}
\end{figure}



% Ätztechnichk -> Ätzrate -------------------------------------------------------------------------------------------------------------------------------------------------


\vspace{1em}

\subsection{Ätzrate}

Die Ätzrate beschreibt, \textbf{wie schnell Material während eines Ätzprozesses abgetragen wird}. Sie wird als die \textbf{Menge des entfernten Materials pro Zeiteinheit} angegeben, oft in Einheiten wie \textbf{Nanometer pro Minute} (nm/min) oder \textbf{Mikrometer pro Stunde} ($\mu$m/h). Die Ätzrate \textbf{hängt von verschiedenen Faktoren ab}, darunter das verwendete \textbf{Ätzmittel}, die \textbf{Temperatur} und die \textbf{Kristallrichtung des Materials}. Eine kontrollierte Ätzrate ist entscheidend für die Herstellung präziser Strukturen \cite{madou2002, schmid2024Aetzen}. 



% Ätztechnichk -> Ätzprofile ----------------------------------------------------------------------------------------------------------------------------------------------


\vspace{1em}

\subsection{Ätzprofile}

In der Mikrosystemtechnik gibt es verschiedene Ätzprofile, \textbf{die sich durch das Verhalten der Ätzrate in verschiedenen Raumrichtungen unterscheiden und die Form der erzeugten Strukturen bestimmen}. Diese Ätzprofile werden als \textbf{isotrop oder anisotrop} klassifiziert \cite{schmid2024Aetzen}. Die \autoref{fig:aetzprofile} liefert einen Überblick über diese Profile. 

\begin{figure}[htb!]
    \centering
    \includegraphics[width=.75\textwidth]{images/aetzprofile.png} % Pfad und Dateiname des Bildes angeben
    \captionsetup{labelfont=bf, width=.85\textwidth} % Setzt die Bildnummer fett
    \caption{Darstellung der verschiedenen Ätzprofile: \protect\circled{1} isotrop, \protect\circled{2} anisotrop durch Trockenätzen und \protect\circled{3} anisotrop durch KOH-Ätzen \cite{schmid2024Aetzen}.}
    \label{fig:aetzprofile}
\end{figure}

\paragraph{\protect\circled{1} \uline{Isotropes Ätzprofil:}} Ein isotropes Ätzprofil entsteht, wenn die \textbf{Ätzrate in alle Raumrichtungen gleich groß} ist. Dadurch entstehen \textbf{abgerundete, viertelkreisförmige Strukturen}, da das \textbf{Material gleichmäßig in} die \textbf{Tiefe und Breite abgetragen} wird \cite{schmid2024Aetzen}.

\paragraph{\protect\circled{2} \uline{und} \protect\circled{3} \uline{Anisotrope Ätzprofile:}} Bei einem anisotropen Ätzprofil \textbf{hängt die Ätzrate von der Raumrichtung ab}, was \textbf{gezielte Strukturen mit steilen Kanten} ermöglicht. Es gibt verschiedene Ansätze für anisotropes Ätzen, wie das \textbf{Trockenätzen} \protect\circled{2}, bei dem \textbf{Teilchen hoher kinetischer Energie} gezielt \textbf{entlang einer Richtung die Oberfläche angreifen}, und das \textbf{KOH-Ätzen} \protect\circled{3}. Beim KOH-Ätzen weisen die \{111\}-Kristallebenen die \textbf{geringste Ätzrate} auf, wodurch \textbf{natürliche schräge Wände und präzise Winkel} entstehen. Diese \textbf{gezielte Richtungsabhängigkeit} ist in der Mikrosystemtechnik besonders wichtig, um Strukturen mit scharfen Kanten und definierten Winkeln zu erzeugen \cite{schmid2024Aetzen, madou2002}.



% Ätztechnichk -> Ätzprofile -> Anisotropiefaktor -------------------------------------------------------------------------------------------------------------------------


\vspace{1em}

\subsubsection{Anisotropiefaktor}

Der Anisotropiefaktor $A_f$ ist ein \textbf{Maß für die Richtungsempfindlichkeit des Ätzvorgangs} (\autoref{fig:af}). Er beschreibt das Verhältnis der \textbf{lateralen Ätzrate} $R_l$ zur \textbf{vertikalen Ätzrate} $R_v$ indirekt durch die Formel:
$$
A_f = 1 - \frac{R_l}{R_v}.
$$
Ein \textbf{vollständig anisotroper Ätzprozess}, bei dem $R_l = 0$, ergibt $A_f = 1$ und führt zu einer \textbf{exakt senkrechten Struktur}. Ein \textbf{isotroper Ätzvorgang} hingegen hat $A_f = 0$, was auf \textbf{gleiche Ätzraten in alle Richtungen} hinweist \cite{schmid2024Aetzen}.

\begin{figure}[htb!]
    \centering
    \includegraphics[width=.8\textwidth]{images/af.jpeg} % Pfad und Dateiname des Bildes angeben
    \captionsetup{labelfont=bf, width=.9\textwidth} % Setzt die Bildnummer fett
    \caption{Veranschaulichung des Anisotropiefaktors \cite{schmid2024Aetzen}.}
    \label{fig:af}
\end{figure}



% Ätztechnichk -> Ätzprofile -> Selektivität ------------------------------------------------------------------------------------------------------------------------------


\vspace{1em}

\subsubsection{Selektivität}

Die Selektivität eines Ätzprozesses beschreibt das \textbf{Verhältnis der Ätzrate der zu ätzenden Schicht $R_1$ zur Ätzrate der darunterliegenden Schicht (z.B. Si-Substrat)} $R_2$, definiert durch:
$$
S = \frac{R_1}{R_2}.
$$
Eine hohe Selektivität bedeutet, dass die zu ätzende Schicht schneller abgetragen wird als die darunterliegende Schicht, sodass diese gut erhalten bleibt. \textbf{Im Allgemeinen ist eine hohe Selektivität erwünscht}, um einen geringen Abtrag der darunterliegenden Schicht zu erreichen (\autoref{fig:selectivity}) \cite{schmid2024Aetzen}.

\begin{figure}[htb!]
    \centering
    \includegraphics[width=.8\textwidth]{images/selectivity.jpeg} % Pfad und Dateiname des Bildes angeben
    \captionsetup{labelfont=bf, width=.9\textwidth} % Setzt die Bildnummer fett
    \caption{Veranschaulichung der Selektivität \cite{schmid2024Aetzen}.}
    \label{fig:selectivity}
\end{figure}



% Ätztechnichk -> Ätzprofile -> Ätzgraben bei anisotroper KOH-Ätzung von 100- und 110-Si-Wafern ---------------------------------------------------------------------------


\vspace{1em}

\subsection{Ätzgraben bei anisotroper KOH-Ätzung von \{100\}- und \{110\}-Si-Wafern}

In diesem Abschnitt wird erläutert, wie \textbf{unterschiedliche Ätzgräben} ausschauen können. Die \autoref{fig:koh-aetzgraeben} zeigt Ätzgräben, die durch \textbf{anisotropes KOH-Ätzen (Ätzrichtung: von ``oben'' nach ``unten'')} in zwei \textbf{verschiedenen Silizium-Wafer-Typen} erzeugt wurden: einen \{100\}- und einen \{110\}-Silizium-Wafer (siehe Begriffsklärung für die Ätztechnik). Beim anisotropen Ätzen ist die \textbf{Ätzrate richtungsabhängig}. Die \textbf{KOH-Lösung} (Kaliumhydroxid) \textbf{ätzt bestimmte Kristallebenen im Silizium langsamer als andere, was spezifische Winkel und geometrische Strukturen erzeugt}.

\begin{figure}[htb!]
    \centering
    \includegraphics[width=\textwidth]{images/koh-aetzgraeben.png} % Pfad und Dateiname des Bildes angeben
    \captionsetup{labelfont=bf, width=\textwidth} % Setzt die Bildnummer fett
    \caption{KOH-Ätzgräben in einer a) \{100\}- und b) \{110\}-Siliziumscheibe.}
    \label{fig:koh-aetzgraeben}
\end{figure}

\subparagraph{a) \uline{\{100\}-Silizium-Wafer:}} Im \{100\}-orientierten Silizium-Wafer verläuft der \textbf{Boden des Ätzgrabens} entlang der \textbf{\{100\}-Ebene}, während die \textbf{Seitenwände} entlang der \textbf{\{111\}-Kristallebe-nen} verlaufen, die eine \textbf{geringere Ätzrate} haben und daher \textbf{als schräge Flächen erhalten} bleiben. Der \textbf{Winkel zwischen} der \textbf{\{100\}- und} der \textbf{\{111\}-Ebene} beträgt \textbf{54,74°}. Dieser Winkel \textbf{ist charakteristisch} für das KOH-Ätzen in \{100\}-Silizium und wird \textbf{durch} die \textbf{Anordnung der Atome} entlang der \{111\}-Ebene \textbf{bestimmt} \cite{madou2002}.

\subparagraph{b) \uline{\{110\}-Silizium-Wafer:}} Beim \{110\}-orientierten Silizium-Wafer verlaufen die eitenwände ebenfalls entlang der \{111\}-Ebenen. Im Gegensatz zum \{100\}-Wafer beträgt der \textbf{Winkel zwischen} der \textbf{\{110\}-Ebene (oben) und} der \textbf{\{111\}-Ebene (Seitenwände)} jedoch \textbf{35,26°}. Dieser Winkel \textbf{ergibt sich aus} der \textbf{unterschiedlichen Kristallstruktur des \{110\}-Wafers} im Vergleich zum \{100\}-Wafers und führt zu einer \textbf{anderen Orientierung der \{111\}-Ebenen relativ zur Oberfläche} \cite{seidel1990, kittel2004}.

\paragraph{Zusammenfassung:} Die \autoref{fig:koh-aetzgraeben} zeigt, wie die Kristallorientierung des Siliziums die Struktur der Ätzprofile beeinflusst. Der Winkel zwischen den Ebenen hängt von der Kristallorientierung ab, und das anisotrope KOH-Ätzen nutzt dies, um präzise Geometrien und schräge Wände zu erzeugen. Die \{111\}-Ebenen bieten aufgrund ihrer geringeren Ätzrate die stabilen Wände, während die Orientierung des Wafers (entweder \{100\} oder \{110\}) die spezifischen Winkel der Grabenwände bestimmt \cite{madou2002, seidel1990, kittel2004}.



% END ---------------------------------------------------------------------------------------------------------------------------------------------------------------------


\thispagestyle{empty}
\newpage
\bibliographystyle{ieeetr}  % IEEE-Standard für das Literaturverzeichnis
\bibliography{references}  % Ersetze "your_bib_file" durch den tatsächlichen Namen der .bib-Datei ohne die Endung .bib

\end{document}