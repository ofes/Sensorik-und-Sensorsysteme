\documentclass{article} % Dokumenttyp "article" für Artikel

\usepackage[utf8]{inputenc}  % UTF-8-Zeichencodierung
\usepackage[T1]{fontenc}     % Europäische Zeichenunterstützung
\usepackage[ngerman]{babel}  % Deutsche Sprachunterstützung
\usepackage{graphicx} % Am Anfang des Dokuments einfügen, um Grafikbefehle zu laden
\usepackage{caption}
\usepackage[colorlinks=true, linkcolor=blue]{hyperref}
\hypersetup{citecolor = blue}
\usepackage{sectsty}
\usepackage{cite}
\usepackage{ulem}

\usepackage[a4paper, left=3cm, right=3cm, top=2.5cm, bottom=2.5cm]{geometry} % A4-Format mit definierten Rändern

\setlength{\footskip}{2cm}       % Abstand zwischen Text und Fußnotenlinie
\setlength{\skip\footins}{1cm}    % Abstand zwischen Fußnotenlinie und -text
\interfootnotelinepenalty = 10000   % Verhindert das Teilen von Fußnoten über Seiten
\setlength{\parindent}{0.0em} % Setzt die Einrückung auf 1em
\sectionfont{\LARGE}      % Setzt \section auf \Large (eine Stufe größer als Standard)
\subsectionfont{\Large}   % Setzt \subsection auf \large (ebenfalls eine Stufe größer)

\subsubsectionfont{\large}

\title{Zusammenfassung Sensorik und Sensorsysteme (TU Wien)}
\author{Omar Filip El Sendiouny \\ \texttt{e1616023@student.tuwien.ac.at}}
\vfill
\date{\today}

\begin{document}

% Titelblatt mit Bild
\begin{titlepage}
    \centering
    \vspace{2cm}
    {\Huge \bfseries Zusammenfassung Sensorik und Sensorsysteme (TU Wien) \par}
    
    \vspace{1cm}
    {\Large Omar Filip El Sendiouny \\ \texttt{e1616023@student.tuwien.ac.at} \par}

    \vspace*{2cm}

    % Bild hinzufügen, Breite anpassen nach Bedarf
    \includegraphics[width=0.75\textwidth]{images/mems-module-hero.png}
    
    % Bild hinzufügen, Breite anpassen nach Bedarf
    \includegraphics[width=0.75\textwidth]{images/mems-gears.png} 

    \vspace*{2cm} 
    
    \vfill
    
    {\large \today \par}
\end{titlepage}

\tableofcontents
\listoftables
\listoffigures
\thispagestyle{empty}
\newpage

\pagenumbering{arabic} % Beginnt die Nummerierung ab hier

\section*{Einleitung} % Einleitung
Diese Zusammenfassung der Vorlesung \textit{Sensorik und Sensorsysteme} (366.071) an der TU Wien dient als Lernhilfe und wurde sorgfältig erstellt. Dennoch sind 
Fehler möglich, da sie menschlich sind. Leserinnen und Leser sind daher eingeladen, auf eventuelle Unstimmigkeiten zu achten und diese an 
\texttt{e1616023@student.tuwien.ac.at} zu melden. Alternativ können Korrekturen direkt im Repository auf GitHub vorgenommen werden: \\ 
\url{https://github.com/ofes/sensorik-und-sensorsysteme}. Bei Bedarf einer Zugriffsberechtigung gerne eine Nachricht an \url{https://github.com/ofes} senden.

\vspace{1em}

Im ersten Abschnitt dieser Zusammenfassung werden grundlegende Herstellungsmethoden der Mikrosystemtechnik erläutert. Anschließend folgt eine Vorstellung 
ausgewählter MEMS-Sensoren (mikroelektromechanische Systeme) sowie ihrer Funktionsweisen.

\vspace{1em}

Viel Erfolg beim Lernen!

\vspace{1em}
\section{Pysical Vapour Deposition} % PVD
PVD (Physical Vapour Deposition) umfasst alle \textbf{physikalischen Verfahren zur Abscheidung dünner Schichten aus der Dampfphase}.
Dabei werden die Ausgangsmaterialien der Schichten durch physikalische Prozesse wie Erhitzen in die Dampfphase überführt und anschließend 
auf dem Zielobjekt abgeschieden. Die \textbf{typischen Schichtdicken} liegen \textbf{zwischen 1~nm und 1~µm} und können 
\textbf{in Ausnahmefällen bis zu 10~µm} erreichen \cite{keplinger2024}.

\vspace{1em}
\subsection{Begriffsklärung für PVD} % PVD Begriffsklärung
Dieser Abschnitt bietet eine Einführung in wichtige, häufig verwendete Begriffe. Es empfiehlt sich, zunächst die nachfolgenden Abschnitte zu lesen 
und bei Bedarf hier nach unbekannten Begriffen zu suchen.

\begin{itemize}
    \item \textbf{Monolage}: Eine Monolage (auch Monoschicht genannt) beschreibt eine Schicht, die nur aus einer einzelnen Lage von Atomen oder 
    Molekülen besteht \cite{kittel2004}.
    \item \textbf{Freie Weglänge}: Die freie Weglänge eines Atoms oder Moleküls in einem Plasma oder Gas ist die durchschnittliche Strecke, 
    die ein Teilchen (wie ein Atom, Ion oder Molekül) zurücklegt, bevor es mit einem anderen Teilchen kollidiert. Diese Weglänge hängt stark 
    von der Teilchendichte und dem Durchmesser der beteiligten Teilchen ab. \cite{kittel2004}.
    \item \textbf{Verdampfen}: Bezeichnet den Übergang eines Stoffes vom flüssigen in den gasförmigen Zustand \cite{kittel2004}.
    \item \textbf{Sublimieren}: Beschreibt den direkten Übergang eines Stoffes vom festen in den gasförmigen Zustand, ohne eine flüssige Phase 
    zu durchlaufen \cite{kittel2004}.
    \item \textbf{Rezipient}: Vakuumkammer \cite{keplinger2024, ohring2002}.
    \item \textbf{Substrat}: Trägerschicht in der Mikroelektronik und Halbleiterfertigung. Der Begriff Substrat bezieht sich allgemein auf das 
    Material oder die Oberfläche, auf der ein Herstellungs- oder Beschichtungsprozess stattfindet \cite{ohring2002}.
    \item \textbf{Wafer}: Dünne Scheibe aus einem Halbleitermaterial, wie z.B. Silizium, und dient als Substrat \cite{sze2006}.
    \item \textbf{Tiegel} (Aufdampfen): Ein Behälter, in dem das Aufdampfmaterial lokal geschmolzen wird, häufig aus Materialien wie Graphit, 
    Keramik oder Wolfram, die hohen Temperaturen standhalten \cite{smith1995, ohring2002}. Beim Elektronenstrahlverdampfen (E-Beam-Verfahren) wird das 
    Aufdampfmaterial im Tiegel erhitzt und verdampft, ohne dass der gesamte Tiegel erhitzt werden muss \cite{smith1995}. Siehe
    \autoref{fig:Tiegel und Schiffchen} b), c) und d).
    \item \textbf{Schiffchen} (Aufdampfen): Ein kleiner, meist bootförmiger Behälter (daher der Name), der 
    ebenfalls zur Aufnahme des Aufdampfmaterials dient, jedoch direkt beheizt wird, häufig durch Widerstandsheizung. Schiffchen bestehen oft aus 
    Materialien wie Molybdän oder Wolfram, da sie hohe Temperaturen aushalten können \cite{mattox2010handbook, ohring2002}. Siehe \autoref{fig:PVD 
    Aufdampfanlage} und \autoref{fig:Tiegel und Schiffchen} a).
    \item \textbf{Target} (Sputtern): Materialquelle für die Schicht, die auf das Substrat abgeschieden werden soll \cite{keplinger2024}.
    \item \textbf{Inerte Gase}: Auch als \textbf{Edelgase} bekannt, sind chemisch reaktionsträge Gase, die unter normalen Bedingungen kaum mit 
    anderen Elementen oder Verbindungen reagieren. Zu den Edelgasen zählen Helium, Neon, Argon, Krypton, Xenon und Radon. Diese geringe Reaktivität ist auf 
    ihre stabile, vollständig besetzte Elektronenkonfiguration in der äußersten Schale zurückzuführen, die sie energetisch stabil und widerstandsfähig gegenüber 
    chemischen Reaktionen macht \cite{atkins_physical_chemistry, zumdahl_chemistry, silberberg_chemistry}.
    \item \textbf{Gasentladung}: Bezeichnet einen physikalischen Prozess, bei dem ein elektrisches Feld in einem gasförmigen Medium zur 
    Ionisation der Gasmoleküle führt, wodurch leitfähige Plasmen entstehen. Die dabei entstehenden freien Elektronen und Ionen ermöglichen den Stromfluss durch 
    das Gas, was zu sichtbaren Leuchterscheinungen führen kann. Solche Entladungen treten beispielsweise in Leuchtstofflampen, Neonröhren und in der 
    Plasmatechnologie auf \cite{prechtl2005grundlagen, boehler1994, hutchings2018}.
    \item \textbf{Ionen}: Elektrisch geladene Atome oder Moleküle, die durch den Verlust oder Gewinn von Elektronen entstehen. \textbf{Positiv geladene Ione}, 
    sogenannte \textbf{Kationen}, entstehen durch den Verlust von Elektronen, während \textbf{negativ geladene Ionen}, sogenannte \textbf{Anionen}, durch den 
    Gewinn von Elektronen entstehen. Ionen spielen eine zentrale Rolle in chemischen Reaktionen, da sie die elektrische Ladung transportieren und Reaktionen 
    ermöglichen, die die Bildung oder Trennung von chemischen Bindungen betreffen \cite{petrucci_general_chemistry, chang_chemistry, brown_chemistry_textbook}.
    \item \textbf{Radikale}: Radikale sind Atome oder Moleküle, die mindestens ein ungepaartes Valenzelektron besitzen, was sie extrem reaktiv macht 
    \cite{atkins2010, feynman2011}. Da sie oft instabil sind, neigen sie dazu, mit anderen Molekülen oder Atomen zu reagieren, um eine stabile 
    Elektronenkonfiguration zu erreichen\footnote{Im Kontext von Dünnschichtprozessen wie dem Sputtern werden Radikale oft durch das energiereiche Plasma gebildet, 
    das Bindungen in Molekülen aufbrechen kann \cite{boehler1994, hutchings2018}. Diese Radikale können dann auf der Oberfläche der zu beschichtenden Materialien 
    reagieren, wodurch die Schichtzusammensetzung und -struktur beeinflusst wird \cite{journal_reference}.}.
    \item \textbf{Valenzelektron}: Valenzelektronen befinden sich in der äußersten Schale eines Atoms und bestimmen die chemischen Eigenschaften eines Elements 
    maßgeblich. Sie ermöglichen die Bindung mit anderen Atomen, wodurch Verbindungen entstehen. Edelgase besitzen volle Valenzschalen, was sie besonders stabil und 
    reaktionsträge macht \cite{petrucci_general_chemistry, chang_chemistry, brown_chemistry_textbook}.
    \item \textbf{Fraktionale Verdampfung von Legierungen} (Aufdampfen): Tritt auf, wenn die unterschiedlichen Siedepunkte der Metallkomponenten dazu führen, dass 
    zunächst die Elemente mit niedrigerem Siedepunkt verdampfen, wodurch die Zusammensetzung des verbleibenden Materials verändert wird \cite{harvard_mrsec, 
    vem_guide}. Dadurch können die Dampfphasen eine andere Zusammensetzung als die ursprüngliche Legierung aufweisen, was zu inhomogenen Beschichtungen führen 
    kann \cite{frontiers_mpea, mdpi_vacuum_distillation}.
    \item \textbf{Dampfphasen}: Die gasförmigen Bestandteile, die entstehen, wenn feste oder flüssige Materialien durch Zuführung von Wärme in die Gasphase 
    übergehen, wie es bei der Verdampfung und Sublimation der Fall ist \cite{thermopedia, thoughtco, opentextbc}.
    \item \textbf{UHV}: ``Ultra-High Vacuum'' bezeichnet ein Vakuum mit einem Druck von unter \(10^{-9}\) mbar \cite{atkins_physical_chemistry, ohring2002, 
    smith1995}.
\end{itemize}

\vspace{1em}
\subsection{PVD-Verfahren} % PVD Verfahren
Die verschiedenen PVD-Verfahren \textbf{unterscheiden sich} hauptsächlich \textbf{in der} Methode zur \textbf{Erzeugung des Dampfstrahls} 
\cite{keplinger2024}:

\begin{enumerate}
    \item \textbf{Thermisches Verdampfen (Aufdampfen)}
    \item \textbf{Sputtern (Zerstäubung)}
    \item \textbf{Laserstrahlverdampfen (pulsed laser deposition, PLD)}
    \item \textbf{Molekularbeamepitaxie (MBE)}
\end{enumerate}

Im Folgenden werden lediglich die ersten beiden Methoden erläutert.

\vspace{1em}
\subsection{Aufdampfen} % PVD Aufdampfen
Das thermische Aufdampfen gehört zu den ältesten Methoden zur Herstellung dünner Schichten (\autoref{fig:PVD Aufdampfanlage}). Dabei wird das 
\textbf{Material der Aufdampfquelle auf} eine Temperatur von \textbf{500 bis 3000°C erhitzt}, bis es einen ausreichend hohen 
Dampfdruck\footnote{Ein hoher Dampfdruck ist beim Aufdampfen entscheidend, da er sicherstellt, dass genügend Atome oder Moleküle des Materials 
in die Gasphase übertreten und eine dampfartige Wolke bilden. Diese Wolke ermöglicht eine ausreichende Partikelzufuhr zum Zielsubstrat, wodurch 
sich eine gleichmäßige, dünne Schicht bildet. Bei zu geringem Dampfdruck wären nur wenige Teilchen verdampft, was zu einer ungleichmäßigen und 
ineffizienten Beschichtung führen würde.} entwickelt \cite{keplinger2024}.

\vspace{1em}

Meist schmilzt das Material dabei, doch einige Stoffe, wie beispielsweise Chrom, erreichen bereits vor dem Schmelzpunkt einen ausreichend hohen 
Dampfdruck und sublimieren, indem sie direkt aus dem festen Zustand verdampfen \cite{keplinger2024}.

\vspace{1em}

Der \textbf{entstehende Dampf} breitet sich aus und \textbf{kondensiert an} allen \textbf{kälteren Oberflächen} in der Vakuumkammer, 
\textbf{einschließlich des Substrats}. Der Prozess findet im \textbf{Vakuum bei Drücken} von typischerweise 
\textbf{unter $\mathbf{10^{-6}}$~mbar} statt \cite{keplinger2024}.

\vspace{1em}

Dabei ist die \textbf{freie Weglänge der Teilchen deutlich größer als} die \textbf{Distanz zwischen} der \textbf{Aufdampfquelle und} dem 
\textbf{Substrat}. Dies führt dazu, dass die \textbf{Teilchen nur selten mit} dem \textbf{Restgas kollidieren und sich geradlinig von der Quelle 
zum Substrat bewegen} \cite{keplinger2024}.

\vspace{1em}

Die \textbf{kinetische Energie der Teilchen, die auf} der \textbf{Substratoberfläche kondensieren}, beträgt etwa \textbf{0,1 bis 0,5~eV}, was im 
Vergleich zum Sputtern (bis zu 100~eV) \textbf{sehr gering} ist \cite{keplinger2024}.

\begin{figure}[htb!]
    \centering
    \includegraphics[width=\textwidth]{images/PVD Aufdampfanlage.png} % Pfad und Dateiname des Bildes angeben
    \captionsetup{labelfont=bf} % Setzt die Bildnummer fett
    \caption{Schematischer Aufbau einer PVD Aufdampfanlage \cite{keplinger2024}.}
    \label{fig:PVD Aufdampfanlage}
\end{figure}

\vspace{1em}
\subsubsection{Funktionsweise}
Basierend auf \autoref{fig:PVD Aufdampfanlage} wird im Folgenden die Funktionsweise dieses Verfahrens erläutert \cite{keplinger2024}:

\begin{enumerate}
    \item \textbf{Vakuum}: Im Rezipient wird mit dem Pumpsystem ein Vakuum erzeugt.
    \item \textbf{Aufheizen}: Das Aufdampfmaterial wird auf eine Temperatur von 500 bis 3000~°C erhitzt.
    \item \textbf{Öffnen des Shutters}: Sobald ein ausreichend hoher Dampfdruck erreicht ist, wird der Shutter weggeschwenkt, um die Bedampfung 
    des Substrats zu ermöglichen. Wird der Shutter zu früh geöffnet oder ist er während des Aufheizens nicht vorhanden, kann es zu einer 
    ungleichmäßigen Abscheidung kommen bzw. es können unerwünschte Oxide des Aufdampfmaterials auf das Substrat abgeschieden werden. Der Weg der
     Teilchen zum Substrat wird daher durch ein Schirmblech (Shutter) blockiert.
    \item \textbf{Abscheidung am Substrat}: Das Aufdampfmaterial kondensiert nun mit einer konstanten Rate sowohl auf dem Substrat als auch auf 
    dem Schwingquarz. Der Schwingquarz dient zur Messung der Schichtdicke, indem Änderungen seiner Resonanzfrequenz erfasst werden.
\end{enumerate}

\vspace{1em}
\subsubsection{Schichtdickenmessung}
Zur Messung der Schichtdicke wird ein \textbf{piezoelektrisches Schwingquarzplättchen} verwendet, das \textbf{nahe\footnote{Da Schwingquarz und 
Substrat unterschiedliche Abstände zur Aufdampfquelle haben und sich zudem unterschiedlich nah an der Achse des Aufdampfstrahls befinden, 
ergeben sich unterschiedliche Aufdampfraten. Diese Abweichungen werden durch einen experimentell zu bestimmenden Geometriefaktor
(Tooling-Factor) korrigiert.} am Substrat positioniert} ist. Durch Anlegen einer Wechselspannung wird das Quarzplättchen in eine 
(Scher-)Schwingung versetzt. \textbf{Wenn} das \textbf{Aufdampfmaterial auf} dem \textbf{Quarz kondensiert} und seine Masse zunimmt, 
\textbf{sinkt} die \textbf{Resonanzfrequenz} des Quarzes. Diese Abnahme der Frequenz ist \textbf{proportional zur Dicke der abgeschiedenen 
Schicht} und dient als Messgrundlage. Die Resonanzfrequenz startet typischerweise bei etwa 5~MHz und sollte, wenn sie auf 4~MHz fällt, zum 
Austausch des Quarzplättchens führen, um präzise Messwerte sicherzustellen \cite{keplinger2024}.

\vspace{1em}
\subsubsection{Erwärmung mittels Elektronenstrahlquelle (E-Beam-Verfahren)}

Bei dieser Methode wird ein \textbf{Elektronenstrahl gezielt auf} einen \textbf{kleinen Bereich des Aufdampfmaterials fokussiert}, sodass das 
Material \textbf{lokal aufgeschmolzen} wird, während der Großteil fest bleibt (\autoref{fig:Tiegel und Schiffchen} b), c) und d)). Der 
\textbf{Tiegel} selbst wird dabei \textbf{nicht auf hohe Temperaturen erhitzt}, was das \textbf{Risiko einer Kontamination durch} das 
\textbf{Tiegelmaterial verringert und höhere Arbeitstemperaturen} als bei herkömmlichen Quellen, wie Schiffchenquellen, ermöglicht 
(\autoref{fig:Tiegel und Schiffchen} a)). Um eine \textbf{effektive Wärmeableitung} sicherzustellen, besteht der \textbf{Tiegel aus Kupfer} und 
wird direkt \textbf{vom Kühlwasser durchströmt}. Das \textbf{Aufdampfmaterial} muss \textbf{leitfähig} sein, da sich andernfalls elektrische 
Ladungen ansammeln könnten, die den Elektronenstrahl ablenken könnten \cite{keplinger2024}.

\vspace{1em}

Ein \textbf{Permanentmagnet in der Nähe der Elektronenstrahlquelle lenkt} den \textbf{Strahl um 270° ab, sodass} die \textbf{Quelle} selbst 
\textbf{nicht beschichtet wird} (\autoref{fig:Tiegel und Schiffchen} b) und c)). Die \textbf{Steuerung des Systems} ermöglicht zudem ein 
\textbf{Fokussieren oder Defokussieren des Strahls}, eine \textbf{präzise Positionierung über Spulen} sowie eine \textbf{kontrollierte 
Leistungsanpassung} bis zu 5~kW. Durch ein leichtes Wobbeln des Strahls kann ein größerer Bereich des Aufdampfmaterials überstrichen werden, was
 zu einem gleichmäßigen Erhitzen führt \cite{keplinger2024}.

\begin{figure}[htb!]
    \centering
    \includegraphics[width=.9\textwidth]{images/Tiegel und Schiffchen.png} % Pfad und Dateiname des Bildes angeben
    \captionsetup{labelfont=bf} % Setzt die Bildnummer fett
    \caption{%
    a) Aufdampfschiffchen, mit Aluminiumoxid (milchige Bereiche) beschichtet, um das Benetzen durch das Aufdampfmaterial zu minimieren. 
    b) Schematischer Aufbau einer Elektronenstrahlquelle: S...Spule, A...Anode, K...Kathode, W...Wehneltzylinder, T...Tiegel mit 
    Aufdampfmaterial, L...Wasserkühlung, M...Permanentmagnet.  
    c) Elektronenstrahlquelle im Inneren der Vakuumkammer (Rezipient). 
    d) Vierlochtiegel für die Elektronenstrahlquelle, der Beschichtungen aus bis zu vier verschiedenen Materialien ermöglicht 
    \cite{keplinger2024}.
    }
    \label{fig:Tiegel und Schiffchen}
\end{figure}

\vspace{1em}
\subsection{Sputtern (Zerstäubung)} % Sputtern (Zerstäubung)

Der Sputterprozess ist ein \textbf{Abscheidungsverfahren}, bei dem \textbf{Atome aus dem Target}, der Materialquelle für die gewünschte Schicht, 
\textbf{durch energiereiche Ionen herausgeschlagen} und \textbf{auf} dem \textbf{Substrat abgeschiede}n werden. \autoref{fig:Sputteranlage}
zeigt den schematischen Aufbau einer Sputteranlage \cite{keplinger2024}.

\begin{figure}[htb!]
    \centering
    \includegraphics[width=.85\textwidth]{images/Sputtern.png} % Pfad und Dateiname des Bildes angeben
    \captionsetup{labelfont=bf} % Setzt die Bildnummer fett
    \caption{Schematischer Aufbau einer Sputteranlage \cite{keplinger2024}.}
    \label{fig:Sputteranlage}
\end{figure}

\vspace{1em}
\subsubsection{Funktionsweise} % Funktionsweise
Anhand der \autoref{fig:Sputteranlage} wird die Funktionsweise der Sputteranlage näher erläutert \cite{keplinger2024}:

\begin{enumerate}
    \item In einem \textbf{inerten Gas}, typischerweise \textbf{Argon} mit einem \textbf{Druck von 0,2 bis 10 Pa}, wird \textbf{durch} eine 
    \textbf{Gasentladung} ein \textbf{Plasma erzeugt}, das die \textbf{für} den \textbf{Sputterprozess benötigten Kationen ($\mathbf{Ar^+}$) 
    bereitstellt}.
    \item Durch die \textbf{Spannung zwischen Plasma und Sputtertarget} (Schichtmaterial) werden die \textbf{Ionen zum Target hin beschleunigt} 
    und \textbf{erreichen} dabei \textbf{Energien von einigen 10~eV}.
    \item \textbf{Trifft} ein \textbf{Ar-Ion} auf die \textbf{Targetoberfläche}, kommt es zu \textbf{elastischen und inelastischen Stößen mit} 
    den \textbf{Targetatomen}. \textbf{Inelastische Stöße erzeugen Sekundärionen} (Sekundärelektronen) \textbf{für} das \textbf{Plasma}, während 
    \textbf{elastische Stöße} eine \textbf{Stoßkaskade} auslösen, die \textbf{durch Impulsübertragung Atome aus} dem \textbf{Target 
    herausschlagen} kann (\autoref{fig:Sputtern Stosskaskade}). Eine Energie von \textbf{mindestens 10 bis 30~eV} ist dafür \textbf{notwendig}, 
    wobei die Sputtereffizienz mit steigender Energie bis zu einem Maximum von etwa 10~keV zunimmt. Danach sinkt die Effizienz, da die Ionen zu 
    tief eindringen und weniger Energie an die Oberfläche übertragen wird. Die \textbf{Effizienz dieser Stoßkaskaden hängt} zudem 
    \textbf{vom Massenverhältnis} zwischen \textbf{Sputterion und Targetatom ab}.
    \item Die \textbf{herausgeschlagenen Atome bewegen} sich \textbf{in Richtung des Substrats}. Auf ihrem Weg \textbf{stoßen} sie 
    \textbf{mehrfach mit} den \textbf{Atomen des Prozessgases zusammen}, wodurch sie einen Teil ihrer \textbf{Energie verlieren und} ihre 
    \textbf{Winkelverteilung breiter} wird.
    \item Die \textbf{Teilchen treffen auf} das \textbf{Substrat} (z.B. mit einer Energie von 10~eV) und \textbf{lagern sich dort ab}. 
    Aufgrund der breiten \textbf{Winkelverteilung} werden dadurch \textbf{nicht nur} die \textbf{Oberseite}, sondern \textbf{auch} die 
    \textbf{Seitenflächen der Strukturen auf} dem \textbf{Substrat beschichtet} (daher auch der Name ``Sputtern'' -- \textit{to sputter} = 
    zerstäuben).
\end{enumerate}

\begin{figure}[htb!]
    \centering
    \includegraphics[width=.55\textwidth]{images/Sputtern Stoßkaskade.png} % Pfad und Dateiname des Bildes angeben
    \captionsetup{labelfont=bf} % Setzt die Bildnummer fett
    \caption{Herauslösen eines Targetatoms durch die vom Sputterion ausgelöste Stoßkaskade \cite{keplinger2024}.}
    \label{fig:Sputtern Stosskaskade}
\end{figure}

\vspace{1em}
\subsubsection{Sputterverfahren} % Sputterverfahren
Im Folgenden werden ausgewählte Sputterverfahren näher erläutert und deren Funktionsweise beschrieben.

\vspace{1em}
\paragraph{\uline{DC-Sputtern:}} Hier wird die \textbf{Gasentladung mittels} einer \textbf{Gleichspannung} erzeugt. Daher muss das \textbf{Target 
leitfähig} sein, da es sich sonst auflädt und die Gasentladung unterbrochen würde. Diese Form des Sputterns \textbf{eignet sich} somit nur 
\textbf{für Metalle und Halbleiter} \cite{keplinger2024}.

\vspace{1em}
\paragraph{\uline{AC-Sputtern:}} Ein \textbf{hochfrequentes elektrisches Feld} wird \textbf{über} einen \textbf{Kondensator an} die \textbf{Elektroden 
angelegt}, wodurch ein \textbf{Gleichspannungsanteil blockiert} wird. \textbf{Ionen können} dem \textbf{Feldwechsel ab etwa 100~kHz nicht mehr 
folgen}, während \textbf{Elektronen weiterhin Energie aufnehmen} und \textbf{mehr Gasteilchen ionisieren}. Dadurch entsteht eine \textbf{höhere 
Plasmarate als beim DC-Sputtern}, und der \textbf{Druck kann bei gleicher Sputterrate auf etwa $\mathbf{10^{-1}}$ bis $\mathbf{10^{-2}}$~Pa 
gesenkt werden}\footnote{Da die herausgeschlagenen Targetatome aufgrund der geringeren Anzahl an Kollisionen mit den Prozessgasatomen weniger 
gestreut werden, erreichen sie das Substrat direkter. Dadurch werden die Seitenflächen der Strukturen auf dem Substrat weniger beschichtet als 
beim DC-Sputtern.} \cite{keplinger2024}. \\

Die \textbf{Vorteile} umfassen \cite{keplinger2024}:

\begin{itemize}
    \item \textbf{Auch Isolatoren} wie \( \mathrm{Al}_2\mathrm{O}_3 \) (Aluminiumoxid) und Halbleiter \textbf{können gesputtert (zerstäubt) werden}.
    \item Das \textbf{Substrat erhitzt sich weniger} als beim DC-Sputtern.
    \item \textbf{Sputterraten} sind \textbf{bis zu zehnmal höher} als beim DC-Sputtern.
\end{itemize}

Ein \textbf{Nachteil} ist der \textbf{höhere apparative Aufwand} \cite{keplinger2024}.

\vspace{1em}
\paragraph{\uline{Magnetronsputtern:}} Beim Magnetronsputtern sind \textbf{Permanentmagneten hinter} der \textbf{Kathode} angebracht, die ein Magnetfeld erzeugen 
(\autoref{fig:Magnetronsputtern}). Dieses bewirkt, dass sich die \textbf{Elektronen in Zykloidenbahnen bewegen} (\autoref{fig:Zykloiden}) und \textbf{länger in der 
Nähe des Targets bleiben}\footnote{Dies ist effektiv, wenn die Magnetfeldlinien parallel zur Oberfläche verlaufen. Liegen sie jedoch parallel zur 
Elektronenflugbahn (senkrecht zur Targetoberfläche), wirkt keine ablenkende Kraft auf die Elektronen.}, wodurch \textbf{mehr Gasatome ionisiert} werden und die 
\textbf{Plasmadichte steigt}. Dies \textbf{erhöht} die \textbf{Sputterrate und ermöglicht} es, den \textbf{Druck bei gleicher Sputterrate deutlich zu senken}, was 
\textbf{dichtere Schichten erzeugt}. Aufgrund der \textbf{kürzeren Beschichtungszeit} wird das Magnetronsputtern \textbf{in industriellen Anlagen häufig 
verwendet} \cite{kittel2004}. \\

\begin{figure}[htb!]
    \centering
    \includegraphics[width=.75\textwidth]{images/Magnetronsputteranlage.png} % Pfad und Dateiname des Bildes angeben
    \captionsetup{labelfont=bf} % Setzt die Bildnummer fett
    \caption{Schematischer Aufbau einer Magnetron-Sputteranlage \cite{keplinger2024}.}
    \label{fig:Magnetronsputtern}
\end{figure}

\begin{figure}[htb!]
    \centering
    \includegraphics[width=.75\textwidth]{images/Zykloiden.png} % Pfad und Dateiname des Bildes angeben
    \captionsetup{labelfont=bf} % Setzt die Bildnummer fett
    \caption{Zykloiden}
    \label{fig:Zykloiden}
\end{figure}

Die Grundlage für diese Überlegung ist die Lorentzkraft, welche die Kraft beschreibt, die auf eine bewegte elektrische Ladung in einem elektrischen und 
magnetischen Feld wirkt. Diese Kraft wird durch
$$
\vec{F} = e (\vec{E} + \vec{v} \times \vec{B})
$$
beschrieben. Dabei stellt \( \vec{F} \) die Lorentzkraft in Newton dar, \( e \) ist die Elementarladung mit einem Wert von etwa 
\( 1{,}602 \cdot 10^{-19} \, \mathrm{C} \), \( \vec{E} \) das elektrische Feld in Volt pro Meter (V/m), \( \vec{v} \) die Geschwindigkeit der
Ladung in Metern pro Sekunde (m/s), und \( \vec{B} \) die magnetische Flussdichte in Tesla (T). Das Kreuzprodukt \( \vec{v} \times \vec{B} \) 
zeigt, dass die Kraft \( \vec{F} \) senkrecht sowohl zur Bewegungsrichtung \( \vec{v} \) als auch zum Magnetfeld \( \vec{B} \) steht \cite{rossnagel2003magnetron, 
mattox2010handbook, anders2010advances, kouznetsov1999new, prechtl2005grundlagen}. 

\vspace{1em}
\paragraph{\uline{Reaktives Sputtern:}} Obwohl \textbf{dem Sputterprozess mehr Materialien zur Verfügung stehen als dem Aufdampfen}, gibt es \textbf{dennoch 
Stoffe, bei denen sich eine Komponente ungenügend niederschlägt}. Durch die \textbf{Zugabe eines Reaktionsgases zum Plasma} können jedoch \textbf{chemische 
Reaktionen mit den Targetatomen in der Gasphase} eingeleitet werden, \textbf{sodass sich} die \textbf{gewünschten Verbindungen abscheiden}. Ein \textbf{Beispiel} 
ist die Bildung von \textbf{Aluminiumoxid und -nitrid}:
$$
\mathrm{2Al} + \mathrm{3O}_2 \rightarrow \mathrm{Al}_2\mathrm{O}_3
$$
$$
\mathrm{2Al} + \mathrm{N}_2 \rightarrow \mathrm{2AlN}
$$
Dieses Verfahren, bekannt als reaktives Sputtern, wird oft \textbf{zur Herstellung von Oxiden, Nitriden, Oxinitriden} (z.B., \(\mathrm{SiO}_x\mathrm{N}_y\)) 
\textbf{und Carbiden} verwendet. Es lässt sich zudem \textbf{mit anderen Sputterverfahren}, wie dem Magnetronsputtern, \textbf{kombinieren}. Ein \textbf{Vorteil} 
des reaktiven Sputterns besteht darin, dass die \textbf{Stöchiometrie\footnote{Stöchiometrie bezieht sich auf das Verhältnis der Atome oder Moleküle in einer 
chemischen Verbindung. In einem Material wie Aluminiumoxid (\(\mathrm{Al}_2\mathrm{O}_3\)) beschreibt die Stöchiometrie das Verhältnis von Aluminium- zu 
Sauerstoffatomen, das hier exakt 2:3 ist. Im Kontext des reaktiven Sputterns bedeutet das, dass die Stöchiometrie der abgeschiedenen Schicht -- also das 
Verhältnis der Elemente im abgeschiedenen Material -- durch den Druck des zugeführten Reaktionsgases beeinflusst werden kann. Das bedeutet, dass durch die Menge 
des Reaktionsgases im Plasma das Verhältnis der chemischen Bestandteile in der abgelagerten Schicht gezielt gesteuert wird. Zum Beispiel könnte eine Erhöhung des 
Sauerstoffdrucks im Plasma die Menge an Sauerstoff in der Schicht erhöhen, was zur Bildung einer Schicht mit einem höheren Anteil an Sauerstoff führt 
\cite{anders2010reactive, ohring2002, mattox2010handbook, rossnagel2003magnetron}.} der abgeschiedenen Schicht durch} den \textbf{Gasdruck des Reaktionsgases 
gezielt beeinflusst} werden kann \cite{keplinger2024}.

\vspace{1em}
\subsection{Vergleich Aufdampfen - Sputtern}
Abschließend werden in \autoref{tab:Vergleich Aufdampfen - Sputtern} das thermische Aufdampfen und das Sputtern gegenübergestellt.

\begin{table}[htb!]
    \centering
    \renewcommand{\arraystretch}{1.5} % Erhöht den Zeilenabstand in der Tabelle
    \begin{tabular}{|p{6cm}|p{6cm}|}
        \hline
        \multicolumn{2}{|c|}{\textbf{1. Dampferzeugung}} \\ \hline
        \textbf{Aufdampfen} & \textbf{Sputtern} \\ \hline
        Thermischer Prozess & Ionenbeschuss und Impulsübertragung \\
        Geringe kinetische Energie (1500°C: $E \approx 0,1 \, \mathrm{eV}$) & Hohe kinetische Energie ($E = 10 - 40 \, \mathrm{eV}$) \\
        Neutrale Teilchen & Neutrale Teilchen, Kationen, Radikale, (Anionen) \\
        Fraktionierte Verdampfung bei Legierungen & Relativ gleichmäßiges Verdampfen aller Komponenten \\
        Verbindungen dissoziieren & Geringe Dissoziation \\ \hline
        \multicolumn{2}{|c|}{\textbf{2. Transport}} \\ \hline
        \textbf{Aufdampfen} & \textbf{Sputtern} \\ \hline
        Teilchen im Hochvakuum & Teilchen im Plasma \\
        Keine Kollisionen & Viele Kollisionen \\
        Große freie Weglänge & Kleine freie Weglänge \\
        Teilchen bewegen sich geradlinig von der Quelle zum Substrat & Breite Richtungsverteilung der beim Substrat ankommenden Teilchen \\ \hline
        \multicolumn{2}{|c|}{\textbf{3. Kondensation}} \\ \hline
        \textbf{Aufdampfen} & \textbf{Sputtern} \\ \hline
        Wenig Restgasatome in den Schichten & Starker Einbau von Atomen des Plasmas \\
        Kaum Beeinflussung des Substrats & Starke Beeinflussung durch Ionen \\
        Seitenflächen werden nicht beschichtet, ebene Schichten & Alle Flächen werden beschichtet \\
        Mittelmäßige Haftung der Schichten & Gute Haftung \\
        Hohe Beschichtungsraten & Geringe Beschichtungsraten \\
        Lange Prozesszeit durch UHV & Geringe Anforderung an die Vakuumqualität \\
        Strukturieren durch Ätzen oder Lift-off & Strukturieren hauptsächlich durch Ätzen, fallweise durch Lift-off \\ \hline
    \end{tabular}
    \captionsetup{labelfont=bf} % Setzt die Bildnummer fett
    \caption{Vergleich Audampfen - Sputtern \cite{keplinger2024}}
    \label{tab:Vergleich Aufdampfen - Sputtern}
\end{table}

\vspace{1em}

``Verbindungen dissoziieren'' bedeutet, dass chemische Bindungen im Material beim Aufdampfen thermisch zerfallen\footnote{Thermischer Zerfall beschreibt den Prozess, 
bei dem chemische Verbindungen bei hohen Temperaturen in ihre Bestandteile zerfallen, da die Bindungsenergie durch zugeführte Wärme überwunden wird. Dieser Zerfall 
kann beim Aufdampfen auftreten und die Zusammensetzung der Schicht beeinflussen \cite{ohring2002, smith1995}.} können, was die Zusammensetzung verändert. Im 
Gegensatz dazu tritt beim Sputtern nur eine geringe Dissoziation auf, da der Ionenbeschuss die chemische Stabilität der Verbindungen weitgehend erhält 
\cite{ohring2002, smith1995, rossnagel2003magnetron}.

\vspace{1em}
``Strukturieren durch Ätzen oder Lift-off'' sind Verfahren zur Formgebung von Dünnschichtstrukturen. Beim Ätzen wird überschüssiges Material selektiv durch 
chemische oder physikalische Methoden entfernt, wodurch nur die gewünschte Struktur verbleibt. Das Lift-off-Verfahren hingegen nutzt eine Maske, die nach der 
Beschichtung entfernt wird, sodass das Material nur an den zuvor maskierten Stellen verbleibt \cite{ohring2002, smith1995, prechtl2005grundlagen}. Mehr dazu folgt 
in den späteren Kapiteln.

\vspace{1em}
\section{Chemical Vapour Deposition} % CVD
PVD-Verfahren sind essenzielle Prozesse in der Herstellung miniaturisierter Sensoren, vor allem für metallische Schichten und Dielektrika. Diese Verfahren 
erfordern feste Ausgangsstoffe, sind jedoch problematisch bei Materialien, die im Prozess disproportionieren\footnote{Disproportionierung ist ein Prozess, bei dem 
ein Stoff sich aufteilt: Ein Teil gibt Elektronen ab (Oxidation) und ein anderer Teil nimmt Elektronen auf (Reduktion). Dadurch entstehen zwei neue Stoffe mit dem 
gleichen Element, aber in verschiedenen Zuständen \cite{petrucci_general_chemistry}. In PVD-Prozessen kann dies problematisch sein, weil beim Erhitzen zum Beispiel 
Metallteile aufbrechen und nicht als ganzes Metall in die Schicht eingebaut werden. Stattdessen bilden sich zwei verschiedene Verbindungen, die nicht alle 
Eigenschaften des Ausgangsstoffes besitzen, was die Qualität der Schicht beeinflussen kann.}, etwa durch Schmelzen beim Aufdampfen. Dadurch können wichtige 
Komponenten unvollständig in die Schicht eingebaut werden. Im Gegensatz dazu werden \textbf{bei} der \textbf{Chemical Vapor Deposition (CVD) Gase in den 
Rezipienten einer Anlage geleitet}, wo sie \textbf{chemisch reagieren und} die \textbf{Reaktionsprodukte am Substrat}, oft ein Wafer, eine \textbf{Schicht bilden}. 
Diese chemische Reaktion \textbf{erfordert Energie}, die \textbf{thermisch oder durch Gasentladung} zugeführt wird \cite{keplinger2024CVD}.

\vspace{1em}

Die \textbf{Unterscheidung} zwischen \textbf{CVD und PVD} ergibt sich durch die \textbf{Form der zugeführten Ausgangsstoffe (Precursor)}: Bei 
\textbf{PVD-Prozessen} werden diese \textbf{in fester Form} und bei \textbf{CVD-Prozessen} als \textbf{Gase} zugeführt. Die thermische Oxidation eines 
Silizium-Wafers wird nicht als CVD-Prozess betrachtet, da nur der Sauerstoff gasförmig ist, während das Silizium bereits fest im Reaktor vorhanden ist
 \cite{keplinger2024CVD}.

 \vspace{1em}
\paragraph{Vorteile:}

\begin{itemize}
    \item \textbf{Hohe Reinheit} der Schichten, da die Ausgangsgase in hoher Reinheit verfügbar sind.
    \item \textbf{Gute Kantenbedeckung}, auch bei ungünstiger Geometrie.
    \item \textbf{Herstellung} vieler \textbf{verschiedener Schichtmaterialien} möglich.
    \item \textbf{Zusammensetzung kann} während des Prozesses \textbf{angepasst werden}.
    \item \textbf{Gute Prozesskontrolle} durch eine \textbf{Vielzahl leicht beeinflussbarer Parameter}: Gase, Flussraten, Temperatur, Druck, Frequenz und Leistung 
    des Plasmas.
\end{itemize}

\vspace{1em}
\paragraph{Nachteile:}

\begin{itemize}
    \item Relativ \textbf{hohe Temperaturen} sind oft \textbf{erforderlich}, wodurch \textbf{mechanische Spannungen} in die Schicht eingebaut werden. Dies führt 
    zu einer \textbf{hohen thermischen Belastung des Wafers} und der darauf befindlichen Strukturen, was Dotierprofile ändern oder Metallschichten zerstören kann.
    \item \textbf{Schwierige Handhabung} der häufig \textbf{giftigen und brennbaren Gase}.
    \item \textbf{Für viele gewünschte Schichten} sind \textbf{keine gasförmigen Precursoren verfügbar}.
    \item Die \textbf{Vielzahl relevanter Parameter erfordert hohe Sachkenntnis} des Betreibers der Anlage, insbesondere bei Nichtstandard-Prozessen.
\end{itemize}

\vspace{1em}
\subsection{Begriffsklärung für CVD} % CVD  Begriffsklärung
Dieser Abschnitt bietet eine Einführung in wichtige, häufig verwendete Begriffe. Es empfiehlt sich, zuerst die nahcfolgenden Abschnitte zu lesen und hier bei Bedarf 
unbekannte Begriffe nachzuschauen. \textbf{Begriffe}, die bereits \textbf{in den vorherigen Abschnitten erklärt} wurden, sind \textbf{hier nicht erneut aufgeführt}.

\begin{itemize}
    \item \textbf{Precursor}: Ausgangsstoffe (Gase) \cite{keplinger2024CVD}. 
    \item \textbf{Reaktanden}: Stoffe, die bei einer chemischen Reaktion miteinander reagieren, um neue Produkte zu bilden. Man könnte sie als die ``Zutaten'' 
    einer chemischen Reaktion betrachten \cite{petrucci_general_chemistry, chang_chemistry}.
    \item \textbf{Diffusionsprozess}: Ein Diffusionsprozess beschreibt die Bewegung von Teilchen von einem Bereich hoher Konzentration zu einem Bereich niedriger 
    Konzentration, bis die Teilchen gleichmäßig verteilt sind \footnote{Man kann es sich vorstellen wie Tinte, die sich langsam in einem Glas Wasser verteilt, ohne 
    dass man umrühren muss.} \cite{bergman2011, incropera2007}. In der Wissenschaft beschreibt ``diffundieren'' den Prozess, bei dem Teilchen sich aufgrund eines 
    Konzentrationsgefälles bewegen \cite{crank1975}.
    \item \textbf{Erzwungene Konvenktion}: Erzwungene Konvektion bedeutet, dass der Transport von Stoffen, wie Gasen, durch eine äußere Kraft unterstützt wird, 
    zum Beispiel durch einen Ventilator oder einen Gasstrom. Dadurch gelangen die Reaktanten gezielt an ihren Bestimmungsort \cite{incropera2007, bergman2011, 
    kays2005, bejan2013}.
    \item{Konformität}: Verhältnis der Beschichtungsdicke auf den vertikalen Flächen zu der auf den horizontalen Flächen der Strukturen am Substrat.
\end{itemize}

\vspace{1em}
\subsection{Funktionsweise} % CVD Funktionsweise
Die Schichtbildung basiert auf einer \textbf{Reihe von Teilprozessen}, die je nach Reaktionsgasen komplex ablaufen können (\autoref{fig:Schichtbildung CVD}). 
Hier werden \textbf{beispielhaft} die Teilprozesse zur \textbf{Bildung von Siliziumnitrid beschrieben}, das häufig als \textbf{Isolationsschicht} dient 
\cite{keplinger2024CVD}.

\begin{figure}[htb!]
    \centering
    \includegraphics[width=.65\textwidth]{images/Schichtbildung CVD.png} % Pfad und Dateiname des Bildes angeben
    \captionsetup{labelfont=bf} % Setzt die Bildnummer fett
    \caption{Prozesse der Schichtbildung bei der CVD-Methode: 1...Gaszufuhr von SiH$_4$ und NH$_3$, 2...Diffusion der Reaktanten durch die Grenzschicht, 
    3...Adsorption der Reaktanten an der Waferoberfläche, 4...chemische Reaktion, 5...Oberflächendiffusion der Reaktionsprodukte, 6...Desorption der Nebenprodukte 
    und 7...Abtransport durch den Gasstrom \cite{keplinger2024CVD}.}
    \label{fig:Schichtbildung CVD}
\end{figure}

\begin{enumerate}
    \item \textbf{Gaszufuhr}: Die Reaktanden SiH$_4$ (Silan) und NH$_3$ (Ammoniak) werden in die Reaktionskammer geleitet und strömen über das Substrat. Dabei 
    gelangen die Ausgangsstoffe durch erzwungene Konvektion zum Substrat \cite{keplinger2024CVD}.
    \item \textbf{Diffusion}: Die Geschwindigkeit des Gases nimmt zur Oberfläche hin innerhalb der Prandtlschen Grenzschicht\footnote{Die Prandtlsche Grenzschicht 
    ist der Bereich in der Nähe einer Oberfläche, in dem die Strömungsgeschwindigkeit eines Fluids von der maximalen Geschwindigkeit bis auf Null abnimmt. 
    Innerhalb dieser Schicht verlangsamt sich das Fluid aufgrund der Reibungskräfte an der Grenzfläche \cite{white2006}.} bis auf Null ab. Deshalb müssen die 
    Reaktanden diese Schicht durch einen Diffusionsprozess überwinden\footnote{Die Reaktanden diffundieren durch die Prandtlsche Grenzschicht zur Oberfläche 
    \cite{bergman2011, incropera2007}.} \cite{keplinger2024CVD}.
    \item \textbf{Adsorption}: Die Moleküle, die zur Oberfläche hin diffundieren, ``kleben'' an der Oberfläche fest (adsorbieren). Dort bleiben sie aber nicht an 
    einer festen Stelle, sondern bewegen sich je nach Temperatur mehr oder weniger stark über die Oberfläche \cite{keplinger2024CVD, bergman2011, incropera2007}.
    \item \textbf{Reaktion}: Die Ausgangsstoffe reagieren an der Oberfläche. Bei der Bildung von Siliziumnitrid folgt die Reaktion der chemischen Gleichung:
    $$
    3\mathrm{SiH}_4 + 4\mathrm{NH}_3 \rightarrow \mathrm{Si}_3\mathrm{N}_4 + 12\mathrm{H}_2
    $$
    Ein Teil der Reaktion kann jedoch auch schon in der Gasphase ablaufen. Das ist meist unerwünscht, weil dabei feste Partikel entstehen können, die sich dann als 
    Staub ablagern. Um sicherzustellen, dass die Reaktion vor allem an der Oberfläche stattfindet, hält man den Prozessdruck typischerweise zwischen 0,01 und 
    10~mbar oder verdünnt die Gase mit einem Trägergas\footnote{Durch das Reduzieren des Drucks oder die Zugabe eines Trägergases wird die Dichte der 
    Ausgangsstoffe verringert, sodass die Moleküle seltener aufeinanderstoßen. Dadurch haben die Reaktanten eine höhere Chance, die Oberfläche zu erreichen und
     dort zu reagieren, statt bereits in der Gasphase zu reagieren.} \cite{keplinger2024CVD}.
    \item \textbf{Oberflächendiffusion}: Durch diesen Vorgang wird das Reaktionsprodukt Si$_3$N$_4$ (Siliziumnitrid) in die Schicht eingebaut\footnote{Damit ist 
    gemeint, dass das Reaktionsprodukt in die Oberfläche diffundiert. Es wandert also in die Schicht hinein und verteilt sich.} und möglichst bis an die 
    Schichtkante transportiert \cite{keplinger2024CVD}.
    \item \textbf{Desorption}: Bei der chemischen Reaktion entstehen nicht nur die gewünschten Schichtbestandteile, sondern auch Nebenprodukte (wie hier H$_2$), 
    die an der Oberfläche ``kleben'' (adsorbiert sind) bleiben. Diese müssen sich wieder lösen (desorbieren) und dann durch die Grenzschicht ``hindurchwandern'' 
    (diffundieren), um entfernt zu werden \cite{keplinger2024CVD}.
    \item \textbf{Abtransport}: Im letzten Schritt werden die Nebenprodukte vom konvektiven Gasfluss mitgenommen und abtransportiert \cite{keplinger2024CVD}.
\end{enumerate}

Je nach \textbf{Beschichtungsparametern}, insbesondere der \textbf{Temperatur}, können auch \textbf{Nebenprodukte in die Schicht eingebaut} werden, was die 
\textbf{Schichtqualität erheblich beeinflusst}. \textbf{Siliziumnitride}, die \textbf{bei niedrigen Temperaturen abgeschieden} werden, e\textbf{nthalten mehr 
Wasserstoff} und sind daher \textbf{weniger ätzresistent} als Hochtemperaturnitride. Auch die \textbf{Oberflächendiffusion hängt stark von} der \textbf{Temperatur 
ab}: \textbf{Niedrigere Temperaturen} führen zu geringerer Beweglichkeit auf der Oberfläche und somit zu \textbf{poröseren Schichten} \cite{keplinger2024CVD}.

\vspace{1em}
\subsection{CVD-Anlagen} % CVD-Anlagen
Die verschiedenen CVD-Anlagen können nach unterschiedlichen Kriterien wie Betriebs- oder Konstruktionsparametern klassifiziert werden. Es gibt zahlreiche mögliche 
Kombinationen, wobei jedoch nicht alle theoretischen Optionen vollständig genutzt werden. Einige dieser Kriterien werden im Folgenden genauer besprochen 
\cite{keplinger2024CVD}:

\begin{itemize}
    \item \textbf{Verbrauch an Reaktanden}
    \item \textbf{Reaktionsort}
    \item \textbf{Energiezufuhr}
    \item \textbf{Erwärmte Anlagenteile}
    \item \textbf{Druck}
\end{itemize}

\vspace{1em}
\subsubsection{Verbrauch an Reaktanden} % Verbruach an Reaktanden
Ein wichtiges Kriterium ist der Verbrauch der Reaktanden, der darüber entscheidet, \textbf{welcher CVD-Reaktor für welchen Anwendungsfall} geeignet ist. Man 
unterschiedet zwischen Differential- und Integralreaktoren \cite{keplinger2024CVD}.

\vspace{0.0em}
\paragraph{\uline{Differnetialreaktor:}} \textbf{Fast die gesamte Menge der Ausgangsstoffe, die in die Reaktionskammer eintritt, verlässt diese auch wieder}. 
Lediglich eine \textbf{vergleichsweise kleine (differentielle) Menge wird tatsächlich verbraucht} und abgeschieden. Aus Sicht des Materialeinsatzes ist 
\textbf{dieser Reaktortyp} zwar \textbf{unökonomisch}, jedoch bleiben die \textbf{Konzentrationen im Reaktor konstant}, was \textbf{gleichmäßige Beschichtungen} 
ermöglicht \cite{keplinger2024CVD}.

\vspace{0.0em}
\paragraph{\uline{Integralreaktor:}} \textbf{Entlang des Gasstroms} entsteht ein \textbf{deutlicher Konzentrationsgradient}, wobei \textbf{stromabwärts} ein 
\textbf{Mangel an Reaktanden} auftritt (\textit{starved reactor}). Diese Reaktoren sind aufgrund der \textbf{hohen Abscheideraten gut für} die \textbf{Produktion} 
geeignet. \textbf{Bei hohen Temperaturen und entsprechend hoher Oberflächenbeweglichkeit} können \textbf{auch hier homogene Schichten} erzielt werden.

\vspace{1em}
\subsubsection{Reaktionsort} % Reaktionsort
Bei diesem Kriterium wird unterschieden, \textbf{an welchem Ort in der CVD-Anlage die Reaktanden miteinander reagieren} \cite{keplinger2024CVD}.

\vspace{0.0em}
\paragraph{\uline{Homogene CVD:}} Die \textbf{Reaktanden reagieren bereits in der Gasphase}, wobei das \textbf{Reaktionsprodukt durch} die 
\textbf{Grenzschicht zum Substrat diffundiert} und sich \textbf{dort abschiedet}. Solche CVD-Prozesse führen häufig zu \textbf{schlecht haftenden Schichten} mit 
deutlich \textbf{mehr Defekten} \cite{keplinger2024CVD}.

\vspace{0.0em}
\paragraph{\uline{Heterogene CVD:}} Die Schicht bildet sich durch die \textbf{chemische Reaktion direkt an der Substratoberfläche}. Aufgrund der 
\textbf{besseren Schichteigenschaften} wird diese Variante \textbf{in der Mikrosystemtechnik bevorzugt} \cite{keplinger2024CVD}.

\vspace{1em}
\subsubsection{Energiezufuhr} % Energiezufuhr
Ein weiteres wesentliches Kriterium ist die Energiezufuhr, die bestimmt, \textbf{wie die für die Reaktion benötigte Energie bereitgestellt 
wird} \cite{keplinger2024CVD}.

\vspace{0.0em}
\paragraph{\uline{Thermische Anregung:}} Das \textbf{Substrat} wird \textbf{mithilfe von Öfen oder Heizplatten auf} die erforderliche \textbf{Reaktionstemperatur 
gebracht} \cite{keplinger2024CVD}.

\vspace{0.0em}
\paragraph{\uline{Plasma:}} Durch eine \textbf{Gasentladung} entsteht ein Plasma, das den \textbf{Großteil der Energie für die chemische Reaktion} bereitstellt, 
weshalb man von \textbf{PECVD (\textit{plasma enhanced} CVD)} spricht. Die \textbf{thermische Belastung der Oberfläche und des Sunstrats} bleibt dabei 
\textbf{gering}\footnote{In der Physik bezeichnet man ein Plasma als ein teilweise oder vollständig ionisiertes Gas. Plasmen werden in ``heiß'' und ``kalt'' 
unterteilt, je nachdem, ob thermisches Gleichgewicht zwischen den Teilchen besteht. In einem heißen Plasma haben Elektronen und Ionen die gleiche Temperatur 
(typischerweise $kT > 5000$ K), die notwendig ist, um Elektronen aus Atomen herauszuschlagen. In einem kalten Plasma hingegen ist nur die Temperatur der Elektronen 
hoch, während die Ionen und neutralen Gasteilchen deutlich kühler bleiben. Das liegt daran, dass Elektronen etwa 100.000-mal leichter als Ionen sind und somit kaum 
Energie mit diesen austauschen können, ähnlich wie ein Tischtennisball, der eine Billardkugel trifft. Aufgrund des geringen Drucks in Plasmaanlagen ist die freie 
Weglänge groß, was zu wenigen Kollisionen führt, bevor die Teilchen die Anlage verlassen. Bei der Anregung des Plasmas durch eine Gasentladung werden die 
Elektronen erhitzt, ohne das Gas insgesamt zu erwärmen. Die hochenergetischen Elektronen ionisieren oder dissoziieren das Restgas und ermöglichen chemische 
Reaktionen, die normalerweise bei niedrigen Temperaturen nicht stattfinden würden, wodurch die thermische Belastung des Substrats und der Oberflächen gering 
bleibt \cite{keplinger2024CVD}.} \cite{keplinger2024CVD}.

\vspace{0.0em}

Das \textbf{Plasma kann kapazitiv} durch parallel angeordnete Elektroden (Parallelplattenanlage) -- zu sehen in \autoref{fig:PECVD} -- \textbf{oder induktiv} 
durch eine Spule um die Reaktorkammer (ICP-Anlage, \textit{inductive coupled plasma}) \textbf{angeregt werden}. Dabei wird häufig die industriefreigegebene 
Frequenz von 13,56~MHz verwendet, und die Anlagen sind aufgrund ihrer \textbf{starken elektromagnetischen Strahlung} abgeschirmt \cite{keplinger2024CVD}.

\begin{figure}[htb!]
    \centering
    \includegraphics[width=\textwidth]{images/PECVD-Anlage.png} % Pfad und Dateiname des Bildes angeben
    \captionsetup{labelfont=bf} % Setzt die Bildnummer fett
    \caption{Schematischer Aufbau einer PECVD-Anlage, bei der das Plasma durch eine Gasentladung erzeugt wird (Parallelplattenanlage). Die 
    Elektronentemperatur beträgt $T_\mathrm{el} = 10\,000{-}90\,000 \, \mathrm{K}$ (1–8 eV), die Ionentemperatur $T_\mathrm{ion} = 500{-}1\,000 \, \mathrm{K}$, 
    und der Druck liegt bei $p = 0.1{-}10 \, \mathrm{mbar}$ \cite{keplinger2024CVD}.}
    \label{fig:PECVD}
\end{figure}

\vspace{0.0em}
\paragraph{\uline{Laser:}} Durch die \textbf{gezielte lokale Erwärmung des Substrats mittels Laser (LCVD)} können \textbf{Schichten schreibend auf} das 
\textbf{Substrat aufgetragen} und sogar \textbf{dreidimensionale Strukturen} erzeugt werden (\autoref{fig:LCVD}). Wie bei den meisten schreibenden Verfahren eignet 
sich der LCVD-Prozess \textbf{gut für Prototypen und Forschungszwecke}, ist jedoch \textbf{weniger geeignet für industrielle Anwendungen} \cite{keplinger2024CVD}.

\begin{figure}[htb!]
    \centering
    \includegraphics[width=.55\textwidth]{images/Laser CVD.png} % Pfad und Dateiname des Bildes angeben
    \captionsetup{labelfont=bf} % Setzt die Bildnummer fett
    \caption{Laser-unterstützte CVD: Der Laser erwärmt das Substrat punktuell, wodurch die chemische Reaktion und Schichtabscheidung nur im Bereich des Laserflecks 
    stattfinden \cite{keplinger2024CVD}.}
    \label{fig:LCVD}
\end{figure}

\vspace{1em}
\subsubsection{Erwärmte Anlagenteile} % Erwärmte Anlagenteile
Bei einer CVD-Anlage ist ebenfalls zu berücksichtigen, welche \textbf{Teile des Reaktors} erwärmt werden \cite{keplinger2024CVD}.

\vspace{0.0em}
\paragraph{\uline{Hot-Wall-Reaktor:}} Der \textbf{Substrathalter (Susceptor) mit} den \textbf{Substraten und die Reaktorkammer werden durch} 
einen \textbf{Ofen mit Widerstandsheizung erwärmt} (\autoref{fig:Hot-Wall-Reaktor}). Die \textbf{Wafer stehen frei} und können 
\textbf{dicht gepackt} werden, wodurch die \textbf{Beschichtung} (z.B. Polysilizium) \textbf{auf beiden Seiten bei Temperaturen zwischen 400 
und 1000°C} erfolgt. Allerdings werden \textbf{auch die Reaktorwände beschichtet}, was durch \textbf{Partikelbildung} zu Problemen führen 
kann \cite{keplinger2024CVD}.

\begin{figure}[htb!]
    \centering
    \includegraphics[width=.65\textwidth]{images/Hot-wall-reactor.png} % Pfad und Dateiname des Bildes angeben
    \captionsetup{labelfont=bf} % Setzt die Bildnummer fett
    \caption{Schematische Anordnung beim Hot-wall-Reaktor \cite{keplinger2024CVD}.}
    \label{fig:Hot-Wall-Reaktor}
\end{figure}

\vspace{0.0em}
\paragraph{\uline{Cold-Wall-Reaktor:}} Der \textbf{Waferhalter} wird hier entweder \textbf{resistiv, induktiv oder durch Strahlungslampen 
erhitzt}, während die \textbf{Reaktorwände kalt} bleiben (\autoref{fig:Cold-Wall-Reaktor}). Dies führt zu \textbf{geringeren Ablagerungen an 
den Wänden} und dadurch zu \textbf{weniger Partikelkontaminationen} im Vergleich zum Hot-Wall-Reaktor \cite{keplinger2024CVD}.

\begin{figure}[htb!]
    \centering
    \includegraphics[width=.60\textwidth]{images/Cold-wall-reactor.png} % Pfad und Dateiname des Bildes angeben
    \captionsetup{labelfont=bf} % Setzt die Bildnummer fett
    \caption{Schematischer Aufbau des Cold-wall-Reaktors \cite{keplinger2024CVD}.}
    \label{fig:Cold-Wall-Reaktor}
\end{figure}

\vspace{1em}
\subsubsection{Druck} % Druck
Schließlich ist auch der Druck zu berücksichtigen, \textbf{unter dem die Reaktionen stattfinden} \cite{keplinger2024CVD}.

\vspace{0.0em}
\paragraph{\uline{Atmospheric Pressure CVD (APCVD):}} Bei diesem Verfahren erfolgt die \textbf{Abscheidung einer Oxidschicht unter 
Atmosphärendruck} und ohne Evakuierung der Anlage bei einer Temperatur von z.B. \textbf{400°C}. Die Oxidschicht bildet sich über \textbf{folgende chemische 
Reaktionen}\footnote{Bei diesen Reaktionen handelt es sich um Oxidationsreaktionen, bei denen aus dem Ausgangsstoff Silan (\(\mathrm{SiH}_4\)) und Sauerstoff 
(\(\mathrm{O}_2\)) Siliziumdioxid (\(\mathrm{SiO}_2\)) gebildet wird \cite{keplinger2024CVD}. Dies geschieht über zwei mögliche Reaktionswege: 1. 
\(\mathrm{SiH}_4 + 2 \, \mathrm{O}_2 \rightarrow \mathrm{SiO}_2 + 2 \, \mathrm{H}_2\mathrm{O}\): Silan reagiert mit zwei Molekülen Sauerstoff und bildet 
Siliziumdioxid sowie Wasser (\(\mathrm{H}_2\mathrm{O}\)) als Nebenprodukt. 2. \(\mathrm{SiH}_4 + \mathrm{O}_2 \rightarrow \mathrm{SiO}_2 + 2 \, \mathrm{H}_2\): 
Hier reagiert Silan mit einem Molekül Sauerstoff, wodurch ebenfalls Siliziumdioxid entsteht, jedoch ist das Nebenprodukt Wasserstoffgas (\(\mathrm{H}_2\)). Beide 
Reaktionen finden bei typischen CVD-Prozesstemperaturen statt und führen zur Bildung einer Siliziumdioxidschicht (\(\mathrm{SiO}_2\)) auf dem Substrat. Die Wahl 
der Reaktion hängt von den Prozessbedingungen wie Temperatur und Sauerstoffkonzentration ab \cite{keplinger2024CVD}.}:
$$
\mathrm{SiH}_4 + 2 \, \mathrm{O}_2 \rightarrow \mathrm{SiO}_2 + 2 \, \mathrm{H}_2\mathrm{O}
$$
$$
\mathrm{SiH}_4 + \mathrm{O}_2 \rightarrow \mathrm{SiO}_2 + 2 \, \mathrm{H}_2
$$
Aufgrund der \textbf{niedrigen Prozesstemperatur} werden die \textbf{Seitenwände (Flanken) der Strukturen weniger stark beschichtet als die horizontalen Bereiche}, 
was zu einer \textbf{geringen Konformität}\footnote{Die Konformität beschreibt das Verhältnis der Beschichtungsdicke an den Seitenwänden zur Beschichtungsdicke in 
horizontalen Bereichen \cite{keplinger2024CVD}.} führt \cite{keplinger2024CVD}.
    
\vspace{0.0em}
\paragraph{\uline{Low Pressure CVD (LPCVD):}} Werden die \textbf{Schichten in} einem \textbf{evakuierten Reaktor abgeschieden}, handelt es sich um Low Pressure 
CVD. Die \textbf{Drücke} liegen typischerweise im Bereich von \textbf{10 bis 100 Pa}. Ein häufiges \textbf{Anwendungsbeispiel} ist die \textbf{Abscheidung dünner 
Schichten aus Polysilizium oder Siliziumnitrid}. Diese Schichten weisen eine \textbf{nahezu perfekte Konformität} auf, d.h., die \textbf{vertikalen Schichten in 
Kanälen haben fast die gleiche Dicke wie die horizontalen Flächen}, mit einer \textbf{Konformität von bis zu 0,98} \cite{keplinger2024CVD}.

\vspace{1em}
\subsection{Atomlagenabscheidung} % Atomlagenabscheidung

\vspace{1em}
\section{Lithographie} % Lithographie
This sections teaches you some more stuff ...

\thispagestyle{empty}
\newpage
\bibliographystyle{ieeetr}  % IEEE-Standard für das Literaturverzeichnis
\bibliography{references}  % Ersetze "your_bib_file" durch den tatsächlichen Namen der .bib-Datei ohne die Endung .bib

\end{document}