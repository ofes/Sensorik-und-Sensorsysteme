\documentclass{article} % Dokumenttyp "article" für Artikel

\usepackage[utf8]{inputenc}  % UTF-8-Zeichencodierung
\usepackage[T1]{fontenc}     % Europäische Zeichenunterstützung
\usepackage[ngerman]{babel}  % Deutsche Sprachunterstützung
\usepackage{graphicx} % Am Anfang des Dokuments einfügen, um Grafikbefehle zu laden
\usepackage{caption}
\usepackage[colorlinks=true, linkcolor=blue]{hyperref}
\hypersetup{citecolor = blue}
\usepackage{sectsty}
\usepackage{cite}

\usepackage[a4paper, left=3cm, right=3cm, top=2.5cm, bottom=2.5cm]{geometry} % A4-Format mit definierten Rändern

\setlength{\footskip}{2cm}       % Abstand zwischen Text und Fußnotenlinie
\setlength{\skip\footins}{1cm}    % Abstand zwischen Fußnotenlinie und -text
\interfootnotelinepenalty = 10000   % Verhindert das Teilen von Fußnoten über Seiten
\setlength{\parindent}{0.75em} % Setzt die Einrückung auf 1em
\sectionfont{\LARGE}      % Setzt \section auf \Large (eine Stufe größer als Standard)
\subsectionfont{\Large}   % Setzt \subsection auf \large (ebenfalls eine Stufe größer)

\subsubsectionfont{\large}

\title{Sensorik und Sensorsysteme (TU Wien)}
\author{Omar Filip El Sendiouny \\ \texttt{e1616023@student.tuwien.ac.at}}
\vfill
\date{\today}

\begin{document}

\maketitle
\thispagestyle{empty} % Entfernt die Seitenzahl von der Titelseite
\newpage  

\tableofcontents
\thispagestyle{empty}
\newpage

\pagenumbering{arabic} % Beginnt die Nummerierung ab hier

\section*{Einleitung} % Einleitung
Was beigebracht wird...

\vspace{1em}
\section{Pysical Vapour Deposition} % PVD
PVD (Physical Vapour Deposition) umfasst alle \textbf{physikalischen Verfahren zur Abscheidung dünner Schichten aus der Dampfphase}.
Dabei werden die Ausgangsmaterialien der Schichten durch physikalische Prozesse wie Erhitzen in die Dampfphase überführt und anschließend 
auf dem Zielobjekt abgeschieden. Die \textbf{typischen Schichtdicken} liegen \textbf{zwischen 1 nm und 1 µm} und können 
\textbf{in Ausnahmefällen bis zu 10 µm} erreichen \cite{keplinger2024}.

\vspace{1em}
\subsection{Begriffsklärung für PVD} % PVD Verfahren
\begin{itemize}
    \item \textbf{Monolage}: Eine Monolage (auch Monoschicht genannt) beschreibt eine Schicht, die nur aus einer einzelnen Lage von Atomen oder 
    Molekülen besteht \cite{kittel2004}.
    \item \textbf{Freie Weglänge}: Der Abstand, den die Atome oder Moleküle in der Dampfphase von der Verdampfungsquelle bis zum Substrat 
    zurücklegen müssen, bevor sie auf diesem abgeschieden werden \cite{kittel2004}.
    \item \textbf{Verdampfen}: Bezeichnet den Übergang eines Stoffes vom flüssigen in den gasförmigen Zustand \cite{kittel2004}.
    \item \textbf{Sublimieren}: Beschreibt den direkten Übergang eines Stoffes vom festen in den gasförmigen Zustand, ohne eine flüssige Phase 
    zu durchlaufen \cite{kittel2004}.
    \item \textbf{Rezipient}: Vakuumkammer \cite{keplinger2024, ohring2002}.
    \item \textbf{Substrat}: Trägerschicht in der Mikroelektronik und Halbleiterfertigung. Der Begriff Substrat bezieht sich allgemein auf das 
    Material oder die Oberfläche, auf der ein Herstellungs- oder Beschichtungsprozess stattfindet \cite{ohring2002}.
    \item \textbf{Wafer}: Dünne Scheibe aus einem Halbleitermaterial, wie z.B. Silizium, und dient als Substrat \cite{sze2006}.
    \item \textbf{Tiegel}: Ein Behälter, in dem das Aufdampfmaterial lokal geschmolzen wird, häufig aus Materialien wie Graphit, Keramik oder 
    Wolfram, die hohen Temperaturen standhalten \cite{smith1995, ohring2002}. Beim Elektronenstrahlverdampfen (E-Beam-Verfahren) wird das 
    Aufdampfmaterial im Tiegel erhitzt und verdampft, ohne dass der gesamte Tiegel erhitzt werden muss \cite{smith1995}. Siehe \autoref{fig:Tiegel und Schiffchen} b), c) und d).
    \item \textbf{Schiffchen}: Ein kleiner, meist bootförmiger Behälter (daher der Name), der ebenfalls zur Aufnahme des Aufdampfmaterials dient, 
    jedoch direkt beheizt wird, häufig durch Widerstandsheizung. Schiffchen bestehen oft aus Materialien wie Molybdän oder Wolfram, da sie hohe 
    Temperaturen aushalten können \cite{mattox1998, ohring2002}. Siehe \autoref{fig:PVD Aufdampfanlage} und \autoref{fig:Tiegel und Schiffchen} a).
\end{itemize}

\vspace{1em}
\subsection{PVD-Verfahren} % PVD Verfahren
Die verschiedenen PVD-Verfahren \textbf{unterscheiden sich} hauptsächlich \textbf{in der} Methode zur \textbf{Erzeugung des Dampfstrahls} 
\cite{keplinger2024}:
\begin{enumerate}
    \item \textbf{Thermisches Verdampfen}
    \item \textbf{Sputtern (Zerstäubung)}
    \item \textbf{Laserstrahlverdampfen (pulsed laser deposition, PLD)}
    \item \textbf{Molekularbeamepitaxie (MBE)}
\end{enumerate}

\vspace{1em}
\subsection{Thermisches Verdampfen (Aufdampfen)} % PVD Aufdampfen
Das thermische Aufdampfen gehört zu den ältesten Methoden zur Herstellung dünner Schichten (\autoref{fig:PVD Aufdampfanlage}). Dabei wird das 
\textbf{Material der Aufdampfquelle auf} eine Temperatur von \textbf{500 bis 3000 °C erhitzt}, bis es einen ausreichend hohen 
Dampfdruck\footnote{Ein hoher Dampfdruck ist beim Aufdampfen entscheidend, da er sicherstellt, dass genügend Atome oder Moleküle des Materials 
in die Gasphase übertreten und eine dampfartige Wolke bilden. Diese Wolke ermöglicht eine ausreichende Partikelzufuhr zum Zielsubstrat, wodurch 
sich eine gleichmäßige, dünne Schicht bildet. Bei zu geringem Dampfdruck wären nur wenige Teilchen verdampft, was zu einer ungleichmäßigen und 
ineffizienten Beschichtung führen würde.} entwickelt \cite{keplinger2024}.

\vspace{1em}

Meist schmilzt das Material dabei, doch einige Stoffe, wie beispielsweise Chrom, erreichen bereits vor dem Schmelzpunkt einen ausreichend hohen 
Dampfdruck und sublimieren, indem sie direkt aus dem festen Zustand verdampfen \cite{keplinger2024}.

\vspace{1em}

Der \textbf{entstehende Dampf} breitet sich aus und \textbf{kondensiert an} allen \textbf{kälteren Oberflächen} in der Vakuumkammer, 
\textbf{einschließlich des Substrats}. Der Prozess findet im \textbf{Vakuum bei Drücken} von typischerweise 
\textbf{unter $\mathbf{10^{-6}}$ mbar} statt \cite{keplinger2024}.

\vspace{1em}

Dabei ist die \textbf{freie Weglänge der Teilchen deutlich größer als} die \textbf{Distanz zwischen} der \textbf{Aufdampfquelle und} dem 
\textbf{Substrat}. Dies führt dazu, dass die \textbf{Teilchen nur selten mit} dem \textbf{Restgas kollidieren und sich geradlinig von der Quelle 
zum Substrat bewegen} \cite{keplinger2024}.

\vspace{1em}

Die \textbf{kinetische Energie der Teilchen, die auf} der \textbf{Substratoberfläche kondensieren}, beträgt etwa \textbf{0,1 bis 0,5 eV}, was im 
Vergleich zum Sputtern (mit 0 bis 100 eV) \textbf{sehr gering} ist \cite{keplinger2024}.

\vspace{2em}

\begin{figure}[ht]
    \centering
    \includegraphics[width=\textwidth]{images/PVD Aufdampfanlage.png} % Pfad und Dateiname des Bildes angeben
    \captionsetup{labelfont=bf} % Setzt die Bildnummer fett
    \caption{Schematischer Aufbau einer PVD Aufdampfanlage \cite{keplinger2024}}
    \label{fig:PVD Aufdampfanlage}
\end{figure}

\vspace{1em}
\subsubsection{Funktionsweise}
Basierend auf \autoref{fig:PVD Aufdampfanlage} wird im Folgenden die Funktionsweise dieses Verfahrens erläutert \cite{keplinger2024}:

\begin{enumerate}
    \item \textbf{Vakuum}: Im Rezipient wird mit dem Pumpsystem ein Vakuum erzeugt.
    \item \textbf{Aufheizen}: Das Aufdampfmaterial wird auf eine Temperatur von 500 bis 3000 °C erhitzt.
    \item \textbf{Öffnen des Shutters}: Sobald ein ausreichend hoher Dampfdruck erreicht ist, wird der Shutter weggeschwenkt, um die Bedampfung 
    des Substrats zu ermöglichen. Wird der Shutter zu früh geöffnet oder ist er während des Aufheizens nicht vorhanden, kann es zu einer 
    ungleichmäßigen Abscheidung kommen bzw. es können unerwünschte Oxide des Aufdampfmaterials auf das Substrat abgeschieden werden. Der Weg der
     Teilchen zum Substrat wird daher durch ein Schirmblech (Shutter) blockiert.
    \item \textbf{Abscheidung am Substrat}: Das Aufdampfmaterial kondensiert nun mit einer konstanten Rate sowohl auf dem Substrat als auch auf 
    dem Schwingquarz. Der Schwingquarz dient zur Messung der Schichtdicke, indem Änderungen seiner Resonanzfrequenz erfasst werden.
\end{enumerate}

\vspace{1em}
\subsubsection{Schichtdickenmessung}
Zur Messung der Schichtdicke wird ein \textbf{piezoelektrisches Schwingquarzplättchen} verwendet, das \textbf{nahe\footnote{Da Schwingquarz und Substrat 
unterschiedliche Abstände zur Aufdampfquelle haben und sich zudem unterschiedlich nah an der Achse des Aufdampfstrahls befinden, ergeben sich unterschiedliche Aufdampfraten. Diese Abweichungen 
werden durch einen experimentell zu bestimmenden Geometriefaktor (Tooling-Factor) korrigiert.} am Substrat positioniert} ist. Durch Anlegen einer Wechselspannung wird 
das Quarzplättchen in eine (Scher-)Schwingung versetzt. Wenn das \textbf{Aufdampfmaterial auf} dem \textbf{Quarz kondensiert} und seine Masse zunimmt, \textbf{sinkt} die \textbf{Resonanzfrequenz} des 
Quarzes. Diese Abnahme der Frequenz ist \textbf{proportional zur Dicke der abgeschiedenen Schicht} und dient als Messgrundlage. Die Resonanzfrequenz startet typischerweise bei 
etwa 5 MHz und sollte, wenn sie auf 4 MHz fällt, zum Austausch des Quarzplättchens führen, um präzise Messwerte sicherzustellen \cite{keplinger2024}.

\vspace{1em}
\subsubsection{Erwärmung mittels Elektronenstrahlquelle (E-Beam-Verfahren)}

Bei dieser Methode wird ein \textbf{Elektronenstrahl gezielt auf einen kleinen Bereich des Aufdampfmaterials fokussiert}, sodass das Material 
\textbf{lokal aufgeschmolzen} wird, während der Großteil fest bleibt (\autoref{fig:Tiegel und Schiffchen} b), c) und d)). Der \textbf{Tiegel} selbst wird dabei 
\textbf{nicht auf hohe Temperaturen erhitzt}, was das \textbf{Risiko einer Kontamination durch} das \textbf{Tiegelmaterial verringert und höhere Arbeitstemperaturen} als bei 
herkömmlichen Quellen, wie Schiffchenquellen, ermöglicht (\autoref{fig:Tiegel und Schiffchen} a)). Um eine \textbf{effektive Wärmeableitung} sicherzustellen, 
besteht der \textbf{Tiegel aus Kupfer} und wird direkt \textbf{vom Kühlwasser durchströmt}. Das \textbf{Aufdampfmaterial} muss \textbf{leitfähig} sein, da sich 
andernfalls elektrische Ladungen ansammeln könnten, die den Elektronenstrahl ablenken könnten \cite{keplinger2024}.

\vspace{1em}

Ein \textbf{Permanentmagnet in der Nähe der Elektronenstrahlquelle lenkt} den \textbf{Strahl um 270° ab, sodass} die \textbf{Quelle} selbst \textbf{nicht beschichtet wird} 
(\autoref{fig:Tiegel und Schiffchen} b) und c)). Die \textbf{Steuerung des Systems} ermöglicht zudem ein \textbf{Fokussieren oder Defokussieren des Strahls}, 
eine \textbf{präzise Positionierung über Spulen} sowie eine \textbf{kontrollierte Leistungsanpassung} bis zu 5 kW. Durch ein leichtes Wobbeln des Strahls kann 
ein größerer Bereich des Aufdampfmaterials überstrichen werden, was zu einem gleichmäßigen Erhitzen führt \cite{keplinger2024}.

\begin{figure}[ht]
    \centering
    \includegraphics[width=\textwidth]{images/Tiegel und Schiffchen.png} % Pfad und Dateiname des Bildes angeben
    \captionsetup{labelfont=bf} % Setzt die Bildnummer fett
    \caption{%
    a) Aufdampfschiffchen, mit Aluminiumoxid (milchige Bereiche) beschichtet, um das Benetzen durch das Aufdampfmaterial zu minimieren. 
    b) Schematischer Aufbau einer Elektronenstrahlquelle: S...Spule, A...Anode, K...Kathode, W...Wehneltzylinder, T...Tiegel mit Aufdampfmaterial, L...Wasserkühlung, M...Permanentmagnet.  
    c) Elektronenstrahlquelle im Inneren der Vakuumkammer (Rezipient). 
    d) Vierlochtiegel für die Elektronenstrahlquelle, der Beschichtungen aus bis zu vier verschiedenen Materialien ermöglicht \cite{keplinger2024}.
}
    \label{fig:Tiegel und Schiffchen}
\end{figure}

\vspace{1em}
\section{Chemical Vapour Deposition} % CVD
Hier kommt der Hauptinhalt. \LaTeX{} ist sehr leistungsfähig für die Erstellung wissenschaftlicher Dokumente.

\vspace{1em}
\section{Lithographie} % Lithographie
This sections teaches you some more stuff ...

\newpage
\bibliographystyle{ieeetr}  % IEEE-Standard für das Literaturverzeichnis
\bibliography{references}  % Ersetze "your_bib_file" durch den tatsächlichen Namen der .bib-Datei ohne die Endung .bib

\end{document}